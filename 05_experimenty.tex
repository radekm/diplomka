\chapter{Experimenty}

V této kapitole srovnáme program \crossbow{} s jinými programy
pro hledání modelů. Srovnání budeme provádět na splnitelných problémech
s \texttt{cnf} formulemi z~kolekce TPTP 6.1.0 \cite{sutcliffe2009tptp}.

Programy budeme spouštět na počítači s procesorem
Intel Core i5-3570, ope\-račním systémem openSUSE 13.1 a
verzí Linuxu 4.1.2. Na vyřešení jednoho problému bude mít
každý program k dispozici 11 GiB RAM a 2 minuty procesorového
času\footnote{Procesorový čas na jednotlivých jádrech se sčítá. To
znamená, že program může vyčerpat časový limit například i za jednu minutu,
pokud bude po dobu jedné minuty používat dvě jádra procesoru.}.

Program \crossbow{} budeme srovnávat s následujícími programy:
\begin{itemize}
\item Mace4\footnote{Program Mace4 neumí hledat
  konečné modely s velikostí domény 1.} \cite{mccune03mace4},
\item Paradox \cite{paradox} (program byl upraven,
  aby ho bylo možné přeložit GHC 7.6.3),
\item iProver 1.0 \cite{iprover}.
\end{itemize}
Ke kompilaci programů použijeme kompilátory GCC 4.8.1,
OCaml 4.02.2 a GHC 7.6.3 s knihovnami z Haskell Platform 2013.2.

Pro překlad programu \crossbow{} použijeme následující programy
a knihovny (dostupné z repozitáře OPAM \cite{opam}):
\begin{itemize}
\item batteries 2.3.1,
\item cmdliner 0.9.7,
\item menhir 20141215,
\item ocamlfind 1.5.5,
\item oclock 0.4.0,
\item omake 0.9.8.6-0.rc1,
\item ounit 2.0.0,
\item pprint 20140424,
\item ppx\_tools 0.99.2,
\item sexplib 112.24.01,
\item sqlite 3.2.0.9,
\item tptp 0.3.1,
\item zarith 1.3.
\end{itemize}

Bohužel v našem srovnání nemáme hledače modelů
založené na metodách Model Evolution
a SGGS \cite{bonacina2015}.
Implementace E-Darwin 1.4 \cite{edarwin} a MELIA 0.1.3 \cite{melia}
metody Model Evolution se nám totiž nepodařilo zprovoznit
a o žádné implementaci SGGS nevíme.

\bigskip
Nyní budeme prezentovat výsledky našeho srovnání.
Začneme tabulkou, jenž obsahuje počty vyřešených problémů.
V prvním sloupci tabulky jsou názvy kategorií problémů
z kolekce TPTP. Kategorie, kde bylo ostře méně než 10
splnitelných problémů, byly sloučeny pod položku \uv{Ostatní}.
První řádek tabulky obsahuje názvy konfigurací\footnote{Soubor
\texttt{mereni/run\_all\_provers} na přiloženém DVD obsahuje
přesné parametry, s nimiž byly hledače modelů spouštěny.}
hledačů modelů. Pro cizí programy jsou názvy konfigurací:
\begin{itemize}
\item Mace. Program Mace4.
\item Paradox. Program Paradox.
\item iProver. Program iProver, hledání modelů.
\item iProver/Fin. Program iProver, hledání konečných modelů.
\end{itemize}
Ostatní názvy konfigurací značí program \crossbow{}.
Použitý řešič se pozná z prefixu názvu konfigurace (CMSat značí
CryptoMiniSat). Pokud název konfigurace končí řetězcem \uv{+E},
znamená to, že byl po dobu 5 sekund spuštěn dokazovač E za účelem
generování redundantních klauzulí. V opačném případě nebyl dokazovač E
spuštěn vůbec.

\newcommand\summary[1]{\noindent
\begin{minipage}{1.0\textwidth}
\begin{center}
\include{#1}
\end{center}
\end{minipage}}

\newpage

Tabulka s počty vyřešených problémů:

\summary{sum_counts}

Jak je vidět, nejlépe si vedl program \crossbow{} s řešičem MiniSat
a se zapnutým generováním redundantních klauzulí.
Všimněme si, že program \crossbow{} vý\-raz\-něji zaostává
pouze na problémech z kategorií NLP a SYN.
Důvodem je velikost explicitně reprezentovaných modelů. Modely z těchto
kategorií mají při vypsání programem \crossbow{} do formátu TPTP
velikost i několik gigabajtů. Například Josat našel
v kategorii NLP více modelů než iProver, jenže
modely problémů 75, 82-93 nestihl celé vypsat.
Tyto problémy také ukazují výhody řešiče Josat --
řešiči MiniSat došla paměť, zatímco řešič Josat dokázal modely najít.
Domníváme se, že výhody řešiče Josat by se projevily ještě více,
kdyby modely problémů měly větší domény.

Zklamáním je naopak řešič Gecode. Je pomalejší a spotřebovává více
paměti než SAT řešiče. Důvodem jsou pravděpodobně použité podmínky
--  SAT řešiče pracují s klauzulemi efektivněji než Gecode.

\newpage

Nyní se podíváme na graf ukazující vývoj počtu vyřešených problémů
v čase:

\summary{sum_plot}

Z grafu je patrné, že iProver po jedné minutě zkouší jinou metodu.
Od osmdesáté sekundy se počet vyřešených problémů mění pouze velmi
mírně -- nezdá se, že by zvýšení časového limitu dramaticky změnilo
výsledky.

\newpage

Gecode podporuje jak ploché, tak i neploché klauzule.
Následující tabulka ukazuje, že kvůli kategorii LAT\footnote{Důvodem
jsou klauzule, jejichž zploštění způsobí výrazný nárůst počtu proměnných.}
je lepší nezplošťovat:

\summary{sum_countsF}

Gecode+E označuje konfiguraci bez zplošťování,
Gecode-F+E označuje konfiguraci se zplošťováním.

\newpage

Některé modifikace lze v programu \crossbow{} deaktivovat.
Díky tomu můžeme zkoumat, jaký přínos tyto modifikace mají.
Zde jsou výsledky pro definice termů:

\summary{sum_countsNDef}

Řetězec NDef v názvu konfigurace značí, že
definice termů nebyly použity. Pro řešič MiniSat
je největší přínos v kategorii LAT, důvodem
jsou opět klauzule, jejichž zploštění způsobí
výrazný nárůst počtu proměnných.
Díky opakovanému používání pomocných proměnných zavedených
pro podmínky \Linear{}, \Element{}, \Eq{}
a faktu, že Gecode+E nezplošťuje, nemá řešič Gecode
s kategorií LAT obtíže, i když jsou definice termů vypnuté.

\newpage

Ukazuje se, že detekce axiomů komutativity,
grup, kvazigrup a involutorních funkcí nemá prakticky
žádný přínos:

\summary{sum_countsNDet}

Řetězec NDet v názvu konfigurace značí, že
detekce axiomů nebyla použita.

\newpage

Odzplošťování nemá na problémy z kolekce TPTP vůbec žádný efekt:

\summary{sum_countsNU}

Řetězec NU v názvu konfigurace značí, že
odzplošťování nebylo použito.

\newpage

Na rozdíl od předešlých dvou modifikací je přínos
rozdělování klauzulí značný:

\summary{sum_countsNS}

Řetězec NS v názvu konfigurace značí, že
rozdělování klauzulí nebylo použito.
Jelikož řešiči Gecode dáváme neploché klauzule,
není přínos rozdělování klauzulí tak výrazný.

% Zplošťování přidává nové proměnné -- bez zplošťování
% se rozdělování klauzulí nemůže tolik projevit.

\newpage

Nakonec se podíváme, jaký přínos má inference sort:

\summary{sum_countsNSI}

Řetězec NSI v názvu konfigurace značí, že
inference sort nebyla použita.
Inference sort má výrazný přínos
pouze v kategorii NLP pro řešič Gecode.
Bohužel neznáme přesné důvody\footnote{Jelikož
problémy z kategorie NLP obsahují značné množství
funkčních symbolů, je možným důvodem
velikost podmínek přidaných LNH pro Gecode.
S větším množstvím sort (u některých problémů z kategorie NLP
lze inferovat stovky sort)
klesá velikost těchto podmínek.},
proč tomu tak je.

% Alternativně: Může být přínos způsoben blokováním LNH?
% Tj. LNH se u některých sort vůbec neprovádí
% kvůli předpokladu, že všechny hodnoty z $D_\sort$ jsou nepoužité.
