\chapter{Experimenty}

V této kapitole srovnáme program \crossbow{} s jinými programy
pro hledání modelů. Srovnání budeme provádět na splnitelných problémech
s \texttt{cnf} formulemi z~kolekce TPTP 6.1.0 \cite{sutcliffe2009tptp}.

Programy budeme spouštět na počítači s procesorem
Intel Core i5-3570, ope\-račním systémem openSUSE 13.1 a
verzí Linuxu 4.1.2. Na vyřešení jednoho problému bude mít
každý program k dispozici 11 GiB RAM a 2 minuty procesorového
času\footnote{Procesorový čas na jednotlivých jádrech se sčítá. To
znamená, že program může vyčerpat časový limit například i za jednu minutu,
pokud bude po dobu jedné minuty používat dvě jádra procesoru.}.

Program \crossbow{} budeme srovnávat s následujícími programy:
\begin{itemize}
\item Mace 4 \cite{mccune03mace4},
\item Paradox \cite{paradox} (program byl upraven,
  aby ho bylo možné přeložit GHC 7.6.3),
\item iProver 1.0 \cite{iprover}.
\end{itemize}
Ke kompilaci programů použijeme kompilátory GCC 4.8.1,
OCaml 4.02.2 a GHC 7.6.3 s knihovnami z Haskell Platform 2013.2.

Pro překlad programu \crossbow{} použijeme následující programy
a knihovny (dostupné z repozitáře OPAM \cite{opam}):
\begin{itemize}
\item batteries 2.3.1,
\item cmdliner 0.9.7,
\item menhir 20141215,
\item ocamlfind 1.5.5,
\item oclock 0.4.0,
\item omake 0.9.8.6-0.rc1,
\item ounit 2.0.0,
\item pprint 20140424,
\item ppx\_tools 0.99.2,
\item sexplib 112.24.01,
\item sqlite 3.2.0.9,
\item tptp 0.3.1,
\item zarith 1.3.
\end{itemize}

Bohužel v našem srovnání nemáme hledače modelů
založené na metodách Model Evolution
a SGGS \cite{bonacina2015}.
Implementace E-Darwin 1.4 \cite{edarwin} a MELIA 0.1.3 \cite{melia}
metody Model Evolution se nám totiž nepodařilo zprovoznit
a o žádné implementaci SGGS nevíme.
