\chapter{Základní definice}

Hlavním účelem této kapitoly je sjednotit základní definice a značení.
Kromě toho si v této kapitole popíšeme,
co přesně má náš program dělat -- co přesně je vstupem a výstupem programu.
Předpokládáme, že čtenář již zná základy klasické logiky prvního
řádu a základní techniky pro řešení omezujících podmínek.
Chybějící znalosti logiky lze doplnit z
\cite{enderton2001logic} a omezujících podmínek z
\cite{dechter2003constraints}.

Budeme pracovat ve vícesortové klasické logice prvního řádu.
V celém textu budeme
používat pevně danou signaturu, která se skládá z nekonečné množiny sort
$\sorts$, množiny funkčních symbolů $\funcs$, množiny predikátových symbolů
$\preds$ a funkce $\arity: \funcs \cup \preds \to \sorts^\star$,
kde $\sorts^\star$ značí množinu konečných posloupností sort.
Funkce $\arity$ přiřazuje každému funkčnímu symbolu neprázdnou posloupnost
sort $\langle \sort_1 \comdots \sort_n, \sort \rangle$,
kde $n \ge 0$ je počet argumentů,
$\sort_1 \comdots \sort_n$ jsou sorty argumentů a $\sort$ je sorta výsledku.
Funkce $\arity$ přiřazuje každému predikátovému symbolu posloupnost
sort $\langle \sort_1 \comdots \sort_n \rangle$, kde $n \ge 0$
je počet argumentů a $\sort_1 \comdots \sort_n$ jsou sorty argumentů.

Pro každou neprázdnou posloupnost sort $a$ obsahuje $\funcs$ nekonečně
mnoho symbolů s aritou $a$. Pro každou posloupnost sort $a$ obsahuje
$\preds$ nekonečně mnoho symbolů s aritou $a$.

Dále je každé sortě $\sort$ přiřazena nekonečná množina proměnných
$\vars_\sort$ a tyto množiny jsou navzájem disjunktní.
$\vars$ je množina všech proměnných,
tj. $\bigcup_{\sort \in \sorts} \vars_\sort$. Množiny $\natZ$, $\sorts$, $\funcs$,
$\preds$, $\vars$ jsou navzájem disjunktní.

Termy sorty $\sort$ jsou definovány induktivně:

\begin{itemize}
\item proměnná sorty $\sort$ je term sorty $\sort$,
\item je-li $\func$ funkční symbol s aritou
  $\langle \sort_1 \comdots \sort_n, \sort \rangle$
  a $\term_i$ term sorty $\sort_i$ pro každé $i \in \{ 1 \comdots n \}$, pak
  $\func(\term_1 \comdots \term_n)$ je term sorty $\sort$.
\end{itemize}

Atomy jsou definovány induktivně:

\begin{itemize}
\item jsou-li $\termS$ a $\term$ termy stejné sorty,
  pak $\termS \approx \term$ je atom,
\item je-li $\pred$ predikátový symbol
  s aritou $\langle \sort_1 \comdots \sort_n \rangle$
  a $\term_i$ term sorty $\sort_i$ pro každé $i \in \{ 1 \comdots n \}$, pak
  $\pred(\term_1 \comdots \term_n)$ je atom.
\end{itemize}

Literál je atom nebo jeho negace. Klauzule je disjunkce literálů.
Řekneme, že sorta $\sort$ se vyskytuje v termu, atomu, literálu, klauzuli,
množině klauzulí, pokud se tam vyskytuje proměnná sorty $\sort$ nebo
funkční či predikátový symbol, kde $\sort$ je sorta argumentu nebo výsledku.

% POZN: Nedefinujeme, ale používáme "sorta se vyskytuje v aritě",
% "term, atom, literál, klauzule, množina klauzulí, arita obsahuje sortu".
% Předpokládáme, že význam je čtenáři zřejmý.

Sorty budeme značit písmenem $\sort$, funkční symboly písmeny
$\funcC$, $\func$ a $\funcG$, predikáty písmenem $\pred$
a proměnné písmeny $\var$, $\varY$ a $\varZ$.
Termy budeme značit písmeny $\termS$ a $\term$, atomy písmenem $\atom$,
literály písmenem $\lit$, klauzule písmenem $\clause$ a množiny
klauzulí písmenem $\clauses$.

Nyní zadefinujeme interpretaci.
Mějme množiny $\sorts' \subseteq \sorts$, $\funcs' \subseteq \funcs$
a $\preds' \subseteq \preds$, kde $\sorts'$ obsahuje alespoň sorty
vyskytující se v aritách symbolů z $\funcs'$ a $\preds'$.
Funkce $\interp$ definovaná na množině $\sorts' \cup \funcs' \cup \preds'$
je interpretace, jestliže

\begin{itemize}
\item každé sortě z $\sorts'$ přiřadí neprázdnou množinu (doménu),
\item každému funkčnímu symbolu $\func \in \funcs'$ s aritou
  $\langle \sort_1 \comdots \sort_n, \sort \rangle$ přiřadí funkci
  $\func^\interp: \interp(\sort_1) \timdots \interp(\sort_n) \to
  \interp(\sort)$
\item a každému predikátovému symbolu $\pred \in \preds'$ s aritou
  $\langle \sort_1 \comdots \sort_n \rangle$ přiřadí relaci
  $\pred^\interp \subseteq \interp(\sort_1) \timdots \interp(\sort_n)$.
\end{itemize}

Interpretace je konečná, jsou-li $\sorts'$, $\funcs'$ a $\preds'$
konečné a je-li každá doména konečná.
Konečná interpretace je číselná, pokud je každá doména
počáteční úsek $\natZ$, jinak řečeno, pokud je každá doména rovna
$\{ 0 \comdots n - 1 \}$, kde $n$ je velikost domény.
Pokud není řečeno jinak, bude
$\interp$ v dalším textu označovat interpretaci definovanou
na množině $\sorts' \cup \funcs' \cup \preds'$.

Buď $\interp$ interpretace a $\beta$ funkce definovaná na množině
$\bigcup_{\sort \in \sorts'} \vars_\sort$. Funkci $\beta$ nazveme ohodnocením
proměnných, pokud každé proměnné $\var \in \vars_\sort$ přiřazuje
prvek domény $\sort$.

Je-li $R$ term nebo atom nebo literál nebo klauzule, jenž
obsahuje pouze sorty z~$\sorts'$, funkční symboly z $\funcs'$ a
predikátové symboly z $\preds'$, pak
hodnotu $R$ při interpretaci $\interp$ a ohodnocení proměnných
$\beta$ značíme $\interp_\beta(R)$. Pro term $\term$ je $\interp_\beta(\term)$
definováno následovně:
\begin{align*}
\interp_\beta(\var) &= \beta(\var), \\
\interp_\beta\bigl(\func(\term_1 \comdots \term_n)\bigr) &=
  \interp\bigl(\func\bigr)\bigl(\interp_\beta(\term_1) \comdots
    \interp_\beta(\term_n)\bigr).
\end{align*}
Pro literál $\lit$ je $\interp_\beta(\lit)$ definováno takto:
\begin{align*}
\interp_\beta(\termS \approx \term) &=
  \begin{cases}
    1 & \text{pokud } \interp_\beta(\termS) = \interp_\beta(\term), \\
    0 & \text{jinak},
  \end{cases} \\
\interp_\beta\bigl(\pred(\term_1 \comdots \term_n)\bigr) &=
  \begin{cases}
    1 & \text{pokud }
      \bigl(\interp_\beta(\term_1) \comdots \interp_\beta(\term_n)\bigr)
      \in \interp(\pred), \\
    0 & \text{jinak},
  \end{cases} \\
\interp_\beta(\neg \atom) &= 1 - \interp_\beta(\atom).
\end{align*}
A nakonec definice $\interp_\beta(\clause)$ pro klauzuli $\clause$:
\begin{align*}
\interp_\beta(\lit_1 \vee \cdots \vee \lit_n) &=
  \max \bigl\{ 0, \interp_\beta(\lit_1) \comdots \interp_\beta(\lit_n) \bigr\}.
\end{align*}

Skutečnost, že $\interp_\beta(R) = 1$, kde $R$ je atom, literál nebo klauzule,
zapisujeme
\[
\interp_\beta \models R.
\]
Platí-li $\interp_\beta \models R$ pro libovolné $\beta$, říkáme,
že $\interp$ je model $R$, a píšeme
\[
\interp \models R.
\]
Model $\interp$ je konečný resp. číselný, je-li interpretace
$\interp$ konečná resp. číselná.
$\interp$ je model množiny klauzulí $N$,
pokud je $\interp$ model každé klauzule z $N$.

Interpretace $\interp$ a $\interp'$ nad
$\sorts' \cup \funcs' \cup \preds'$ jsou izomorfní
(značíme $\interp \simeq \interp'$), pokud pro každou
sortu $\sort \in \sorts'$ existuje bijekce
$i_\sort: \interp(\sort) \to \interp'(\sort)$, že
\begin{itemize}
\item pro každý funkční symbol $\func \in \funcs'$ s aritou
  $\langle \sort_1 \comdots \sort_n, \sort \rangle$
  a pro každé $d_j \in \interp(\sort_j)$,
  kde $j \in \{ 1 \comdots n \}$, platí
  \[
     i_\sort\bigl(\interp(\func)(d_1 \comdots d_n)\bigr) =
     \interp'\bigl(\func\bigr)\bigl(i_{\sort_1}(d_1) \comdots
       i_{\sort_n}(d_n)\bigr)
  \]
\item a pro každý predikátový symbol $\pred \in \preds'$ s aritou
  $\langle \sort_1 \comdots \sort_n \rangle$
  a pro každé $d_j \in \interp(\sort_j)$,
  kde $j \in \{ 1, \comdots, n \}$, platí
  \[
     (d_1 \comdots d_n) \in \interp(\pred) \iff
     \bigl(i_{\sort_1}(d_1) \comdots i_{\sort_n}(d_n)\bigr) \in \interp'(\pred).
  \]
\end{itemize}

Připomeňme vlastnosti izomorfních interpretací:
\begin{itemize}
\item[(1)] Každá konečná interpretace je izomorfní číselné interpretaci.
\item[(2)] Je-li $\interp \simeq \interp'$ a $\clauses$ množina klauzulí,
  pak $\interp$ je model $\clauses$ právě tehdy,
  když $\interp'$ je model $\clauses$.
\end{itemize}

V tomto okamžiku máme potřebné definice,
abychom specifikovali chování programu -- jaký problém program řeší:

\begin{defn}
Program pracuje ve dvou režimech -- hledání jednoho modelu a
hledání všech neizomorfních modelů. V obou režimech je vstupem programu
sorta $\sort$ a konečná množina klauzulí $\clauses$,
kde se vyskytuje pouze sorta $\sort$.
V obou režimech jsou výstupem konečné modely $\clauses$
s jedinou sortou $\sort$, které
interpretují právě symboly z $\clauses$.

V režimu hledání jednoho modelu jsou navíc vstupem přirozená čísla
$n \leq n'$. Výstupem v tomto režimu je model
s doménou velikosti $m$, kde $n \leq m \leq n'$, pokud existuje.

V režimu hledání všech neizomorfních modelů je
navíc vstupem programu přirozené číslo $n$.
Výstupem v tomto režimu jsou všechny navzájem
neizomorfní modely s doménou velikosti $n$.
\end{defn}

Z (1) a (2) plyne, že pro každý vstup programu existuje výstup
obsahující pouze číselné modely. Aniž by to mělo vliv na úplnost
programu, může program hledat pouze číselné modely.
Navíc číselných interpretací, jenž jsou kandidáty na model,
je pouze konečně mnoho (důvodem je, že
číselných interpretací s jedinou sortou $\sort$ obsahujících právě symboly
z $\clauses$ a doménou velikosti $n$ je pouze konečně mnoho),
díky čemuž lze hledané
modely získat metodou generuj a testuj a stačí k tomu primitivní rekurze.

Zbývající definice se nám budou hodit při detekci izomorfních modelů.
Zobrazení kanonických reprezentantů $\rho$ pro číselné interpretace
je zobrazení z číselných interpretací do číselných interpretací splňující
\begin{itemize}
\item $\rho(\interp) \simeq \interp$
\item a $\interp \simeq \interp' \iff \rho(\interp) = \rho(\interp')$.
\end{itemize}

Orientovaný barevný graf (dále jen graf) je trojice $G = (V, E, c)$, kde
$V$ je konečná množina vrcholů, $E \subseteq V \times V$ je množina hran
a $c : V \to \natZ$ je obarvení vrcholů. Grafy
$G = (V, E, c)$ a $G' = (V, E', c')$ jsou izomorfní (značíme $G \simeq G'$),
pokud existuje bijekce $i : V \to V$, že
\begin{itemize}
\item $c'\bigl(i(v)\bigr) = c(v)$
\item a $(u, v) \in E \iff \bigl(i(u), i(v)\bigr) \in E'$.
\end{itemize}

Zobrazení kanonických reprezentantů $\rho$ pro grafy
je zobrazení z grafů do grafů splňující
\begin{itemize}
\item $\rho(G) \simeq G'$
\item a $G \simeq G' \iff \rho(G) = \rho(G')$.
\end{itemize}


\begin{note}
K definicím.
\begin{itemize}
\item Některé transformace problému zavádějí nové symboly --
  v takovém případě je možné rozšířit signaturu nebo zajistit,
  že stávající signatura obsahuje dostatek nepoužitých symbolů. Domníváme se,
  že druhá možnost, ač méně obvyklá, je přehlednější --
  v celém textu používáme jednu signaturu, která je
  zkonstruována tak, že pro každou aritu vždy existuje
  nepoužitý symbol. Díky tomu, že signatura je pouze jedna,
  není třeba ji zmiňovat v každém tvrzení.
\item Proč je množina sort $\sorts'$ v interpretaci zadána explicitně, proč
  není implicitně určena na základě interpretovaných symbolů?
  Interpretovat pouze sorty obsažené ve funkčních a predikátových
  symbolech nestačí, je třeba interpretovat i sorty proměnných.
\item Proč je sorta na vstupu zadána explicitně, proč není implicitně určena
  z klauzulí na vstupu? Vstupní množina klauzulí nemusí žádnou sortu
  obsahovat (prázdná množina nebo množina s prázdnou klauzulí), v takovém
  případě není
  jednoznačně určeno, jakou sortu má model interpretovat, a číselných
  interpretací, jenž jsou kandidáty na model, je nekonečně mnoho.
\item Vstup i výstup programu jsou jednosortové, proč výklad komplikovat
  více\-sor\-tovou logikou? Některé problémy s jednou sortou
  lze přeformulovat jako rychleji vyřešitelné problémy s více sortami.
\end{itemize}
\end{note}
