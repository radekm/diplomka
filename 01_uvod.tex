\chapter{Úvod}

Cílem této práce je implementovat program pro hledání konečných
modelů v~klasické logice prvního řádu s rovností.
Bohatost logiky prvního řádu hraje důležitou roli při modelování
praktických problémů.
Díky ní se hledače modelů mohou uplatnit nejen při výzkumu
matematických struktur, ale i při verifikaci softwaru a hardwaru,
jako součást běhových prostředí deklarativních programovacích jazyků
anebo v jiných oblastech, kde se dnes používají SAT řešiče a SMT řešiče.

Existuje celá řada metod pro hledání modelů, tato práce se však
zabývá pouze metodou MACE a jejími modifikacemi.
Metoda MACE převádí problém hledání
konečného modelu velikosti $n$ v klasické logice prvního řádu
na problém SAT. Většina modifikací metody MACE se snaží o to, aby
byl výsledný SAT vyřešitelný rychle a s malým množstvím paměti.

Náš program implementuje metodu MACE, její modifikace známé
z programu Paradox a některé nové modifikace,
které, pokud je nám známo, zatím ještě nikdo nezkoušel.
Například kromě převodu na SAT je implementován
převod do jazyka řešiče omezujících podmínek Gecode
nebo přidávání redundantních formulí ke vstupu.

Práce je rozdělena do šesti kapitol. Tato kapitola
stručně představuje téma práce.
Druhá kapitola obsahuje definice základních pojmů,
jenž budeme používat v dalších kapitolách,
a přesnější popis řešeného problému.

Třetí kapitola popisuje existující metody pro hledání modelů.
Důraz je kladen na metody MACE a SEM pro hledání konečných modelů.

Čtvrtá kapitola začíná popisem našich vlastních vylepšení metody MACE,
která jsou v programu implementována.
Následuje velmi stručný popis samotné implementace.
Pátá kapitola obsahuje experimentální srovnání našeho hledače konečných modelů
s jinými programy.
Šestá kapitola shrnuje dosažené výsledky a popisuje další možná vylepšení
nejen metody MACE, která však nejsou implementována v našem programu.
