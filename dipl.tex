\documentclass{cs-dipl}

\hypersetup{pdftitle=Automatická konstrukce modelů}
\hypersetup{pdfauthor=Radek Miček}

\usepackage{bookmark}

\bibliography{dipl}

\DeclareMathOperator{\arity}{arita}
%\def\arita{\operatorname{\mathtt{arita}}}

\newcommand\comdots{, \ldots,}
\newcommand\timdots{\times \cdots \times}
\newcommand\crossbow{Crossbow}

\newglossaryentry{nat}
{
  name={\ensuremath{\mathbb{N}}},
  description={množina přirozených čísel, tj. $\{ 1, 2, 3, \ldots \}$},
  sort=nat
}
\newcommand\nat{\gls{nat}}

\newglossaryentry{natZ}
{
  name={\ensuremath{\mathbb{N}_0}},
  description={množina nezáporných celých čísel, tj. $\{ 0, 1, 2, \ldots \}$},
  sort=natZ
}
\newcommand\natZ{\gls{natZ}}

\newglossaryentry{sorts}
{
  name={\ensuremath{\mathcal{S}}},
  description={množina sort},
  sort=Sset
}
\newcommand\sorts{\gls{sorts}}

\newglossaryentry{funcs}
{
  name={\ensuremath{\mathcal{F}}},
  description={množina funkčních symbolů},
  sort=Fset
}
\newcommand\funcs{{\gls{funcs}}}

\newglossaryentry{preds}
{
  name={\ensuremath{\mathcal{P}}},
  description={množina predikátových symbolů},
  sort=Pset
}
\newcommand\preds{{\gls{preds}}}

\newglossaryentry{vars}
{
  name={\ensuremath{\mathcal{X}}},
  description={množina proměnných},
  sort=Xset
}
\newcommand\vars{\gls{vars}}

\newglossaryentry{interp}
{
  name={\ensuremath{\mathcal{I}}},
  description={interpretace},
  sort=I
}
\newcommand\interp{{\gls{interp}}}

\newglossaryentry{sort}
{
  name={\ensuremath{S}},
  description={sorta},
  sort=S
}
\newcommand\sort{{\gls{sort}}}

\newglossaryentry{var}
{
  name={\ensuremath{x, y, z}},
  description={proměnné},
  sort=x
}
\newcommand\var{{\glsdisp{var}{\ensuremath{x}}}}
\newcommand\varY{{\glsdisp{var}{\ensuremath{y}}}}
\newcommand\varZ{{\glsdisp{var}{\ensuremath{z}}}}

\newglossaryentry{func}
{
  name={\ensuremath{c, f, g}},
  description={funkční symboly ($c$ může značit i buňku)},
  sort=f
}
\newcommand\func{{\glsdisp{func}{\ensuremath{f}}}}
\newcommand\funcC{{\glsdisp{func}{\ensuremath{c}}}}
\newcommand\funcG{{\glsdisp{func}{\ensuremath{g}}}}

\newglossaryentry{pred}
{
  name={\ensuremath{P}},
  description={predikátový symbol},
  sort=P
}
\newcommand\pred{{\gls{pred}}}

\newglossaryentry{term}
{
  name={\ensuremath{s, t}},
  description={termy},
  sort=t
}
\newcommand\term{{\glsdisp{term}{\ensuremath{t}}}}
\newcommand\termS{{\glsdisp{term}{\ensuremath{s}}}}

\newglossaryentry{atom}
{
  name={\ensuremath{A}},
  description={atom},
  sort=A
}
\newcommand\atom{{\gls{atom}}}

\newglossaryentry{lit}
{
  name={\ensuremath{L}},
  description={literál},
  sort=L
}
\newcommand\lit{{\gls{lit}}}

\newglossaryentry{clause}
{
  name={\ensuremath{C}},
  description={klauzule},
  sort=C
}
\newcommand\clause{{\gls{clause}}}

\newglossaryentry{clauses}
{
  name={\ensuremath{N}},
  description={množina klauzulí},
  sort=N
}
\newcommand\clauses{{\gls{clauses}}}


\begin{document}

\renewcommand\thepage{Titulni strana}
\bookmark[page=1,level=0]{Titulní strana}
\pagestyle{empty}

% Aby stranky s obsahem nemely zobrazena cisla.
\addtocontents{toc}{\protect\thispagestyle{empty}}

\begin{center}

\large
\vspace*{5mm}

Univerzita Karlova v Praze

\vspace{1mm}

Matematicko-fyzikální fakulta

\vspace{5mm}

{\Large\bf DIPLOMOVÁ PRÁCE}

\vspace{10mm}

% Logo MFF.
\includegraphics[scale=0.4]{logo.eps}

\vspace{15mm}

{\Large Radek Miček}

\vspace{5mm}
{\LARGE\bf Automatická konstrukce modelů}

\vspace{5mm}
Katedra algebry

\vspace{15mm}

\begin{tabular}{rl}
Vedoucí diplomové práce: & doc. RNDr. David Stanovský, Ph.D.\\
\noalign{\vspace{8mm}}
Studijní program:        & Informatika \\
\noalign{\vspace{3mm}}
Studijní obor:           & Teoretická informatika\\
\end{tabular}

\vspace{26mm}

Praha, 2015
\end{center}

\normalsize

\newpage

\renewcommand\thepage{Podekovani}
\bookmark[page=2,level=0]{Poděkování}

\vspace*{10mm}

\noindent
Děkuji panu doc. RNDr. Davidu Stanovskému, Ph.D., vedoucímu mé diplomové
práce, za téma i nápady, jenž mi věnoval.

\vfill

\newpage

\renewcommand\thepage{Cestne prohlaseni}
\bookmark[page=3,level=0]{Čestné prohlášení}

% Cestne prohlaseni (okopirovano ze sablony Martina Marese).

\vglue 0pt plus 1fill

\noindent
Prohlašuji, že jsem tuto diplomovou práci vypracoval samostatně a výhradně
s~použitím citovaných pramenů, literatury a dalších odborných zdrojů.

\medskip\noindent
Beru na~vědomí, že se na moji práci vztahují práva a povinnosti vyplývající
ze zákona č. 121/2000 Sb., autorského zákona v~platném znění, zejména
skutečnost, že Univerzita Karlova v Praze má právo na~uzavření licenční
smlouvy o~užití této práce jako školního díla podle §60 odst. 1 autorského
zákona.

\vspace{10mm}

\hbox{\hbox to 0.65\hsize{%
V ..................... dne .....................
\hss}\hbox to 0.35\hsize{%
Podpis autora
\hss}}

\vspace{20mm}

\newpage

\renewcommand\thepage{Abstrakt}
\bookmark[page=4,level=0]{Abstrakt}

\noindent
Název práce: Automatická konstrukce modelů\\
Autor: Radek Miček\\
Katedra: Katedra algebry\\
Vedoucí diplomové práce: doc. RNDr. David Stanovský, Ph.D., Katedra algebry

\vskip2\bigskipamount

\noindent Abstrakt:
V této práci implementujeme metodu MACE pro hledání konečných
modelů v klasické logice prvního řádu bez sort.
Kromě známých modifikací metody MACE jsou popsány
a implementovány i úplně nové modifikace, například:
odzplošťování, generování redundantních klauzulí
a kódování do podmínek řešiče Gecode. Naše implementace
je porovnána s programy Mace4, Paradox a iProver.
V závěru práce jsou dány náměty na vylepšení a další výzkum.

\bigskip

\noindent Klíčová slova: konečné modely, propagace omezujících podmínek, SAT,
  redukce symetrií

\vskip8\bigskipamount

\noindent
Title: Automated model building\\
Author: Radek Miček\\
Department: Department of Algebra\\
Supervisor: doc. RNDr. David Stanovský, Ph.D., Department of Algebra

\vskip2\bigskipamount

\noindent Abstract:
\foreignlanguage{english}{
We implement a MACE-style method for finding finite models
in unsorted classical first-order logic. Additionally to well-known
modifications of the method we also describe and implement several
new modifications such as: unflattening, generation of redundant clauses
and encoding into Gecode constraints. Our implementation
is benchmarked together with Mace4, Paradox and iProver.
Finally, some suggestions for improvements and further
research are given.
}

\bigskip

\noindent Keywords: finite models, constraint propagation, SAT,
  symmetry reduction

\newpage

% Prvni stranka obsahu.
\renewcommand\thepage{Obsah}
\bookmark[page=5,level=0]{Obsah}

% Misto \tableofcontents - v obsahu nebude polozka obsah.
\chapter*{Obsah}
\makeatletter
\@starttoc{toc}
\makeatother

% Druha stranka obsahu.
\renewcommand\thepage{Obsah, pokracovani}

\newpage

\pagestyle{plain}
\pagenumbering{arabic}
\setcounter{page}{1}

\theoremstyle{sdefinition}
\newtheorem{defn}{Definice}[chapter]

\theoremstyle{sdefinition}
\newtheorem{note}{Poznámka}[chapter]

\theoremstyle{sdefinition}
\newtheorem{example}{Příklad}[chapter]

\theoremstyle{stheorem}
\newtheorem{thm}{Tvrzení}[chapter]

\printglossary[title={Značení},style=notationstyle]

\chapter{Úvod}

Cílem této práce je implementovat program pro hledání konečných
modelů v~klasické logice prvního řádu s rovností.
Bohatost logiky prvního řádu hraje důležitou roli při modelování
praktických problémů.
Díky ní se hledače modelů mohou uplatnit nejen při výzkumu
matematických struktur, ale i při verifikaci softwaru a hardwaru,
jako součást běhových prostředí deklarativních programovacích jazyků
anebo v jiných oblastech, kde se dnes používají SAT řešiče a SMT řešiče.

Existuje celá řada metod pro hledání modelů, tato práce se však
zabývá pouze metodou MACE a jejími modifikacemi.
Metoda MACE převádí problém hledání
konečného modelu velikosti $n$ v klasické logice prvního řádu
na problém SAT. Většina modifikací metody MACE se snaží o to, aby
byl výsledný SAT vyřešitelný rychle a s malým množstvím paměti.

Náš program implementuje metodu MACE, její modifikace známé
z programu Paradox a některé nové modifikace,
které, pokud je nám známo, zatím ještě nikdo nezkoušel.
Například kromě převodu na SAT je implementován
převod do jazyka řešiče omezujících podmínek Gecode
nebo přidávání redundantních formulí ke vstupu.

Práce je rozdělena do šesti kapitol. Tato kapitola
stručně představuje téma práce.
Druhá kapitola obsahuje definice základních pojmů,
jenž budeme používat v dalších kapitolách,
a přesnější popis řešeného problému.

Třetí kapitola popisuje existující metody pro hledání modelů.
Důraz je kladen na metody MACE a SEM pro hledání konečných modelů.

Čtvrtá kapitola začíná popisem našich vlastních vylepšení metody MACE,
která jsou v programu implementována.
Následuje velmi stručný popis samotné implementace.
Pátá kapitola obsahuje experimentální srovnání našeho hledače konečných modelů
s jinými programy.
Šestá kapitola shrnuje dosažené výsledky a popisuje další možná vylepšení
nejen metody MACE, která však nejsou implementována v našem programu.

\chapter{Základní definice}

Hlavním účelem této kapitoly je sjednotit základní definice a značení.
Kromě toho si v této kapitole popíšeme,
co přesně má náš program dělat -- co přesně je vstupem a výstupem programu.
Předpokládáme, že čtenář již zná základy klasické logiky prvního
řádu a základní techniky pro řešení omezujících podmínek.
Chybějící znalosti logiky lze doplnit z
\cite{enderton2001logic} a omezujících podmínek z
\cite{dechter2003constraints}.

Budeme pracovat ve vícesortové klasické logice prvního řádu.
V celém textu budeme
používat pevně danou signaturu, která se skládá z nekonečné množiny sort
$\sorts$, množiny funkčních symbolů $\funcs$, množiny predikátových symbolů
$\preds$ a funkce $\arity: \funcs \cup \preds \to \sorts^\star$,
kde $\sorts^\star$ značí množinu konečných posloupností sort.
Funkce $\arity$ přiřazuje každému funkčnímu symbolu neprázdnou posloupnost
sort $\langle \sort_1 \comdots \sort_n, \sort \rangle$,
kde $n \ge 0$ je počet argumentů,
$\sort_1 \comdots \sort_n$ jsou sorty argumentů a $\sort$ je sorta výsledku.
Funkce $\arity$ přiřazuje každému predikátovému symbolu posloupnost
sort $\langle \sort_1 \comdots \sort_n \rangle$, kde $n \ge 0$
je počet argumentů a $\sort_1 \comdots \sort_n$ jsou sorty argumentů.

Pro každou neprázdnou posloupnost sort $a$ obsahuje $\funcs$ nekonečně
mnoho symbolů s aritou $a$. Pro každou posloupnost sort $a$ obsahuje
$\preds$ nekonečně mnoho symbolů s aritou $a$.

Dále je každé sortě $\sort$ přiřazena nekonečná množina proměnných
$\vars_\sort$ a tyto množiny jsou navzájem disjunktní.
$\vars$ je množina všech proměnných,
tj. $\bigcup_{\sort \in \sorts} \vars_\sort$. Množiny $\nat0$, $\sorts$, $\funcs$,
$\preds$, $\vars$ jsou navzájem disjunktní.

Termy sorty $\sort$ jsou definovány induktivně:

\begin{itemize}
\item proměnná sorty $\sort$ je term sorty $\sort$,
\item je-li $\func$ funkční symbol s aritou
  $\langle \sort_1 \comdots \sort_n, \sort \rangle$
  a $\term_i$ term sorty $\sort_i$ pro každé $i \in \{ 1 \comdots n \}$, pak
  $\func(\term_1 \comdots \term_n)$ je term sorty $\sort$.
\end{itemize}

Atomy jsou definovány induktivně:

\begin{itemize}
\item jsou-li $\termS$ a $\term$ termy stejné sorty,
  pak $\termS \approx \term$ je atom,
\item je-li $\pred$ predikátový symbol
  s aritou $\langle \sort_1 \comdots \sort_n \rangle$
  a $\term_i$ term sorty $\sort_i$ pro každé $i \in \{ 1 \comdots n \}$, pak
  $\pred(\term_1 \comdots \term_n)$ je atom.
\end{itemize}

Literál je atom nebo jeho negace. Klauzule je disjunkce literálů.
Řekneme, že sorta $\sort$ se vyskytuje v termu, atomu, literálu, klauzuli,
množině klauzulí, pokud se tam vyskytuje proměnná sorty $\sort$ nebo
funkční či predikátový symbol, kde $\sort$ je sorta argumentu nebo výsledku.

% POZN: Nedefinujeme, ale používáme "sorta se vyskytuje v aritě",
% "term, atom, literál, klauzule, množina klauzulí, arita obsahuje sortu".
% Předpokládáme, že význam je čtenáři zřejmý.

Sorty budeme značit písmenem $\sort$, funkční symboly písmenem $\func$,
predikáty písmenem $\pred$ a proměnné písmeny $\var$, $\varY$ a $\varZ$.
Termy budeme značit písmeny $\termS$ a $\term$, atomy písmenem $\atom$,
literály písmenem $\lit$, klauzule písmenem $\clause$ a množiny
klauzulí písmenem $\clauses$.

Nyní zadefinujeme interpretaci.
Mějme množiny $\sorts' \subseteq \sorts$, $\funcs' \subseteq \funcs$
a $\preds' \subseteq \preds$, kde $\sorts'$ obsahuje alespoň sorty
vyskytující se v aritách symbolů z $\funcs'$ a $\preds'$.
Funkce $\interp$ definovaná na množině $\sorts' \cup \funcs' \cup \preds'$
je interpretace, jestliže

\begin{itemize}
\item každé sortě z $\sorts'$ přiřadí neprázdnou množinu (doménu),
\item každému funkčnímu symbolu $\func \in \funcs'$ s aritou
  $\langle \sort_1 \comdots \sort_n, \sort \rangle$ přiřadí funkci
  $\func^\interp: \interp(\sort_1) \timdots \interp(\sort_n) \to
  \interp(\sort)$
\item a každému predikátovému symbolu $\pred \in \preds'$ s aritou
  $\langle \sort_1 \comdots \sort_n \rangle$ přiřadí relaci
  $\pred^\interp \subseteq \interp(\sort_1) \timdots \interp(\sort_n)$.
\end{itemize}

Interpretace je konečná, jsou-li $\sorts'$, $\funcs'$ a $\preds'$
konečné a je-li každá doména konečná.
Konečná interpretace je číselná, pokud je každá doména
počáteční úsek $\nat0$, jinak řečeno, pokud je každá doména rovna
$\{ 0 \comdots n - 1 \}$, kde $n$ je velikost domény.
Pokud není řečeno jinak, bude
$\interp$ v dalším textu označovat interpretaci definovanou
na množině $\sorts' \cup \funcs' \cup \preds'$.

Buď $\interp$ interpretace a $\beta$ funkce definovaná na množině
$\bigcup_{\sort \in \sorts'} \vars_\sort$. Funkci $\beta$ nazveme ohodnocením
proměnných, pokud každé proměnné $\var \in \vars_\sort$ přiřazuje
prvek domény $\sort$.

Je-li $R$ term nebo atom nebo literál nebo klauzule, jenž
obsahuje pouze sorty z~$\sorts'$, funkční symboly z $\funcs'$ a
predikátové symboly z $\preds'$, pak
hodnotu $R$ při interpretaci $\interp$ a ohodnocení proměnných
$\beta$ značíme $\interp_\beta(R)$. Pro term $\term$ je $\interp_\beta(\term)$
definováno následovně:
\begin{align*}
\interp_\beta(\var) &= \beta(\var) \\
\interp_\beta\bigl(\func(\term_1 \comdots \term_n)\bigr) &=
  \interp\bigl(\func\bigr)\bigl(\interp_\beta(\term_1) \comdots
    \interp_\beta(\term_n)\bigr)
\end{align*}
Pro literál $\lit$ je $\interp_\beta(\lit)$ definováno takto:
\begin{align*}
\interp_\beta(\termS \approx \term) &=
  \begin{cases}
    1 & \text{pokud } \interp_\beta(\termS) = \interp_\beta(\term) \\
    0 & \text{jinak}
  \end{cases} \\
\interp_\beta\bigl(\pred(\term_1 \comdots \term_n)\bigr) &=
  \begin{cases}
    1 & \text{pokud }
      \bigl(\interp_\beta(\term_1) \comdots \interp_\beta(\term_n)\bigr)
      \in \interp(\pred) \\
    0 & \text{jinak}
  \end{cases} \\
\interp_\beta(\neg \atom) &= 1 - \interp_\beta(\atom)
\end{align*}
A nakonec definice $\interp_\beta(\clause)$ pro klauzuli $\clause$:
\begin{align*}
\interp_\beta(\lit_1 \vee \cdots \vee \lit_n) &=
  \max \bigl\{ 0, \interp_\beta(\lit_1) \comdots \interp_\beta(\lit_n) \bigr\}
\end{align*}

Skutečnost, že $\interp_\beta(R) = 1$, kde $R$ je atom, literál nebo klauzule,
zapisujeme
\[
\interp_\beta \models R.
\]
Platí-li $\interp_\beta \models R$ pro libovolné $\beta$, říkáme,
že $\interp$ je model $R$, a píšeme
\[
\interp \models R.
\]
Model $\interp$ je konečný resp. číselný, je-li interpretace
$\interp$ konečná resp. číselná.
$\interp$ je model množiny klauzulí $N$,
pokud je $\interp$ model každé klauzule z $N$.

Interpretace $\interp$ a $\interp'$ nad
$\sorts' \cup \funcs' \cup \preds'$ jsou izomorfní
(značíme $\interp \simeq \interp'$), pokud pro každou
sortu $\sort \in \sorts'$ existuje bijekce
$i_\sort: \interp(\sort) \to \interp'(\sort)$, že
\begin{itemize}
\item pro každý funkční symbol $\func \in \funcs'$ s aritou
  $\langle \sort_1 \comdots \sort_n, \sort \rangle$
  a pro každé $d_j \in \interp(\sort_j)$,
  kde $j \in \{ 1 \comdots n \}$, platí
  \[
     i_\sort\bigl(\interp(\func)(d_1 \comdots d_n)\bigr) =
     \interp'\bigl(\func\bigr)\bigl(i_{\sort_1}(d_1) \comdots
       i_{\sort_n}(d_n)\bigr)
  \]
\item a pro každý predikátový symbol $\pred \in \preds'$ s aritou
  $\langle \sort_1 \comdots \sort_n \rangle$
  a pro každé $d_j \in \interp(\sort_j)$,
  kde $j \in \{ 1, \comdots, n \}$, platí
  \[
     (d_1 \comdots d_n) \in \interp(\pred) \iff
     \bigl(i_{\sort_1}(d_1) \comdots i_{\sort_n}(d_n)\bigr) \in \interp'(\pred).
  \]
\end{itemize}

Připomeňme vlastnosti izomorfních interpretací:
\begin{itemize}
\item[(1)] Každá konečná interpretace je izomorfní číselné interpretaci.
\item[(2)] Je-li $\interp \simeq \interp'$ a $\clauses$ množina klauzulí,
  pak $\interp$ je model $\clauses$ právě tehdy,
  když $\interp'$ je model $\clauses$.
\end{itemize}

V tomto okamžiku máme potřebné definice,
abychom specifikovali chování programu -- jaký problém program řeší:

\begin{defn}
Program pracuje ve dvou režimech -- hledání jednoho modelu a
hledání všech neizomorfních modelů. V obou režimech je vstupem programu
sorta $\sort$ a konečná množina klauzulí $\clauses$,
kde se vyskytuje pouze sorta $\sort$.
V obou režimech jsou výstupem konečné modely $\clauses$
s jedinou sortou $\sort$, které
interpretují právě symboly z $\clauses$.

V režimu hledání jednoho modelu jsou navíc vstupem přirozená čísla
$n \leq n'$. Výstupem v tomto režimu je model
s doménou velikosti $m$, kde $n \leq m \leq n'$, pokud existuje.

V režimu hledání všech neizomorfních modelů je
navíc vstupem programu přirozené číslo $n$.
Výstupem v tomto režimu jsou všechny navzájem
neizomorfní modely s doménou velikosti $n$.
\end{defn}

Z (1) a (2) plyne, že pro každý vstup programu existuje výstup
obsahující pouze číselné modely. Aniž by to mělo vliv na úplnost
programu, může program hledat pouze číselné modely.
Navíc číselných interpretací, jenž jsou kandidáty na model,
je pouze konečně mnoho (důvodem je, že
číselných interpretací s jedinou sortou $\sort$ obsahujících právě symboly
z $\clauses$ a doménou velikosti $n$ je pouze konečně mnoho),
díky čemuž lze hledané
modely získat metodou generuj a testuj a stačí k tomu primitivní rekurze.

Zbývající definice se nám budou hodit při detekci izomorfních modelů.
Zobrazení kanonických reprezentantů $\rho$ pro číselné interpretace
je zobrazení z číselných interpretací do číselných interpretací splňující
\begin{itemize}
\item $\rho(\interp) \simeq \interp$
\item a $\interp \simeq \interp' \iff \rho(\interp) = \rho(\interp')$.
\end{itemize}

Orientovaný barevný graf (dále jen graf) je trojice $G = (V, E, c)$, kde
$V$ je konečná množina vrcholů, $E \subseteq V \times V$ je množina hran
a $c : V \to \nat0$ je obarvení vrcholů. Grafy
$G = (V, E, c)$ a $G' = (V, E', c')$ jsou izomorfní (značíme $G \simeq G'$),
pokud existuje bijekce $i : V \to V$, že
\begin{itemize}
\item $c'\bigl(i(v)\bigr) = c(v)$
\item a $(u, v) \in E \iff \bigl(i(u), i(v)\bigr) \in E'$.
\end{itemize}

Zobrazení kanonických reprezentantů $\rho$ pro grafy
je zobrazení z grafů do grafů splňující
\begin{itemize}
\item $\rho(G) \simeq G'$
\item a $G \simeq G' \iff \rho(G) = \rho(G')$.
\end{itemize}


\begin{note}
K definicím.
\begin{itemize}
\item Některé transformace problému zavádějí nové symboly --
  v takovém případě je možné rozšířit signaturu nebo zajistit,
  že stávající signatura obsahuje dostatek nepoužitých symbolů. Domníváme se,
  že druhá možnost, ač méně obvyklá, je přehlednější --
  v celém textu používáme jednu signaturu, která je
  zkonstruována tak, že pro každou aritu vždy existuje
  nepoužitý symbol. Díky tomu, že signatura je pouze jedna,
  není třeba ji zmiňovat v každém tvrzení.
\item Proč je množina sort $\sorts'$ v interpretaci zadána explicitně, proč
  není implicitně určena na základě interpretovaných symbolů?
  Interpretovat pouze sorty obsažené ve funkčních a predikátových
  symbolech nestačí, je třeba interpretovat i sorty proměnných.
\item Proč je sorta na vstupu zadána explicitně, proč není implicitně určena
  z klauzulí na vstupu? Vstupní množina klauzulí nemusí žádnou sortu
  obsahovat (prázdná množina nebo množina s prázdnou klauzulí), v takovém
  případě není
  jednoznačně určeno, jakou sortu má model interpretovat, a číselných
  interpretací, jenž jsou kandidáty na model, je nekonečně mnoho.
\item Vstup i výstup programu jsou jednosortové, proč výklad komplikovat
  více\-sor\-tovou logikou? Některé problémy s jednou sortou
  lze přeformulovat jako rychleji vyřešitelné problémy s více sortami.
\end{itemize}
\end{note}

\chapter{Známé metody hledání modelů}

Jak, jsme viděli v minulé kapitole, najít model
velikosti $n$ nebo ukázat, že neexistuje, je snadné --
stačí použít metodu generuj a testuj.
Tuto metodu lze jednoduše implementovat,
kvůli své pomalosti je však prakticky nepoužitelná.
V této kapitole budeme studovat pokročilejší metody
pro hledání modelů, které jsou obvykle rychlejší
než metoda generuj a testuj.
Začneme popisem základních variant metod MACE a SEM,
poté popíšeme modifikace obou metod.
Metody MACE a SEM interně pracují s klauzulemi bez proměnných
a interpretace reprezentují explicitně,
na konci kapitoly pak zmíníme metody, které interně
pracují s klauzulemi s proměnnými a interpretace reprezentují
implicitně.

\section{Metody MACE a SEM}

Obě metody řeší následující úlohu:
Vstupem úlohy je konečná množina
klauzulí $\clauses$ a velikost domény $n_\sort$
pro každou sortu $\sort \in \sorts'$, kde $\sorts'$ je množina
obsahující právě sorty z $\clauses$. Výstupem úlohy jsou číselné modely
nad $\sorts' \cup \funcs' \cup \preds'$, kde domény mají velikosti
$n_\sort$ a $\funcs'$ resp. $\preds'$ je množina obsahující právě funkční
resp. predikátové symboly z $\clauses$.

Všimněme si, že zadání úlohy jednoznačně určuje domény sort
(domény jsou počátečním úsekem $\natZ$ a známe i jejich velikosti).
Díky tomu jsou hledané modely jednoznačně určeny pouze funkcemi a relacemi.
Navíc víme, nad jakými množinami jsou funkce i relace definovány.
Označme $D_\sort = \{ 0 \comdots n_\sort - 1 \}$ doménu
sorty $\sort \in \sorts'$.
Každému funkčnímu symbolu $\func \in \funcs'$ arity
$\langle \sort_1 \comdots \sort_n, \sort \rangle$ pak model přiřazuje funkci
$f^\interp : D_{\sort_1} \timdots D_{\sort_n} \to D_\sort$
a každému predikátovému symbolu $\pred \in \preds'$ arity
$\langle \sort_1 \comdots \sort_n \rangle$
je přiřazena relace $P^\interp \subseteq D_{\sort_1} \timdots D_{\sort_n}$.

\newcommand\cells{\ensuremath{\mathcal{C}}}

Tyto funkce a relace můžeme reprezentovat tabulkou.
Funkce $\func^\interp$ přiřazená symbolu $\func$
s aritou $\langle \sort_1 \comdots \sort_n, \sort \rangle$
je reprezentována tabulkou o rozměrech $n_{\sort_1} \timdots n_{\sort_n}$,
kde buňky tabulky obsahují prvky z $D_\sort$.
\[
\cells_\func = \bigl\{ \func(i_1 \comdots i_n) \bigm| i_j \in D_{\sort_j},
  \text{ kde } j \in \{ 1 \comdots n  \}  \bigr\}
\]
je množina buněk tabulky pro funkci $\func^\interp$.
Buňka $\func(i_1 \comdots i_n)$ obsahuje hodnotu $v$ (zapisujeme
$\func(i_1 \comdots i_n) = v$) právě tehdy,
když $\func^\interp(i_1 \comdots i_n) = v$.

Relace $\pred^\interp$
přiřazená predikátovému symbolu $\pred$ s aritou
$\langle \sort_1 \comdots \sort_n \rangle$
je reprezentována tabulkou o rozměrech $n_{\sort_1} \timdots n_{\sort_n}$,
kde buňky tabulky obsahují prvky 0 nebo 1.
\[
\cells_\pred = \bigl\{ \pred(i_1 \comdots i_n) \bigm| i_j \in D_{\sort_j},
  \text{ kde } j \in \{ 1 \comdots n  \}  \bigr\}
\]
je množina buněk tabulky pro relaci $\pred^\interp$.
Fakt, že buňka $\pred(i_1 \comdots i_n)$ obsahuje hodnotu $v$
zapisujeme $\pred(i_1 \comdots i_n) = v$.
Buňka $\pred(i_1 \comdots i_n)$ obsahuje hodnotu 1
právě tehdy, když $(i_1 \comdots i_n) \in \pred^\interp$.

% Poznámka: Aplikace funkčních resp. predikátových symbolů bez argumentů
% budeme zapisovat bez prázdných závorek tj. c resp. P
% a nikoliv c() resp. P().

V dalším textu budeme hledání modelů metodami MACE a SEM chápat jako
vyplňování tabulek funkcí a relací.

\subsection{MACE -- základní varianta}

Metoda MACE je popsána v \cite{mccune94davis-putnam}.
Jak jsme již řekli v úvodu, metoda MACE převádí problém na SAT.
Napřed vytvoříme instanci SATu, jejíž řešení odpovídají 1 ku 1
tabulkám funkcí\footnote{Pro funkce v této práci používáme přímé kódování.
Existují však i jiné možnosti.
Pokud bychom například chtěli mít méně
výrokových proměnných pro funkce, mohli bychom použít logaritmické kódování.
Nicméně množství proměnných pro funkce nebývá problém,
neboť se obvykle pracuje pouze s malými doménami.
Naopak problém je horší propagace logaritmického kódování.
Různá kódování včetně přímého a logaritmického jsou popsána
v \cite{gavanelli2007}.} a relací:

% Z předchocízho textu je zřejmé, že
% tabulky funkcí a relací odpovídají číselným
% interpretacím nad $\sorts' \cup \funcs' \cup \preds'$, kde doména
% $\sort \in \sorts'$ má velikost $n_\sort$:

\begin{itemize}
\item[1)] Pro každou funkci $\func^\interp$, jejíž tabulku chceme vyplnit:
  \begin{itemize}
  \item[a)] Pro každou buňku $\func(i_1 \comdots i_n)$ a hodnotu $v$,
    kterou daná buňka může obsahovat, přidáme výrokovou proměnnou
    $A_{\func(i_1 \comdots i_n) = v}$.
  \item[b)] Pro každou buňku $\func(i_1 \comdots i_n)$
    a dvojici hodnot $v < v'$,
    které buňka může obsahovat, přidáme klauzuli
    $\neg A_{\func(i_1 \comdots i_n) = v} \vee \neg A_{\func(i_1 \comdots i_n) = v'}$.
  \item[c)] Pro každou buňku $\func(i_1 \comdots i_n)$ a všechny hodnoty
    $0 \comdots v$, jenž může obsahovat, přidáme klauzuli
    $A_{\func(i_1 \comdots i_n) = 0} \vee \cdots \vee A_{\func(i_1 \comdots i_n) = v}$.
  \end{itemize}
\item[2)] Pro každou relaci $\pred^\interp$, jejíž tabulku chceme vyplnit: Pro
  každou buňku $\pred(i_1 \comdots i_n)$ přidáme výrokovou proměnnou
  $A_{\pred(i_1 \comdots i_n) = 1}$.
\end{itemize}

Na základě řešení instance SATu můžeme vyplnit tabulky funkcí a relací:
Výrok $V$ o buňce tabulky bude pravdivý právě tehdy,
když výroková proměnná $A_V$ má v řešení hodnotu 1.
Klauzule z kroku 1b) zajišťují, že každá buňka obsahuje
nejvýše jednu hodnotu. Klauzule z kroku 1c) zajišťují, že každá
buňka obsahuje alespoň jednu hodnotu.

Tabulky, které takto získáme, nejsou nutně modely $\clauses$.
Aby tomu tak bylo, rozšíříme instanci SATu o další výrokové klauzule,
které získáme zakódováním klauzulí z $\clauses$ do výrokové logiky.
Uvažme například klauzuli
$\clause_1 = \{ \func(\var, \varY) \approx \var \vee \funcC \approx \var \}$,
kde $\func$ je funkční symbol arity
$\langle \sort_1, \sort_2, \sort_1 \rangle$, $\funcC$
je funkční symbol arity $\langle \sort_1 \rangle$ a sorta $\sort_1$
resp. $\sort_2$
má doménu velikosti 3 resp. 2.
Aby tabulky symbolů $\func$ a $\funcC$ byly modelem klauzule
$\clause_1$, musí platit následující podmínky:
\begin{gather*}
  \func(0, 0) = 0 \text{ nebo } \funcC = 0, \\
  \func(1, 0) = 1 \text{ nebo } \funcC = 1, \\
  \func(2, 0) = 2 \text{ nebo } \funcC = 2, \\
  \func(0, 1) = 0 \text{ nebo } \funcC = 0, \\
  \func(1, 1) = 1 \text{ nebo } \funcC = 1, \\
  \func(2, 1) = 2 \text{ nebo } \funcC = 2.
\end{gather*}
Uvedené podmínky se skládají z výroků, jenž přímo odpovídají
vý\-ro\-ko\-vým pro\-měnným.
Díky tomu můžeme podmínky snadno zakódovat do výrokových klauzulí:
\begin{gather*}
  A_{\func(0, 0) = 0} \vee A_{\funcC = 0}, \\
  A_{\func(1, 0) = 1} \vee A_{\funcC = 1}, \\
  A_{\func(2, 0) = 2} \vee A_{\funcC = 2}, \\
  A_{\func(0, 1) = 0} \vee A_{\funcC = 0}, \\
  A_{\func(1, 1) = 1} \vee A_{\funcC = 1}, \\
  A_{\func(2, 1) = 2} \vee A_{\funcC = 2}.
\end{gather*}

Klauzuli $\clause_1$ jsme tedy zakódovali do 6 výrokových klauzulí,
které přidáme do naší instance SATu.
Bohužel, ne každou klauzuli jde zakódovat takto přímočaře.
Například, aby tabulky symbolů byly modelem klauzule
$\clause_2 = \func(\funcC, \varY) \approx \funcC$, musí platit podmínky
\begin{gather*}
  \func(\funcC, 0) = \funcC, \\
  \func(\funcC, 1) = \funcC.
\end{gather*}
Na rozdíl od podmínek pro $\clause_1$ výroky v podmínkách pro $\clause_2$
neodpovídají přímo výrokovým proměnným.
Odpovídaly by, kdybychom znali přesnou hodnotu $\funcC$, jenže tu neznáme.
Víme však, že $\funcC$ má jednu z hodnot 0, 1, nebo 2, nabízí se tedy
použít podmiňování. Pokud má $\funcC$ hodnotu 0, pak musí platit
podmínky
\begin{gather*}
  \func(0, 0) = 0, \\
  \func(0, 1) = 0,
\end{gather*}
pokud má $\funcC$ hodnotu 1, musí platit
\begin{gather*}
  \func(1, 0) = 1, \\
  \func(1, 1) = 1,
\end{gather*}
a nakonec, pokud má $\funcC$ hodnotu 2, musí platit
\begin{gather*}
  \func(2, 0) = 2, \\
  \func(2, 1) = 2.
\end{gather*}
Tyto podmínky již do výrokových klauzulí zakódujeme snadno:
\begin{gather*}
  A_{\funcC = 0} \implies A_{\func(0, 0) = 0}, \\
  A_{\funcC = 0} \implies A_{\func(0, 1) = 0}, \\
  A_{\funcC = 1} \implies A_{\func(1, 0) = 1}, \\
  A_{\funcC = 1} \implies A_{\func(1, 1) = 1}, \\
  A_{\funcC = 2} \implies A_{\func(2, 0) = 2}, \\
  A_{\funcC = 2} \implies A_{\func(2, 1) = 2}.
\end{gather*}

Klauzule, které lze zakódovat do výrokových klauzulí
bez podmiňování, nazveme ploché:

\begin{defn}
Klauzule je plochá, pokud má každý atom jeden z následujících tvarů
\begin{itemize}
\item $\func(\var_1 \comdots \var_n) \approx \varY$
  (případně $\varY \approx \func(\var_1 \comdots \var_n)$,
\item $\pred(\var_1 \comdots \var_n)$,
\item nebo $\var \approx \varY$,
\end{itemize}
kde $\func, \pred, \var, \var_1 \comdots \var_n, \varY$ jsou
vhodně zvolené symboly a proměnné.
\end{defn}

Pomocí těchto dvou pravidel můžeme zploštit každou
klauzuli $\clause$:
\begin{itemize}
\item Je-li $\term$ term, jenž se vyskytuje v $\clause$, a $\var$ proměnná,
  jenž se v $\clause$ nevyskytuje, pak klauzuli $\clause$ můžeme
  nahradit novou klauzulí $\term \not\approx \var \vee \clause$.
\item Je-li $\lit$ literál z $\clause$ tvaru $\term \not\approx \var$
  (resp. $\var \not\approx \term$),
  že $\term$ neobsahuje $\var$,
  pak můžeme $\clause$ upravit tak, že jeden výskyt $\term$
  mimo $\lit$ nahradíme $\var$.
\end{itemize}
Navíc má nová klauzule stejné modely jako původní klauzule.

Použití pravidel pro zplošťování si ukážeme na klauzuli $\clause_2$.
Klauzule $\clause_2$ není plochá, neboť obsahuje atom
$\func(\funcC, \varY) \approx \funcC$, který nemá požadovaný tvar
(atom může obsahovat nejvýše jeden výskyt funkčního nebo predikátového
symbolu). Potřebujeme tedy nahradit $\funcC$ proměnnou.
Napřed použijeme první pravidlo a ke klauzuli $\clause_2$ přidáme literál
$\funcC \not\approx \var$, tj. dostaneme novou klauzuli
$\clause_2' = \funcC \not\approx \var \vee
  \func(\funcC, \varY) \approx \funcC$.
Na klauzuli $\clause_2'$ pak můžeme použít dvakrát druhé pravidlo
a nahradit dva výskyty $\funcC$ proměnnou $\var$, čímž
dostaneme plochou klauzuli
$\clause_2'' = \funcC \not\approx \var \vee
\func(\var, \varY) \approx \var$.

Kdybychom zakódovali
$\clause_2''$ do výrokových klauzulí, dostali bychom stejné klauzule,
jako při kódování $\clause_2$ s pomocí podmiňování.
V dalším textu již budeme používat výhradně zplošťování,
důvodem je, že zplošťování může plně nahradit
podmiňování\footnote{Podmiňování je vlastně zplošťování
přímo integrované do procesu, který transformuje klauzule z $\clauses$
na výrokové klauzule. My jsme tento složitější proces rozdělili na dva
jednodušší procesy -- prvním je zplošťování a druhým je transformace
plochých klauzulí z $\clauses$ na výrokové klauzule. Nyní
můžeme oba procesy vylepšovat zvlášť, případně vložit další procesy mezi
ně, což by jinak nebylo možné.}.
Později zplošťování rozšíříme o další pravidla,
díky nimž v některých případech dostaneme méně výrokových klauzulí,
než kdybychom použili podmiňování.

Kódujeme-li plochou klauzuli $\clause$ do výrokových klauzulí,
počet výrokových klauzulí závisí exponenciálně na počtu proměnných
v $\clause$.
Nepříjemnou vlastností zplošťování je, že jeho první pravidlo
zvyšuje počet proměnných v klauzuli. Tato nepříjemnost je hlavní motivací
pro metodu, jenž si nyní představíme.

\subsection{SEM -- základní varianta}

Metoda SEM je popsána v \cite{zhang1995sem}.
Na rozdíl od metody MACE metoda SEM problém nikam nepřevádí, ale rovnou
ho sama řeší.
Pro popis metody SEM budeme předpokládat $v, v' \in \natZ$ a $c$ je buňka.

Metoda SEM sestává ze dvou funkcí \textproc{Search} a \textproc{Propagate}.
Tyto funkce reprezentují stav řešeného problému jako
pětici $(A_\funcs, A_\preds, U_\funcs, U_\preds, E)$:
% U \funcs a \preds nejsou apostrofy -- dolním indexem
% chceme naznačit, že se množina A_\funcs týká funkcí a A_\preds
% týká predikátů.

% A - assigned.
% U - unassigned.
% E - ensure.

% Místo C (jako constraints nebo clauses) používáme E,
% C již označuje klauzule.
\begin{itemize}
\item Množiny $A_\funcs$ a $A_\preds$ obsahují buňky,
  jimž byla přiřazena hodnota. $A_\funcs$ je množina dvojic
  $(c, v)$, kde $c$ je buňka funkce a $v$
  je hodnota přiřazená této buňce. $A_\preds$ je množina
  jejíž prvky mají tvar $c$ nebo $\neg c$, kde $c$ je buňka
  relace. Je-li $c \in A_\preds$, pak je buňce $c$ přiřazena hodnota 1,
  je-li $\neg c \in A_\preds$, pak je buňce $c$ přiřazena hodnota 0.
\item Množiny $U_\funcs$ a $U_\preds$ obsahují buňky,
  jimž nebyla přiřazena hodnota. $U_\funcs$ je množina
  dvojic $(c, D)$, kde $c$ je buňka funkce a $D$
  je množina hodnot, jenž lze přiřadit buňce.
  $U_\preds$ je množina buněk relací.
% Poznámka: Komponenty nemusí být nutně literály podle definice
% z předchozí kapitoly, nicméně nebezpečí nedorozumnění nehrozí.
\item $E$ je množina podmínek, které zbývá splnit.
  Každá podmínka je disjunkce komponent, těmto komponentám
  budeme říkat literály.
\end{itemize}
Invariantem je, že každá buňka má právě jeden výskyt
v právě jedné z~množin $A_\funcs, A_\preds, U_\funcs, U_\preds$.
Konflikt nastává v okamžiku, kdy se v $E$ objeví podmínka bez literálů
nebo kdy se v $U_\funcs$ objeví dvojice tvaru $(c, \emptyset)$.
Model je nalezen v okamžiku, kdy jsou všechny podmínky
splněny (množina $E$ je prázdná) a každé buňce je přiřazena
hodnota (množiny $U_\funcs$ a $U_\preds$ jsou prázdné).
Model je určen množinami $A_\funcs$ a $A_\preds$.

Funkce \textproc{Search} je definována následovně:
\medskip
\begin{algorithmic}
\Function{Search}{$A_\funcs, A_\preds, U_\funcs, U_\preds, E$}
  \State $r \gets \Call{Propagate}{A_\funcs, A_\preds, U_\funcs, U_\preds, E}$
  \If{$r \neq \text{\texttt{conflict}}$}
    \State $(A_\funcs, A_\preds, U_\funcs, U_\preds, E) \gets r$
    \If{$U_\funcs \cup U_\preds = \emptyset$}
      \State \textbf{print model} $(A_\funcs, A_\preds)$
    \Else
      \State $c \gets$ vyber buňku z $U_\funcs \cup U_\preds$
      \If{$(c, D) \in U_\funcs$ pro nějaké $D$}
        \For{$v \in D$}
          \State \Call{Search}{$A_\funcs \cup \{ (c, v) \}, A_\preds,
            U_\funcs \setminus \{ (c, D) \}, U_\preds, E$}
        \EndFor
      \Else
        \State \Call{Search}{$A_\funcs, A_\preds \cup \{ c \},
          U_\funcs, U_\preds \setminus \{ c \}, E$}
        \State \Call{Search}{$A_\funcs, A_\preds \cup \{ \neg c \},
          U_\funcs, U_\preds \setminus \{ c \}, E$}
      \EndIf
    \EndIf
  \EndIf
\EndFunction
\end{algorithmic}
\medskip

Vstupem funkce \textproc{Search} je již zmíněná pětice
$(A_\funcs, A_\preds, U_\funcs, U_\preds, E)$. Výstupem funkce
\textproc{Search} jsou modely, pro výpis modelů se používá příkaz
\textbf{print model}.

Na začátku metoda SEM zavolá funkci \textproc{Search} s argumenty
\begin{align*}
A_\funcs &= \emptyset, \\
A_\preds &= \emptyset, \\
U_\funcs &=
  \bigl\{ (c, D_\sort) \bigm|
    \func \in \funcs',
    \sort \text{ je sorta výsledku } \func,
    c \in \cells_\func \bigr\}, \\
U_\preds &=
  \bigl\{ c \bigm|
    \pred \in \preds',
    c \in \cells_\pred \bigr\}, \\
E &=
  \Bigl\{ e \Bigm|
    \array{c}
        e \text{ vznikne z $\clause \in \clauses$ použitím substituce,
                která každou} \\[-0.3ex]
      \text{proměnnou $\var$ z $\clause$ nahradí hodnotou z $D_\sort$, kde $\sort$
            je sorta $\var$}
    \endarray
    \Bigr\}.
\end{align*}
Výstupem takového volání jsou hledané modely.
Funkce \textproc{Search} používá pomocnou funkci \textproc{Propagate}.
Jejím vstupem je opět pětice
$(A_\funcs, A_\preds, U_\funcs, U_\preds, E)$.
Funkce \textproc{Propagate} pak na danou pětici opakovaně
aplikuje ná\-sle\-du\-jící pravidla, dokud nenastane konflikt nebo
dokud se pětice mění:
\begin{itemize}
% Pravidlo 1 z clanku o SEM:
\item Pro každé $(c, v) \in A_\funcs$ nahradit všechny výskyty $c$ v $E$
  hodnotou $v$.
% Odstranovani splnenych podminek a nesplnenych literalu
% (v clanku o SEM jsou tyto pravidla implicitni):
\item Pro každé $v, v'$ takové, že $v \neq v'$:
  \begin{itemize}
  \item Z $E$ odebrat podmínky, jenž obsahují literál $v \approx v$
    nebo literál $v \not\approx v'$ nebo literál z $A_\preds$.
  \item Z každé podmínky v $E$ odebrat literály tvaru $v \not\approx v$,
    literály tvaru $v \approx v'$ a literály tvaru $\neg \lit$,
    kde $\lit \in A_\preds$.
  \end{itemize}
% Pravidlo 1b z clanku o SEM:
\item Pro každou podmínku s jedním literálem tvaru $c \approx v$
  (resp. $v \approx c$),
  kde $(c, D) \in U_\funcs$ pro nějaké $D$,
  odebrat $(c, D)$ z $U_\funcs$
  a přidat $(c, v)$ do $A_\funcs$.
\item Pro každou podmínku s jedním literálem tvaru $c \not\approx v$
  (resp. $v \not\approx c$),
  kde $(c, D) \in U_\funcs$ pro nějaké $D$,
  odebrat $v$ z $D$.
% Pravidlo 1a z clanku o SEM:
\item Pro každou podmínku s jedním literálem tvaru $c$,
  kde $c \in U_\preds$,
  odebrat $c$ z $U_\preds$
  a přidat $c$ do $A_\preds$.
\item Pro každou podmínku s jedním literálem tvaru $\neg c$,
  kde $c \in U_\preds$,
  odebrat $c$ z $U_\preds$
  a přidat $\neg c$ do $A_\preds$.
% Pravidlo 3 z clanku o SEM:
\item Pro každou dvojici $(c, \{ v \}) \in U_\funcs$
  odebrat $(c, \{ v \})$ z $U_\funcs$
  a přidat $(c, v)$ do $A_\funcs$.
\end{itemize}

V případě, že nastal konflikt, funkce
\textproc{Propagate} vrátí hodnotou \texttt{conflict},
v opačném případě je vrácena nová pětice.

\subsection{Lepší propagace pro SEM}

Popsali jsme základní varianty metod MACE a SEM,
nyní se podíváme na jejich modifikace.
První modifikací, kterou si ukážeme, bude lepší
propagace pro metodu SEM.

Obsahuje-li $E$ podmínku s jedním literálem tvaru $c \approx v$
(resp. $v \approx c$) a je-li $(c, D) \in U_\funcs$ pro nějaké $D$,
pak funkce \textproc{Propagate} přiřadí buňce $c$ hodnotu $v$
bez ohledu na to, zda $v \in D$. Pokud $v \notin D$, dojde
v každém případě ke konfliktu -- lepší by tedy bylo přiřazení vůbec
neprovádět a okamžitě hlásit konflikt.

První dvě pravidla funkce \textproc{Propagate} slouží
ke zjednodušování množiny podmínek $E$.
Tato pravidla využívají množiny $A_\funcs$ a $A_\preds$
a nevyužívají množinu $E$.
Například, máme-li
\[
  E = \bigl\{ \func(\funcG(0)) \approx 1,
              \func(\funcG(0)) \approx \funcG(0) \bigr\}
\]
a obsahuje-li $U_\funcs$ dvojice $(\func(1), D)$
a $(\funcG(0), D)$, kde $D = \{ 0, 1, 2, 3 \}$, pak funkce
\textproc{Propagate} nic neudělá.
Kdybychom však propagaci rozšířili o pravidlo demodulace,
mohli bychom první podmínku $\func(\funcG(0)) \approx 1$
použít ke zjednodušení druhé podmínky $\func(\funcG(0)) \approx \funcG(0)$,
čímž bychom dostali:
\[
  E = \bigl\{ \func(\funcG(0)) \approx 1, 1 \approx \funcG(0) \bigr\}.
\]
S těmito podmínkami si již poradí i původní funkce \textproc{Propagate}
-- buňkám $\func(1)$ a $\funcG(0)$ přiřadí hodnotu 1 a podmínky eliminuje.

Demodulace používá rovnosti, mohli bychom využít i nerovnosti?
Je-li
\[
  E = \bigl\{ \func(2, 3) \approx 4,
              \func(2, \funcG(5)) \not\approx 4 \bigr\},
\]
pak buňce $\funcG(5)$ nelze přiřadit hodnotu 3. Stejný
závěr vyvodíme, když prohodíme rovnost s nerovností
\[
  E' = \bigl\{ \func(2, 3) \not\approx 4,
              \func(2, \funcG(5)) \approx 4 \bigr\}
\]
nebo když informace $\func(2, 3) = 4$ resp. $\func(2, 3) \neq 4$  bude
pocházet z $A_\funcs$ resp. $U_\funcs$ místo z $E$ resp. $E'$.

Obecně, obsahuje-li $E$ podmínky tvaru
\begin{align*}
\func(v_1 \comdots v_n) &\approx v, \\
\func(v_1 \comdots v_{i-1}, c, v_{i+1} \comdots  v_n) &\not\approx v,
\end{align*}
kde $v_1 \comdots v_n \in \natZ$, pak odvodíme, že buňka $c$ nemůže
obsahovat hodnotu $v_i$. Stejný závěr lze učinit, když
v první podmínce bude nerovnost a ve druhé rovnost
nebo když informace $\func(v_1 \comdots v_n) = v$ resp. její
negace nebude pocházet z $E$, ale z $A_\funcs$ resp. $U_\funcs$.
Rovněž lze prohodit levou a pravou stranu rovnosti anebo
nerovnosti. Analogické pravidlo lze formulovat i pro relace.
Tento druh propagace je implementován v hledači modelů Mace4
\cite{mccune03mace4}, kde se nazývá negativní propagace.

V zobecňování bychom mohli pokračovat. Například
z následujících podmínek
\[
  E = \bigl\{ \neg \pred(0, \funcG(2), \funcG(2)),
              \pred(0, \funcG(2), 1) \bigr\}
\]
bychom chtěli odvodit, že buňce $\funcG(2)$ nelze přiřadit hodnotu 1.
K tomu nám negativní propagace, jak jsme ji formulovali, nestačí.
Místo ní bychom mohli použít pravidlo:
Pokud $E$ obsahuje
podmínky tvaru $\pred(s_1 \comdots s_n)$
a $\neg \pred(t_1 \comdots t_n)$, pak buňkám nesmíme přiřadit
hodnoty, aby $s_i = t_i$ pro všechna $i \in \{ 1 \comdots n \}$.
Je zřejmé, že jednu podmínku z $E$ lze nahradit
literálem z $A_\preds$. Například je-li
\[
  E = \bigl\{ \pred(\funcG(1), \funcG(1)) \bigr\}
\]
a $\neg \pred(0, 0) \in A_\preds$, pak buňce $\funcG(1)$
nelze přiřadit hodnotu 0.

% Analogicky můžeme postupovat i u rovností:
% Obsahuje-li $E$ podmínky tvaru
% $s \approx s'$ a $t \not\approx t'$,
% pak buňkám nesmíme přiřadit hodnoty, aby $(s = t) \wedge (s' = t')$
% nebo $(s' = t) \wedge (s = t')$.
%
% Toto analogické pravidlo opět musí uvažovat množiny
% A_\funcs a U_\funcs, aby bylo ostře silnější než negativní propagace
% pro rovnosti.

% Další metoda propagace: Vybrat n-tici podmínek
% a použít zobecněnou hranovou konzistenci: Každé buňce
% zkusit přiřadit každou její hodnotu a pak zkusit najít podporu
% v ostatních buňkách - tj. zda ostatním buňkám jde přiřadit
% hodnota, aby všechny podmínky z vybrané n-tice byly splněny.

\subsection{Opakované použití výrokových klauzulí pro MACE}

Předpokládejme, že chceme opakovaně hledat modely metodou MACE
s tím, že při každém hledání zvětšíme domény některých sort.

Naivním řešením je vytvořit úplně novou instanci SATu pro každé hledání.
To je ovšem zbytečné, neboť nová instance obsahuje skoro
všechny výrokové klauzule z původní instance. Výjimkou
jsou klauzule vytvořené v bodu 1c) pro funkce $\func^\interp$,
kde sorta výsledku $\func$ patří mezi sorty, jejichž doménu zvětšujeme,
a klauzule z~nich odvozené například pomocí učení klauzulí
\cite{silva1997grasp}.

S lepším řešením přišel program Paradox \cite{claessen03paradox}.
Před každým hledáním modelu se vytvoří nová výroková proměnná $A$.
Literál $A$ se přidá do klauzulí vytvářených v~bodu 1c), které
bude třeba odstranit. Jelikož je učení klauzulí založeno na rezoluci,
literál $A$ se rozšíří i do naučených klauzulí, které byly
odvozeny z klauzulí, jenž bude třeba odstranit.
SAT řešič je spuštěn s předpokladem $\neg A$,
tudíž literál $A$ neovlivní nalezený model.
Pro následující hledání s většími doménami jsou klauzule obsahující
literál $A$ odstraněny přidáním jednotkové klauzule $A$.

Opakované hledání modelů metodou MACE se používá pro implementaci
režimu hledání jednoho modelu,
kdy jsou dána čísla $n \leq n'$ z $\nat$ a hledáme
model s doménou velikosti $m$, že $n \leq m \leq n'$.

\subsection{Další pravidla pro zplošťování}

Počet různých proměnných v klauzuli $\clause \in \clauses$
ovlivňuje, do kolika výrokových klauzulí bude $\clause$
zakódována metodou MACE a do kolika podmínek metodou SEM.
Obecně počet výrokových klauzulí a počet podmínek
závisí exponenciálně na počtu různých proměnných.
Například, obsahuje-li klauzule $\clause$ 3 proměnné z~$\vars_{\sort_1}$
a 5 proměnných z $\vars_{\sort_2}$ a je-li $n_{\sort_1} = 4$
a $n_{\sort_2} = 8$, pak počet výrokových klauzulí resp. počet podmínek
bude $4^3 \cdot 8^5 = 2^{21}$.

Jak je vidět, i pro klauzuli s pouhými 8 proměnnými může vzniknout
velké množství výrokových klauzulí resp. podmínek.
Připomeňme navíc, že zplošťování, které je třeba
provést u metody MACE, přidává do klauzulí nové proměnné.
Následující tři modifikace se tedy snaží transformovat
$\clauses$ tak, aby klauzule v $\clauses$ obsahovaly méně proměnných.

Začneme tím, že vylepšíme zplošťování. Oproti pravidlům
pro zplošťování, která jsme uvedli, obsahuje program Paradox \cite{paradox}
navíc následující pravidla:
\begin{itemize}
\item Klauzuli $\clause$, jenž obsahuje literál $\lit$ tvaru
  $\var \not\approx \varY$, můžeme upravit tak,
  že z ní odstraníme literál $\lit$ a všechny výskyty
  $\var$ nahradíme $\varY$.
\item Klauzuli $\clause = \lit_1 \vee \cdots \vee \lit_n$, kde literál
  $\lit_i$ je tvaru $\var \approx \varY$ a literál $L_j$ je tvaru
  $\term \not\approx \varY$ můžeme nahradit klauzulí
  $\clause' = \term \approx \var \vee
  \bigvee_{k \in \{ 1 \comdots n \} \setminus \{ i, j \}} \lit_k$, pokud $\clause'$
  neobsahuje $\varY$. Poznamenejme, že pravidlo rovněž platí,
  když prohodíme levou a pravou stranu některých rovností anebo nerovností.
\end{itemize}
První pravidlo je obsaženo i v článku \cite{claessen03paradox},
druhé je obsaženo pouze v implementaci.

\subsection{Definice termů}

Abychom zploštili klauzuli
\[
  \func(\funcG(\funcC)) \approx \var,
\]
musíme do ní přidat dvě nové proměnné:
\[
  \funcC \not\approx \varZ \vee
    \funcG(\varZ) \not\approx \varY \vee
    \func(\varY) \approx \var.
\]
Je-li velikost každé domény $n$, pak tato klauzule bude
zakódována do $n^3$ vý\-ro\-ko\-vých klauzulí o 3 literálech.

Článek \cite{claessen03paradox} ukazuje způsob, jak to udělat
úsporněji. Pro term $\funcG(\funcC)$ zavedeme zkratku $\funcC'$.
Původní klauzuli nahradíme klauzulí
\[
  \func(\funcC') \approx \var
\]
a přidáme klauzuli
\[
  \funcC' \approx \funcG(\funcC),
\]
která zaručí, že $\funcC'$ je zkratka za $\funcG(\funcC)$.
Zploštěním těchto klauzulí dostaneme
\begin{gather*}
  \funcC' \not\approx \varY \vee \func(\varY) \approx \var, \\
  \funcC \not\approx \varY \vee
    \funcG(\varY) \not\approx \var \vee
    \funcC' \approx \var.
\end{gather*}
Počet výrokových klauzulí bude $2 \cdot n^2$ a počet literálů
$5 \cdot n^2$.

Jako zkratka slouží funkční symbol\footnote{Modely, jenž jsou
výstupem programu, musí interpretovat právě symboly z původní vstupní množiny
klauzulí $\clauses$. Pomocné symboly zavedené
při transformování množiny $\clauses$ jako například zkratka $\funcC'$
nesmí být součástí výstupních modelů. Naopak, pokud symbol
z původní vstupní množiny $\clauses$ v důsledku určitých transformací
z $\clauses$ vypadne, musí být k výstupnímu modelu přidán.
Příkladem transformace, která může vyřadit symboly z $\clauses$,
je zjednodušování -- odstraňuje klauzule, jenž jsou vždy splněné
(klauzule obsahující literál $\term \approx \term$ nebo literály
$\lit$ a $\neg \lit$), odstraňuje klauzule, jenž
jsou subsumovány jinými klauzulemi,
a z klauzulí odstraňuje literály, jenž nejsou nikdy splněné
($\term \not\approx \term$).},
který se nevyskytuje v $\clauses$.
Zkratku můžeme zavést za libovolný term bez proměnných
a lze s ní nahradit více výskytů daného termu i v různých klauzulích.

\subsection{Rozdělování klauzulí}

Třetí modifikace, která snižuje počty proměnných v klauzulích,
je rozdělování klauzulí. Na rozdíl od předchozích dvou modifikací
se používá nejen pro metodu MACE, ale i pro metodu SEM.
Princip této metody je použít více klauzulí k~nahrazení jedné
klauzule $\clause \in \clauses$.
Samozřejmě, každá z nových klauzulí, jimiž nahrazujeme klauzuli $\clause$,
musí obsahovat méně proměnných než $\clause$.

Začneme příkladem. Rozdělíme klauzuli z $\clauses$
\[
\clause = \pred_1(\var) \vee \pred_2(\varY) \vee \pred_3(\var, \varZ)
\]
na dvě klauzule s menším množstvím proměnných.
První dva literály $\clause$ dáme do první klauzule
a zbylý literál $\clause$ do druhé klauzule:
\begin{align*}
  \clause_1 &= \pred(\var) \vee \pred_1(\var) \vee \pred_2(\varY) \\
  \clause_2 &= \neg \pred(\var) \vee \pred_3(\var, \varZ).
\end{align*}
$\pred$ je predikátový symbol, jenž se nevyskytuje v $\clauses$.
Propojovací literály $\pred(\var)$ a $\neg \pred(\var)$ zajišťují,
že každý model původní klauzule $\clause$ jde převést na model
$\clause_1$ a $\clause_2$ rozšířením o vhodnou interpretaci $\pred$.
Pro převod modelu opačným směrem stačí odstranit interpretaci $\pred$.
Klauzuli se třemi proměnnými jsme rozdělili na dvě klauzule po dvou
proměnných. Navíc klauzuli $\clause_1$ můžeme rozdělit
na dvě klauzule po jedné proměnné.

Obecně můžeme klauzuli $\clause \in \clauses$ s literály
$\lit_1 \comdots \lit_n$ rozdělit na dvě klauzule
\begin{align*}
  \clause_1 &= \pred(\var_1 \comdots \var_n) \vee
    \lit_1 \vee \cdots \vee \lit_i, \\
  \clause_2 &= \neg \pred(\var_1 \comdots \var_n) \vee
    \lit_{i+1} \vee \cdots \vee \lit_n,
\end{align*}
kde $\pred$ je predikátový symbol, jenž se nevyskytuje v $\clauses$,
a $\var_1 \comdots \var_n$ jsou proměnné, jenž se vyskytují v
$\lit_1 \comdots \lit_i$ a zároveň v $\lit_{i+1} \comdots \lit_n$.
Rozdělení provádíme pouze v případě, že obě klauzule mají
méně proměnných než $\clause$.

Otázkou je, jak rozdělit literály původní klauzule $\clause$ mezi
$\clause_1$ a $\clause_2$. Algoritmus z článku \cite{tammet03}
od autora programu Gandalf prochází všechny podmnožiny
proměnných z $\clause$ a pro každou podmnožinu $\vars'$
zkouší literály z $\clause$ rozdělit do dvou skupin tak,
aby proměnné sdílené oběma skupinami $\var_1 \comdots \var_n$
byly právě proměnné z $\vars'$. Podmnožiny jsou procházeny
v pořadí od nejmenší po největší -- díky tomu algoritmus
najde podmnožinu minimální velikosti
a predikátový symbol $\pred$ má minimální aritu.

% Arita není nejmenší, je pouze minimální -- arity jsou posloupnosti
% a posloupnosti se stejnou délkou jsou neporovnatelné.

Program Paradox \cite{claessen03paradox} používá jiný postup.
Řekneme, že dvě proměnné jsou v~klauzuli spojené, pokud
existuje literál v němž se obě vyskytují.
Paradox hledá proměnnou $\var$, jenž je spojena s minimem proměnných.
Do klauzule $\clause_1$ dá právě literály obsahující $\var$.
Rozdělení se provede pouze, pokud $\var$ není spojena se všemi proměnnými
(kdyby $\var$ byla spojena se všemi proměnnými, klauzule $\clause_1$
by obsahovala stejný počet proměnných jako $\clause$).

\subsection{Redukce symetrií}

Doposud jsme při hledání jednoho modelu nebo všech neizomorfních modelů
prozkoumávali všechna možná ohodnocení buněk. Nyní si ukážeme,
že to není třeba, že prozkoumáním určitých ohodnocení získáme
informace i o jiných ohodnoceních, která pak prozkoumávat už nemusíme.

Například, když prozkoumáme interpretaci $\interp$,
nemusíme již prozkoumávat interpretace, které jsou s $\interp$ izomorfní.
Toto jednoduché pozorování lze zobecnit pro stavy řešeného
problému.
Mějme dva stavy řešeného problému z metody SEM
$(A_\funcs, A_\preds, U_\funcs, U_\preds, E)$
a $(A_\funcs', A_\preds', U_\funcs', U_\preds', E')$.
Řekneme, že tyto stavy jsou izomorfní,
pokud zpermutováním prvků domén\footnote{Buňky v $A_\funcs, A_\preds$
i podmínky $E$ obsahují prvky domén.
Pokud máme pro každou doménu permutaci,
můžeme tyto permutace použít k přejmenování prvků domén,
jenž se vyskytují $(A_\funcs, A_\preds, E)$.}
dostaneme z trojice $(A_\funcs, A_\preds, E)$
trojici $(A_\funcs', A_\preds', E')$.
Použité permutace tvoří dohromady jednu symetrii.
Jestliže jsou dva stavy řešeného problému izomorfní,
získáme jejich prohledáním izomorfní modely.

% Množiny $E$ a $E'$ se nemusí shodovat,
% stačí, když budou ekvivalentní.

Dalším příkladem jsou symetrické symboly. Uvažme
množinu klauzulí $\clauses$, která obsahuje
predikátové symboly $\pred$ a $\pred'$ takové,
že jejich záměnnou se $\clauses$ nezmění.
Pokud najdeme model $\clauses$, tak záměnnou interpretací
$\pred$ a $\pred'$ získáme jiný model $\clauses$, který
v obecném případě není izomorfní původnímu modelu.
Toto pozorování lze použít i naopak, například,
pokud zjistíme, že žádný model neobsahuje nějakou interpretaci pro $\pred$,
pak automaticky dostaneme, že žádný model neobsahuje tuto interpretaci
ani pro $\pred'$.

Chceme-li metodu SEM modifikovat, aby neprozkoumávala některá
ohodnocení, o nichž získá nebo získala informace
z jiných ohodnocení, máme dvě možnosti:
\begin{itemize}
\item Můžeme upravit množinu podmínek $E$ tak,
  aby některé symetrie přestaly existovat,
\item nebo upravit funkci \textproc{Search}, aby
  neprohledávala symetrické stavy.
\end{itemize}
První možnosti se říká statická modifikace, neboť
pracuje pouze na začátku před spuštěním funkce \textproc{Search}.
Druhé možnosti se říká dynamická modifikace,
jelikož pracuje za běhu funkce \textproc{Search}.

Statická modifikace u metody MACE upraví výrokové klauzule,
dynamická modifikace upravuje samotný SAT řešič.
Nyní popíšeme konkrétní modifikace.

\subsubsection{LNH}

Modifikace LNH \cite{lnh} je založena na následujícím principu:
Uspořádáme-li všechny buňky funkcí do
posloupnosti $c_1 \comdots c_n$,
tak ke každému modelu existuje\footnote{
Takový model můžeme snadno zkonstruovat. Ohodnocené buňky
zapíšeme do posloupnosti $c_1 = v_1 \comdots c_n = v_i$.
Tuto posloupnost procházíme zleva doprava a kdykoliv
najdeme $c_i = v_i$, kde $v_i$ je nepoužitá hodnota, ale není to nejnižší
nepoužitá hodnota $v$, tak v celé posloupnosti zaměníme
$v$ a $v_i$. Tato záměna se nedotkne buněk před $c_i$,
neboť tam se hodnoty $v$ a $v_i$ nevyskytují. U buňky $c_i$ se změní
pouze přiřazená hodnota. Záměna však pokazí pořadí buněk po $c_i$,
pokud se tam $v$ a $v_i$ vyskytují jako argumenty.
Požadovaný model získáme, až dojdeme na konec posloupnosti.
Ze záměn můžeme složit permutace, čímž dostaneme izomorfismus modelů.}
izomorfní model,
kde je každé buňce $c_i$ přiřazena buď nějaká použitá hodnota
nebo nejnižší nepoužitá hodnota.
Mezi použité hodnoty pro buňku $c_i$ patří hodnoty, jenž se vyskytují
jako argument buněk $c_1 \comdots c_i$, a hodnoty
přiřazené buňkám $c_1 \comdots c_{i-1}$.
Pozor, hodnoty z~různých domén chápeme jako různé hodnoty.

Buňka $c_i$ tedy může bez omezení nabývat hodnot, jenž se vyskytují mezi
argumenty $c_1 \comdots c_i$, a také nejnižší hodnoty $v_1$, která se mezi
argumenty $c_1 \comdots c_i$ nevyskytuje.
Jsou-li $v_1 < \cdots < v_{n'}$ hodnoty, jenž jde přiřadit buňce $c_i$
a jenž se nevyskytují mezi argumenty $c_1 \comdots c_i$, pak
buňka $c_i$ může obsahovat hodnotu $v_k$ pro $k \ge 2$ pouze tehdy,
pokud nějaká buňka před $c_i$ obsahuje hodnotu $v_{k-1}$.

Z výše uvedeného vyplývá, jak LNH implementovat jako statickou modifikaci.
Pro buňku $c_i$ a hodnotu
$v_k$, kde $k \in \{ 2 \comdots n' \}$, přidáme podmínku tvaru
\[
c_i \approx v_k \implies
  \bigvee_{\substack{
      j \in \{ 1 \comdots i-1 \}, \\
      \text{$c_i$ a $c_j$ mají stejnou} \\
      \text{sortu výsledku}}}
    c_j \approx v_{k-1}.
\]
$c_j$ označuje buňky nalevo od $c_i$, které mají stejnou sortu
výsledku jako $c_i$, buňky s jinou sortou výsledku
nabývají hodnot z jiných domén, tudíž by rovnost s $v_{k-1}$
nemohla nastat.

Díky těmto podmínkám, nemohou některé buňky nabývat určitých hodnot.
Například, máme-li buňky $\funcC, \func(0, 0), \func(0, 1), \ldots$
a jednu doménu velikosti 3, pak buňka $\funcC$ může nabývat pouze hodnoty
0. Důvodem jsou dvě podmínky, které jsme pro ni vytvořili --
jelikož žádné buňky nejsou nalevo od $\funcC$,
je pravá strana implikace s předpokladem
$\funcC = 1$ resp. $\funcC = 2$ prázdná.

Buňka $\func(0, 0)$ může nabývat pouze hodnot 0, 1. Důvodem
je opět podmínka, kterou jsme pro ni vytvořili:
\[
  \func(0, 0) \approx 2 \implies \funcC \approx 1.
\]
Pravá strana implikace nikdy neplatí, neboť $\funcC = 0$.

Skutečnosti, že některé buňky nenabývají určitých hodnot,
můžeme využít v~metodě MACE při generování výrokových klauzulí.
Jednak pro tyto buňky není třeba generovat některé výrokové proměnné.
Dále klauzule z bodu 1c) pro tyto buňky mohou obsahovat méně literálů
a není třeba je odstraňovat při zvětšení domény. Viz také
\cite{claessen03paradox}.

Velmi důležitou roli hraje pořadí buněk. Například, kdybychom dali
buňku $\func(0, 1)$ jako první, tj.
$\func(0, 1), \funcC, \func(0, 0), \ldots$,
tak by každá buňka mohla nabýt každé hodnoty z domény.
Výroková klauzule z bodu 1c) pro buňku $\func(0, 1)$
by však platila i po zvětšení domény na velikost 4 -- klauzuli
bychom nemuseli odstraňovat.

Pro metodu MACE se LNH obvykle implementuje jako statická modifikace
-- je to jednodušší než upravovat SAT řešič.
Na druhé straně pro metodu SEM se LNH obvykle
implementuje dynamicky -- ve funkci \textproc{Search} stačí nahradit

\medskip
\begin{algorithmic}
  \For{$v \in D$}
    \State \Call{Search}{$A_\funcs \cup \{ (c, v) \}, A_\preds,
      U_\funcs \setminus \{ (c, D) \}, U_\preds, E$}
  \EndFor
\end{algorithmic}
\medskip

\noindent pomocí (liší se pouze první řádek)

\medskip
\begin{algorithmic}
  \For{$v \in D' \cup D'' $}
    \State \Call{Search}{$A_\funcs \cup \{ (c, v) \}, A_\preds,
      U_\funcs \setminus \{ (c, D) \}, U_\preds, E$}
  \EndFor
\end{algorithmic}
\medskip

\noindent
kde $D'$ obsahuje hodnoty z $D$, jenž se vyskytují v $A_\funcs$ nebo
v argumentech $c$,
a $D''$ obsahuje nejmenší hodnotu z $D \setminus D'$,
pokud taková hodnota existuje.
$D'$ je tedy množina použitých hodnot a $D''$ je buď prázdná množina,
nebo obsahuje nejmenší nepoužitou hodnotu.

Pořadí buněk pro LNH je tedy dáno pořadím, v němž byly
buňky ohodnoceny, což znamená, že v různých větvích
prohledávacího stromu může být různé pořadí buněk pro LNH.
Abychom mohli omezit výběr hodnot, tj. omezit velikost $D'$,
je vhodné při výběru neohodnocených buněk z $U_\funcs$
dávat přednost buňkám, jejichž argumenty se již vyskytují v $A_\funcs$.


\subsubsection{XLNH}

Modifikace XLNH \cite{xlnh} redukuje symetrie u problémů
s funkčním symbolem $\func$ arity $\langle \sort, \sort \rangle$.
XLNH lze implementovat jako dynamickou modifikaci metody SEM:

Implementace postupně generuje všechna navzájem
neizomorfní ohodnocení všech buněk
funkcí bez argumentů a všech buněk funkce $\func^\interp$.
S každým takovým ohodnocením $A_\funcs$ je pak spuštěna upravená funkce
\textproc{Search}, která vygeneruje ohodnocení pro zbylé buňky.
Při ohodnocování zbylých buněk využívá upravená funkce \textproc{Search}
rozklad $\func^\interp$ na cykly
k redukci dalších symetrií. Pokud $\func^\interp$ není bijekce,
použije se její bijektivní restrikce.

Článek \cite{xlnh} jednak ukazuje, jak systematicky
generovat neizomorfní ohodnocení všech buněk
funkcí bez argumentů a všech buněk funkce $\func^\interp$.
Dále ukazuje, jak použít $\func^\interp$ k redukci dalších symetrií.

Na rozdíl od modifikace LNH se XLNH staticky neimplementuje.
Důvodem je velké množství podmínek resp. výrokových klauzulí,
jenž by bylo nutné přidat.

\subsubsection{DASH}

Představené modifikace LNH a XLNH nejsou obecně schopny eliminovat všechny
izomorfní modely. Pokud bychom něčeho takového chtěli dosáhnout
v metodě SEM, mohli bychom si pamatovat všechny prohledané
stavy řešeného problému a před prohledáváním nového stavu,
se podívat, zda již nebyl prohledán izomorfní stav.
Taková modifikace by vskutku eliminovala veškeré izomorfní modely
a mohla by tak nahradit modifikace LNH a XLNH -- viz
\cite{audemard2001symmetry}.
Problém spočívá v tom, že stavů řešeného problému
je velmi mnoho, tudíž hledat mezi uloženými stavy izomorfní stav může
dlouho trvat a navíc se tolik stavů ani nemusí vejít do paměti
běžného počítače.

S lepším řešením přišli autoři modifikace DASH \cite{dash}.
Jelikož je na začátku množina podmínek $E$ invariantní
při zpermutování prvků domén, není třeba tuto množinu uvažovat.
Buňky, jimž přiřadila hodnoty funkce \textproc{Propagate},
rovněž není třeba uvažovat, stačí uvažovat pouze buňky,
jimž přiřadila hodnoty funkce \textproc{Search}.
A nakonec, není třeba si pamatovat všechny stavy,
některé stavy jsou redundantní.

Modifikace DASH začíná ohodnocením konstant.
Konstanty jsou ohodnoceny jako při LNH -- jsou uspořádány do posloupnosti
a každé je buď přiřazena hodnota použitá nějakou předchozí konstantou,
nebo nejmenší nepoužitá hodnota. Hodnoty přiřazené konstantám
nejsou záměnné, ostatní hodnoty jsou záměnné.
Prohledané stavy reprezentuje DASH schématy. Schéma je množina
obsahující prvky z $A_\funcs$ a $A_\preds$, jimž byla přiřazena
hodnota funkcí \textproc{Search}. Před prohledáváním jiného
stavu\footnote{Po volání \textproc{Propagate} ve funkci \textproc{Search} --
použití propagace zvyšuje šanci na nalezení schématu, jenž
stavu odpovídá.}
s ohodnocením $A'_\funcs$ a $A'_\preds$ se zkusí najít schéma,
jemuž tento stav odpovídá. Nalezení takového schématu znamená,
že byl prohledán izomorfní stav řešeného problému, tudíž
stav s ohodnocením $A'_\funcs$ a $A'_\preds$ prohledávat nemusíme.
Po prohledání stavu\footnote{Na konci funkce \textproc{Search},
pokud nenastal konflikt.} přidáme schéma,
jenž daný stav reprezentuje,
a odebereme schémata přidaná pro jeho
podstavy\footnote{Schémata přidaná v rekurzivních voláních pro stavy,
jenž vznikly z aktuálního stavu.}.

Dvě schémata jsou izomorfní,
pokud jedno schéma dostaneme z druhého zpermutováním prvků domén mimo
prvků, jenž byly přiřazeny konstantám.
Stav s~ohodnocením $A'_\funcs$ a $A'_\preds$ odpovídá schématu, pokud
prvky nějakého izomorfního schématu leží v množině $A'_\funcs \cup A'_\preds$.

V praxi se ukázalo, že snaha eliminovat všechny izomorfní modely
metodou DASH se nevyplatí. Proto autoři \cite{dash}
v závislosti na hloubce rekurze \textproc{Search} omezují,
jaká schémata se zkouší.

% Hloubka rekurze je hloubka prohledávacího stromu.

DASH se staticky neimplementuje, důvody jsou stejné jako pro XLNH.

\subsection{Omezení velikosti domén}

Dalším trikem, jak zmenšit prohledávaný prostor, je omezit velikost
domén ně\-kte\-rých sort \cite{claessen03paradox}.
Předpokládejme, že hledáme model, kde $n > k$
je velikost domény $D_\sort$, všechny použité funkční symboly
$\funcC_1 \comdots \funcC_k$ se sortou
výsledku $\sort$ jsou bez argumentů a navíc klauzule neobsahují literál
tvaru $\term \approx \term'$, kde $\term, \term'$ jsou termy sorty
$\sort$. Takový model existuje právě tehdy, když
existuje model, kde doména $D_\sort$ má velikost $k$.

Pokud platí předpoklady, lze doménu $D_\sort$ velikosti
$n$ zmenšit odebráním hodnot, jenž nejsou přiřazeny žádné z buněk
$\funcC_1 \comdots \funcC_k$.
Naopak, když máme model s~doménou $D_\sort$ velikosti $k$,
jenž obsahuje prvek $v$, můžeme tuto doménu rozšířit o~hodnoty
$v_1 \comdots v_j$, přičemž hodnotu každé nové buňky $c$ získáme
tak, že výskyty hodnot $v_1 \comdots v_j$ v argumentech buňky $c$ nahradíme
hodnotou $v$, čímž dostaneme jinou buňku $c'$, hodnotu $c$
definujeme jako hodnotu $c'$.

Pozorování můžeme upravit pro případ, kdy klauzule navíc obsahují
literály $\term \approx \term'$,
kde $\term, \term'$ jsou termy sorty $\sort$ a alespoň jeden
z nich není proměnná.
V~takovém případě model s doménou $D_\sort$ velikosti $n$ existuje
právě tehdy, když existuje model s doménou $D_\sort$ velikosti $k + 1$.

Rovnosti termů sorty $\sort$ nepředstavují žádný problém při
zmenšování domény $D_\sort$ z velikosti $n$ na $k + 1$. Při zvětšování domény
víme, že existuje hodnota $v \in D_\sort$, která není přiřazena žádné z buněk
$\funcC_1 \comdots \funcC_k$. Klauzule, které obsahují rovnosti
termů sorty $\sort$, budou splněny i pro hodnoty přidané do domény,
jelikož jsou splněny i pro hodnotu $v$.

Poznamenejme, že ač je klauzule
\[
  \var \approx \funcC
\]
ekvivalentní s klauzulí
\[
  \var \approx \varY \vee \varY \not\approx \funcC,
\]
tak u první klauzule se podaří omezit velikost sorty a u druhé
klauzule se to nepodaří -- druhá klauzule obsahuje rovnost proměnných.
Postup, jak model s menší doménou převést na model s větší doménou
nezaručuje vygenerování všech neizomorfních modelů.
Omezování velikostí domén tedy použijeme pouze v režimu,
kdy hledáme jediný model.

\subsection{Inference sort}

Metody MACE a SEM a jejich modifikace podporují vstupy s více než
jednou sortou. Vstupní množina klauzulí $\clauses$ však obsahuje
nejvýše jednu sortu. Nyní se podíváme, jak vstupní množinu
$\clauses$ transformovat
na množinu $\clauses'$, která potenciálně
obsahuje více sort než $\clauses$.

Pokud bylo původním úkolem najít model $\clauses$, kde je velikost
jediné domény $n$, tak nyní budeme hledat model $\clauses'$,
kde je velikost každé domény $n$. Takový vícesortový
model snadno převedeme na model jednosortový -- stačí zapomenout,
že argumenty a hodnoty buněk jsou z různých sort.

Motivací pro $\clauses'$ s více sortami je skutečnost,
že některé modifikace mají větší šanci omezit prohledávaný prostor,
když $\clauses'$ obsahuje více sort. Například
redukce symetrií LNH má k dispozici více nepoužitých hodnot,
neboť hodnoty z různých domén chápeme jako různé.
Jiným příkladem je omezování velikosti domén -- velikost původní domény
nemusí jít omezit, ale velikost některé z nových domén už ano.

Vraťme se k samotnému převodu $\clauses$ na $\clauses'$.
Používáme techniku z \cite{claessen03paradox}.
Označme
\[
  s = \{ s_{a, n} \mid  a \in \funcs \cup \preds \cup \vars, n \in \nat\}.
\]
Množina $\clauses'$ vznikne z množiny $\clauses$ tak, že
\begin{itemize}
\item každý funkční symbol $\func$ arity
  $\langle \sort_1 \comdots \sort_n, \sort \rangle$
  nahradíme funkčním symbolem $\func'$ arity
  $\langle [s_{\func, 1}]_E \comdots [s_{\func, n}]_E, [s_{\func, n + 1}]_E \rangle$,
\item každý predikátový symbol $\pred$ arity
  $\langle \sort_1 \comdots \sort_n \rangle$
  nahradíme predikátovým symbolem $\pred'$ arity
  $\langle [s_{\pred, 1}]_E \comdots [s_{\pred, n}]_E \rangle$
\item a každou proměnnou $\var \in \vars_\sort$ nahradíme
  proměnnou $\var' \in \vars_{[s_{\var, 1}]_E}$,
\end{itemize}
kde prvky $s_{a, n} \neq s_{a', n'}$ pokud $a \neq a'$ nebo $n \neq n'$
a kde $E$ je nejjemnější ekvivalence na $s$, že když prvky $s/E$
ztotožníme se sortami (různým prvkům $s/E$ budou odpovídat různé sorty),
tak $\clauses'$ bude obsahovat klauzule.

$\clauses'$ by neobsahovala klauzule v případě, že by nesouhlasily
sorty -- term sorty $\sort$ by byl na pozici,
kde je vyžadován term sorty $\sort' \neq \sort$. Kdyby $E$ byla ekvivalence
s jedním blokem, tak by se v $\clauses'$ vyskytovala nejvýše jedna sorta.
Chceme nejjemnější ekvivalenci, aby se v $\clauses'$ vyskytovalo
co nejvíce sort. Požadovanou ekvivalenci lze najít algoritmem union-find.

\section{Další metody hledání modelů}

Metody MACE a SEM jsou úspěšné při řešení mnoha praktických problémů
-- viz například výsledky soutěže CASC \cite{sutcliffe2006casc}.
Na druhé straně existuje i řada poměrně jednoduchých problémů,
s nimiž si tyto metody neporadí. Jedním z takových problémů
je najít model následujících klauzulí:
\begin{gather*}
  \func(\var_1 \comdots \var_{128}) \approx \funcC, \\
  \funcC \not\approx \funcC'.
\end{gather*}
Klauzule mají zjevně model s doménou velikosti 2, kde funkce $\func^\interp$
je konstantní. Metody MACE a SEM však žádný model nenajdou.
Problém spočívá v explicitní reprezentaci tabulky funkce $\func^\interp$.
Pro doménu velikosti 2 obsahuje tato tabulka $2^{128}$ buněk,
takže se ani nevejde do paměti počítače.

Další nepříjemnost metod MACE a SEM uvádí \cite{hillenbrand2013superposition}.
Mějme tyto klauzule:
\begin{gather*}
\func(\var) \approx \var, \\
\pred(\func(\funcG(\var))).
\end{gather*}
V logice prvního řádu, lze
druhou klauzuli zjednodušit pomocí první klauzule na $\pred(\funcG(\var))$.
Pokud klauzule transformujeme na podmínky pro SEM, dostaneme:
\begin{gather*}
\func(\var) \approx v, \\
\pred(\func(\funcG(v))),
\end{gather*}
kde $v$ je prvek domény, tj. číslo z $\natZ$.
Pak ovšem podmínky vzešlé z první klauzule již nejde\footnote{Teoreticky
se  můžeme vrátit k logice prvního řádu a z podmínek
odvodit první klauzuli. Nicméně nevíme o praktické implementaci
metody SEM, která by toto dělala.} použít
ke zjednodušení podmínek z druhé klauzule, důvodem je, že
$\funcG(v)$ není číslo.

Oba příklady naznačují, že by některé problémy bylo vhodné řešit
přímo v~logice prvního řádu. Budeme tedy postupovat podobně
jako u metody MACE, ale místo do výrokové logiky převedeme
problém do logiky prvního řádu, kde z funkčních symbolů
jsou pouze symboly bez argumentů \cite{fmdarwin}.

Začneme zploštěním všech klauzulí. Na rozdíl od zploštění,
jenž jsme popsali pro metodu MACE, budeme navíc požadovat, aby
rovnost termů byla pouze mezi proměnnými.

Dále všechny funkční symboly nahradíme predikátovými symboly.
Pro každý funkční symbol $\func$
vezmeme dosud nepoužitý predikátový symbol
$\pred_\func$ stejné arity. Každý literál tvaru
$\func(\var_1 \comdots \var_n) \not\approx \var$
(resp. $\var \not\approx \func(\var_1 \comdots \var_n)$)
nahradíme literálem $\neg \pred(\var_1 \comdots \var_n, \var)$.

% Alternativně jde pro každý funkční symboly zavést konstantu,
% a pro všechny funkční symboly s n argumenty
% zavést jeden predikátový symbol s n+2 argumenty.
% Pak f(x1 .. xn) != y nahradíme !P(c, x1 .. xn, y).

Poté pro každý prvek $v$ každé domény $D_\sort$ vezmeme
dosud nepoužitý funkční symbol $c_{\sort, v}$ arity $\langle \sort \rangle$.
Pro každé dva funkční symboly $c_{\sort, v} \neq c_{\sort, v'}$
přidáme klauzuli $c_{\sort, v} \not\approx c_{\sort, v'}$.

% Metoda převádí problém pro dokazovač FM-Darwin, který neumí rovnost,
% tudíž tam přidává i opračné nerovnosti
% c_{\sort, v'} \not\approx c_{\sort, v}.

Každá funkce má v každém bodě právě jednu hodnotu -- tento
invariant garantovaly u metody MACE klauzule přidané v bodech 1b) a 1c).
Nyní nám stačí garantovat pouze to, že každá \uv{funkce}
(reprezentovaná relací) má v každém bodě alespoň jednu hodnotu,
což pro každý funkční symbol $\func$ z původního problému vyjádříme klauzulí
\begin{equation} \label{eq:totality}
  \bigvee_{\substack{v \in D_\sort, \\ \text{kde $\sort$ je sorta výsledku $\func$}}}
    \pred_\func(\var_1 \comdots \var_n, c_{\sort, v}).
\end{equation}
Díky tomu, že se $\pred_\func$ ve všech klauzulích kromě
klauzule (\ref{eq:totality}) vyskytuje pouze v negaci,
můžeme nalezené modely transformovat tak, že $\pred_\func$
bude reprezentovat skutečnou funkci.

Jelikož takto zakódovaný problém neobsahuje funkční symboly
s argumenty, jsou domény Herbrandových interpretací konečné.
Abychom tedy našli konečný model, stačí problém předat metodě
pro hledání Herbrandových modelů.
Popsané kódování používají například programy
FM-Darwin \cite{fmdarwin}
a iProver\footnote{Pouze v režimu hledání konečných modelů.} \cite{iprover}.

Samozřejmě metody pro hledání Herbrandových modelů lze použít
i bez popsaného kódování, pak ovšem nalezené modely nemusí
mít konečné domény. Rozsáhlý přehled metod automatického dokazování
v logice prvního řádu za\-lo\-že\-ných na budování
modelů je \cite{bonacina2015}. Tento přehled mj. zmiňuje
i metody Model Evolution \cite{modelevolution}
a Inst-Gen \cite{instgen}, jenž jsou implementovány
v programech FM-Darwin a iProver.

\chapter{Implementace}

Cílem této kapitoly je popsat náš hledač konečných modelů, který
se jmenuje \crossbow. \crossbow{} implementuje metodu MACE
a některé její modifikace.
Kromě modifikací obsažených v minulé kapitole jsou implementovány
i úplně nové, dosud nevyzkoušené, modifikace.
Tato kapitola začíná popisem nových modifikací metody MACE.
Poté následuje stručný popis samotné implementace.

Hlavním důvodem, proč jsme se rozhodli založit program
\crossbow{} právě na metodě MACE,
byly výsledky metody MACE při řešení problémů --
například program Paradox \cite{paradox} vyhrál
6 ročníků soutěže CASC \cite{sutcliffe2006casc}. Další příjemnou vlastností
metody MACE je, že funguje s existujícími SAT řešiči --
není třeba vytvářet specializovaný řešič jako u metody SEM.

\section{Další modifikace metody MACE}

\subsection{Zplošťování}

Jako první popíšeme algoritmus zplošťování používaný
programem \crossbow. Pravidla pro zplošťování klauzulí
uvedená v předchozí kapitole lze použít mnoha různými způsoby.
Například klauzuli
\begin{equation} \label{eq:flatten-input}
  \func(\var) \approx \funcC \vee \funcG(\var) \approx \funcC
\end{equation}
můžeme zploštit na
\begin{equation} \label{eq:flatten-bad-output}
  \func(\var) \not\approx \varY \vee \funcG(\var) \not\approx \varZ \vee
    \varY \approx \funcC \vee \varZ \approx \funcC
\end{equation}
nebo na
\begin{equation} \label{eq:flatten-good-output}
  \funcC \not\approx \varY \vee
    \func(\var) \approx \varY \vee \funcG(\var) \approx \varY.
\end{equation}
V prvním případě byly přidány dvě nové proměnné, ve druhém případě
pouze jedna nová proměnná. Náš algoritmus se pochopitelně snaží
přidat co nejméně proměnných.

Vstupem algoritmu je klauzule, jenž se má zploštit.
Výstupem algoritmu je buď plochá klauzule nebo rozhodnutí,
že se má klauzule odstranit, neboť je vždy splněná.

Algoritmus se skládá ze sedmi kroků, které se, pokud
není řečeno jinak, vykonávají postupně.
Každý krok se skládá z předpokladu a akce.
Akce se provádí pouze v případě, že je splněn předpoklad.
Pokud není splněn předpoklad, pokračuje se dalším krokem.
Některé kroky využívají tzv. nové proměnné, což
jsou proměnné, které se v klauzuli nevyskytovaly
předtím, než se daný krok začal vykonávat.

{
\def\assumpt{\textbf{Předpoklad:}}
\def\action{\textbf{Akce:}}
\def\goto#1{Dále se pokračuje krokem #1).}
\begin{itemize}
\item[1)]
\assumpt{} Klauzule obsahuje literál $\term \approx \term$ nebo
literály $\lit$ a $\neg \lit$ nebo literály $t \approx s$
a $s \not\approx t$.

\action{} Klauzule se odstraní, zplošťování končí.

\item[2)]
\assumpt{} Klauzule obsahuje literál $\lit$ tvaru $\term \not\approx \term$.

\action{} Literál $\lit$ se odstraní z klauzule.
\goto{2}

\item[3)]
\assumpt{} Klauzule obsahuje literál
$\lit$ tvaru $\var \not\approx \varY$.

\action{} Z klauzule se odstraní literál $\lit$ a všechny
výskyty $\var$ se nahradí $\varY$.
\goto{1}

% Musí se pokračovat krokem 1):
% Máme-li x != y | f(y) = z | f(x) != z, tak zrušením rovnosti proměnných
% dostaneme $f(y) = z | f(y) != z$.

\item[4)]
\assumpt{} Klauzule obsahuje literál $\lit$ tvaru
$\term \not\approx \var$ (resp. $\var \not\approx \term$) a
$\term$ se vyskytuje mimo $\lit$.

\action{} Všechny výskyty $\term$ mimo výskytu v $\lit$ se nahradí $\var$.
\goto{1}

% $\term$ má v $\lit$ právě jeden výskyt
% (díky 3) totiž víme, že $\term$ není proměnná).

\item[5)]
\assumpt{} Klauzule obsahuje atom
$\pred(\ldots, \term, \ldots)$ nebo
$\func(\ldots, \term, \ldots) \approx \termS$
(resp. $\termS \approx \func(\ldots, \term, \ldots)$),
kde $\term$ není proměnná.

\action{} Do klauzule se přidá literál $\term \not\approx \var$, kde
$\var$ je nová proměnná.
\goto{4}

\item[6)]
\assumpt{} Klauzule obsahuje literál $\lit$ tvaru
$\term \not\approx \termS$, kde $\term$ a $\termS$ nejsou
proměnné.

\action{} Z klauzule se odstraní literál $\lit$ a přidají
se tam literály $\term \not\approx \var$ a $\var \not\approx \termS$,
kde $\var$ je nová proměnná.
\goto{4}

\item[7)]
Nechť
$\term_1 \approx \term_2 \comdots \term_{2n-1} \approx \term_{2n}$
jsou všechny literály z klauzule, jenž mají tvar $\term \approx \termS$,
kde $\term$ ani $\termS$ nejsou proměnné.
(Tj. $\term_i$ pro každé $i \in \{ 1 \comdots 2n \}$ není proměnná.)

\assumpt{} $n \ge 1$.

\action{} Vytvoří se neorientovaný graf s vrcholy
$\term_1 \comdots \term_{2n}$. Mezi vrcholy $\term_i$ a $\term_j$ vede hrana
právě tehdy, když literál $\term_i \approx \term_j$ je v klauzuli.
Najde se minimální vrcholové pokrytí $\term_{i_1} \comdots \term_{i_k}$.
Do klauzule se přidá literál $\term_{i_j} \not\approx \var_j$ pro
každé $j \in \{ 1 \comdots k \}$, kde $\var_1 \comdots \var_k$ jsou
navzájem různé nové proměnné.
\goto{4}
\end{itemize}
}

Například u klauzule (\ref{eq:flatten-input}) se v pravidle 7)
vytvoří graf s vrcholy $\func(\var), \funcC, \funcG(\var)$
a hranami $\{ \func(\var), \funcC \}$ a $\{ \funcG(\var), \funcC \}$.
Minimální vrcholové pokrytí tohoto grafu je vrchol $\funcC$,
proto se do klauzule přidá literál $\funcC \not\approx y$.
Dále se pokračuje krokem 4), který nahradí výskyty $\funcC$ mimo literál
$\funcC \not\approx y$, čímž dostaneme klauzuli
(\ref{eq:flatten-good-output}).

Akci v pravidle 6) bychom mohli změnit tak, že by se literál $\lit$
neodstraňoval a pouze by se přidal jeden z literálů
$\term \not\approx \var$, nebo $\var \not\approx \termS$.
Pravidlo 4) by pak s~pomocí přidaného literálu změnilo literál $\lit$
na nepřidaný literál.

Všimněme si asymetrie mezi rovností a nerovností neproměnných.
Pravidlo 6) přidáním jedné nové proměnné získá
dva literály $\term \not\approx \var$ a $\var \not\approx \termS$,
které může použít pravidlo 4) k nahrazení termů $\term$ a $\termS$
proměnnou $\var$.

Na druhé straně pravidlo 7) použité
na rovnost neproměnných $\term \approx \termS$ přidáním jedné
nové proměnné získá pouze jeden z literálů $\term \not\approx \var$,
nebo $\termS \not\approx \var$, které může použít pravidlo 4).
Abychom získali i druhý literál, museli bychom přidat další novou proměnnou.
Na rozdíl od pravidla 6) tedy pravidlo 7) musí pečlivě zvažovat,
zda chce literál pro levou stranu rovnosti, nebo pro pravou stranu
rovnosti, nebo literály pro obě strany rovnosti.

Například pro zploštění klauzule
\[
\funcC \approx \funcC' \vee
  \funcC \approx \funcC_1 \vee \funcC \approx \funcC_2 \vee
  \funcC' \approx \funcC'_1 \vee \funcC' \approx \funcC'_2,
\]
kde $\funcC, \funcC_1, \funcC_2, \funcC', \funcC'_1, \funcC'_2$ jsou
funkční symboly bez argumentů, přidá pravidlo 7)
literály $\funcC \not\approx \var_1$ a $\funcC' \not\approx \var_2$,
jenž obsahují dvě nové proměnné $x_1$ a $x_2$.
Kdyby však první literál klauzule byl nerovnost, tj.
$\funcC \not\approx \funcC'$, tak by pravidlu 6)
stačila pouze jedna nová proměnná.

\subsection{Odzplošťování}

Jelikož preferujeme klauzule s malým
počtem proměnných, snažili jsme se náš zplošťovací algoritmus
navrhnout tak, aby zaváděl co nejmenší množství nových proměnných.
Ovšem i tak existují klauzule, které jde zploštit lépe.
Příkladem takové klauzule je (\ref{eq:flatten-bad-output}).
Tato klauzule již je plochá a náš algoritmus ji nijak nezmění.
Víme však, že existuje ekvivalentní plochá klauzule,
jenž obsahuje méně proměnných, tou klauzulí je
(\ref{eq:flatten-good-output}).

Abychom při vstupu (\ref{eq:flatten-bad-output}) dostali
výstup (\ref{eq:flatten-good-output}),
provedeme odzplošťování před zplošťováním.
Odzplošťování je založeno na stejných pravidlech jako zplošťování
(viz sekce \ref{sec:mace-basic}).
Rozdíl je v tom, že odzplošťování používá pravidla naopak:
\begin{itemize}
\item Je-li $\lit$ literál z $\clause$ tvaru $\term \not\approx \var$
  (resp. $\var \not\approx \term$) takový, že $\var$ není obsažena
  v~$\term$ ani v ostatních literálech $\clause$ mimo $\lit$,
  pak můžeme literál $\lit$ odstranit\footnote{Ve skutečnosti
  se nejedná o inverzní pravidlo k prvnímu pravidlu zplošťování.
  Jednak proto, že při zplošťování se přidávaly literály
  tvaru $\term \not\approx \var$, nyní se však odebírají i literály tvaru
  $\var \not\approx \term$. Dále proto, že při zplošťování
  se přidával literál $\term \not\approx \var$ pouze v případě,
  kdy se term $\term$ vyskytoval v klauzuli $\clause$.
  Nyní se literál $\lit$ odstraňuje, i když se term $\term$ v ostatních
  literálech mimo $\lit$ nevyskytuje.}
  z~klauzule $\clause$.
\item Je-li $\lit$ literál z $\clause$ tvaru $\term \not\approx \var$
  (resp. $\var \not\approx \term$),
  pak můžeme $\clause$ upravit tak, že jeden výskyt $\var$
  mimo $\lit$ nahradíme $\term$.
\end{itemize}

Zkombinováním obou opačných pravidel dostaneme pravidlo odzplošťování:
\begin{itemize}
\item Je-li $\lit$ literál z $\clause$ tvaru $\term \not\approx \var$
  (resp. $\var \not\approx \term$) takový, že $\var$ není obsažena
  v~$\term$, pak můžeme všechny výskyty $\var$ mimo výskytu
  v $\lit$ nahradit $\term$
  a literál $\lit$ odstranit.
\end{itemize}
Odzplošťování klauzule $\clause$ se provádí tak,
že se na ni aplikuje uvedené pravidlo, dokud se klauzule mění.

Například z klauzule (\ref{eq:flatten-bad-output})
vytvoří odzplošťování klauzuli (\ref{eq:flatten-input}),
kterou pak zplošťování převede na klauzuli (\ref{eq:flatten-good-output}).

Poznamenejme, že náš algoritmus zplošťování neobsahuje následující pravidlo
z programu Paradox, jenž jsme uvedli v sekci \ref{sec:flatten-extended}:
\begin{itemize}
\item Klauzuli $\clause = \lit_1 \vee \cdots \vee \lit_n$, kde literál
  $\lit_i$ je tvaru $\var \approx \varY$ a literál $L_j$ je tvaru
  $\term \not\approx \varY$, můžeme nahradit klauzulí
  $\clause' = \term \approx \var \vee
  \bigvee_{k \in \{ 1 \comdots n \} \setminus \{ i, j \}} \lit_k$, pokud $\clause'$
  neobsahuje $\varY$.
\end{itemize}
Místo něj lze totiž použít odzplošťování (term $\term$ neobsahuje
proměnnou $\varY$) ná\-sle\-do\-va\-né zplošťováním.

\subsection{Redundantní klauzule}

Abychom posílili propagaci SAT řešiče, přidáme
ke vstupní množině klauzulí $\clauses$ další klauzule,
jenž plynou z $\clauses$.

Program \crossbow{} tyto klauzule získá pomocí dokazovače
E \cite{eprover18}.
Z klauzulí odvozených dokazovačem E vybere malé klauzule
a přidá je do $\clauses$.

\subsection{Komutativní funkce}

Pokud víme, že funkce $\func^\interp$ je komutativní,
můžeme pro buňky $\func(v_1, v_2)$ a $\func(v_2, v_1)$
používat stejné výrokové proměnné.
To znamená, že při vytváření instance SATu (viz \ref{sec:mace-basic})
budeme v kroku 1) přidávat výrokové proměnné a klauzule pouze
pro buňky $\func(v_1, v_2)$, kde $v_1 \le v_2$.
Každou výrokovou proměnnou $A_{\func(v_1, v_2) = v}$, kde $v_1 > v_2$,
nahradíme proměnnou $A_{\func(v_2, v_1) = v}$.

Program \crossbow{} mezi vstupními klauzulemi a mezi všemi
klauzulemi z~dokazovače E hledá klauzuli ekvivalentní
s klauzulí $\func(\var, \varY) \approx \func(\varY, \var)$.
Pokud takovou klauzuli najde, označí funkci $\func^\interp$
jako komutativní a při vytváření instance SATu se použije
optimalizace uvedená v předchozím odstavci.

Podobně bychom mohli postupovat i u symetrických relací,
to však program \crossbow{} nedělá.

\subsection{Převod pro Gecode}

{
\def\Eq{\textproc{Eq}}
\def\LowerEq{\textproc{Lower-Eq}}
\def\Clause{\textproc{Clause}}
\def\Element{\textproc{Element}}
\def\Linear{\textproc{Linear}}
\def\Precede{\textproc{Precede}}

Chceme-li se vyhnout zplošťování klauzulí a s ním spojenému nárůstu počtu
proměnných, můžeme klauzule kódovat do bohatšího jazyka, než je SAT.
Program \crossbow{} umí klauzule kódovat do jazyka
omezujících podmínek řešiče Gecode \cite{gecode}.
Při kódování se používají booleovské proměnné, celočíselné proměnné,
pole booleovských i celočíselných proměnných, celočíselné konstanty
a jejich pole. Pole jsou indexovány od 0.
Booleovské proměnné značíme písmenem
$b$, celočíselné proměnné nebo konstanty značíme písmenem $i$,
pole booleovských proměnných značíme písmenem $B$
a pole celočíselných proměnných nebo konstant značíme písmenem $I$.
Z~podmínek se používají:
\begin{itemize}
\item \Eq$(i_1, i_2, b)$. Podmínka je splněna právě tehdy,
  když $i_1 = i_2 \wedge b = 1$ nebo $i_1 \neq i_2 \wedge b = 0$.
\item \LowerEq$(i_1, i_2)$. Podmínka je splněna právě tehdy,
  když $i_1 \le i_2$.
\item \Clause$(B_P, B_N)$. Podmínka je splněna právě tehdy,
  když je nějaká booleovská proměnná z $B_P$ ohodnocena 1
  nebo je nějaká booleovská proměnná z~$B_N$ ohodnocena 0.
\item \Element$(H, i, h)$, kde $H$ resp. $h$ je buď pole booleovských
  proměnných resp. booleovská proměnná, nebo pole celočíselných proměnných
  resp. celo\-číselná proměnná. $i$ je vždy celočíselná proměnná.
  Podmínka je splněna právě tehdy,
  když hodnota proměnné v poli $H$ s indexem $i$ je stejná jako
  hodnota proměnné $h$.
\item \Linear$(I_c, I_x, i)$, kde $I_c$ a $I_x$ jsou pole stejné délky,
  $I_c$ je pole konstant, $I_x$ je pole proměnných a $i$ je konstanta.
  Podmínka je splněna právě tehdy, když
  \[
    \sum^{|I_c|-1}_{k=0} I_c(k) \cdot I_x(k) = i.
  \]
\item \Precede$(I_x, I_c)$, kde $I_x$ je pole proměnných a
  $I_c$ je pole konstant. Podmínka je splněna právě tehdy, když
  pro každé $j \in \{ 0 \comdots |I_x| - 1 \}$ a pro každé
  $k \in \{ 1 \comdots |I_c| - 1 \}$ platí
  \[
    I_x(j) = I_c(k) \implies
      \exists j' \in \{ 0 \comdots j - 1 \}: I_x(j') = I_c(k - 1).
  \]
  Jinak řečeno, kdykoliv má proměnná z $I_x$ hodnotu $i$
  z $I_c(1 \comdots \star)$, tak v $I_x$ existuje proměnná
  s nižším indexem a s hodnotou, jenž je v poli $I_c$ těsně před $i$.
\end{itemize}

Začneme tím, že ukážeme, jak do uvedených podmínek zakódovat
ploché klauzule. Následně výklad rozšíříme i o neploché klauzule
a o LNH.

\subsubsection{Ploché klauzule}

Pro zakódování plochých klauzulí stačí podmínky \Eq{} a \Clause{}.
Napřed pro každou funkci $\func^\interp$ a každou její buňku
$c \in \cells_\func$ přidáme celočíselnou proměnnou $i_c$
s doménou $D_\sort$, kde $\sort$ je sorta výsledku $\func$.
Dále pro každou relaci $\pred^\interp$ a každou její buňku
$c \in \cells_\pred$ přidáme booleovskou proměnnou $b_c$.
Při přidávání proměnných pro buňky uplatňuje program \crossbow{}
trik s komutativními funkcemi.

Nyní přidáme podmínky pro klauzule. Mějme klauzuli
$\clause \in \clauses$ a ohodnocení proměnných $\beta$.
Pokud $\clause$ obsahuje literál $\lit$ s atomem $\var \approx \varY$
takový, že $\lit$ je splněný při ohodnocení $\beta$, tak
pro klauzuli $\clause$ a ohodnocení $\beta$ žádné podmínky nepřidáváme.
Dále předpokládejme, že klauzule žádný takový literál neobsahuje.
Označme $\lit_1 \comdots \lit_j$ všechny literály z klauzule,
jenž neobsahují atom tvaru $\var \approx \varY$.
Každému literálu z $\lit_k$ z $\lit_1 \comdots \lit_j$
přiřadíme booleovskou proměnnou $b_k$:
\begin{itemize}
\item Obsahuje-li $\lit_k$ atom tvaru
  $\func(\var_1 \comdots \var_n) \approx \var$ (resp.
  $\var \approx \func(\var_1 \comdots \var_n)$),
  tak $b_k$ bude nová booleovská proměnná
  a přidáme podmínku
  \[
    \Eq(i_{\func(\beta(\var_1) \comdots \beta(\var_n))}, \beta(\var), b_k).
  \]
\item Obsahuje-li $\lit_k$ atom tvaru
  $\pred(\var_1 \comdots \var_n)$, tak $b_k$ bude existující proměnná
  \[
    b_{\pred(\beta(\var_1) \comdots \beta(\var_n))}
  \]
  a žádnou podmínku nebudeme přidávat.
\end{itemize}
Následně přidáme podmínku \Clause$(B_P, B_N)$, kde pole
proměnných $B_P$ obsahuje právě ty proměnné
$b_k$, jenž jsou přiřazeny literálům bez negace.
Naopak pole proměnných $B_N$ obsahuje právě ty proměnné $b_k$,
jenž jsou přiřazeny literálům s~negací.

Abychom dostali podmínky pro všechny klauzule,
použijeme uvedený postup na každou klauzuli $\clause \in \clauses$
se všemi ohodnoceními $\beta$, jenž se liší na proměnných
v klauzuli $\clause$.

Ukažme si, jak získat podmínky pro klauzuli
$\pred(\var) \vee \func(\var) \not\approx \varY \vee \var \approx \varY$
s~jedinou doménou velikosti 2. Uvedený postup musíme
aplikovat na 4 ohodnocení $\beta_1, \beta_2, \beta_3$ a $\beta_4$,
která splňují:
\begin{align*}
\beta_1(\var) &= 0, & \beta_1(\varY) &= 0, \\
\beta_2(\var) &= 0, & \beta_2(\varY) &= 1, \\
\beta_3(\var) &= 1, & \beta_3(\varY) &= 0, \\
\beta_4(\var) &= 1, & \beta_4(\varY) &= 1.
\end{align*}

Jelikož je literál $\var \approx \varY$ splněný při ohodnoceních
$\beta_1$ a $\beta_4$, nebudeme pro tato ohodnocení žádné podmínky přidávat.
Pro obě zbylá ohodnocení označíme literály, jenž neobsahují
atom rovnosti:
\begin{align*}
\lit_1 &= \pred(\var), \\
\lit_2 &= \func(\var) \not\approx \varY.
\end{align*}
Pro ohodnocení $\beta_2$ přiřadíme literálu $\lit_1$
existující booleovskou proměnnou $b_{\pred(0)}$, literálu $\lit_2$
přiřadíme novou booleovskou proměnnou $b_2^{\beta_2}$
a přidáme podmínku \Eq$(i_{\func(0)}, 1, b_2^{\beta_2})$.
Nakonec pro ohodnocení $\beta_2$ přidáme podmínku
\[
  \Clause([b_{\pred(0)}], [b_2^{\beta_2}]).
\]
Pro ohodnocení $\beta_3$ přiřadíme literálu $\lit_1$
existující booleovskou proměnnou $b_{\pred(1)}$, literálu $\lit_2$
přiřadíme novou booleovskou proměnnou $b_2^{\beta_3}$
a přidáme podmínku \Eq$(i_{\func(1)}, 0, b_2^{\beta_3})$.
Nakonec pro ohodnocení $\beta_3$ přidáme podmínku
\[
  \Clause([b_{\pred(1)}], [b_2^{\beta_3}]).
\]
Pro klauzuli
$\pred(\var) \vee \func(\var) \not\approx \varY \vee \var \approx \varY$
jsme dohromady přidali 4 podmínky.

Pro literály s atomem tvaru
$\func(\var_1 \comdots \var_n) \approx \var$ (resp.
$\var \approx \func(\var_1 \comdots \var_n)$), není třeba
v některých situacích přidávat novou booleovskou proměnnou
a podmínku. Uvažme například klauzuli
$\func(\var) \not\approx \var \vee \varY \approx \funcC$
s jedinou doménou velikosti 2.

Označme literály klauzule
\begin{align*}
\lit_1 &= \func(\var) \not\approx \var, \\
\lit_2 &= \varY \approx \funcC.
\end{align*}
Následující tabulka ukazuje proměnné a podmínky,
jenž je třeba přidat pro oba literály pro ohodnocení
$\beta_1, \beta_2, \beta_3$ a $\beta_4$:

\begin{center}
{
\renewcommand{\arraystretch}{1.2}
\begin{tabular}{c|c|c}
& $\lit_1$ & $\lit_2$ \\
\hline
$\beta_1$ &
  \Eq$(i_{\func(0)}, 0, b_1^{\beta_1})$ &
  \Eq$(i_{\funcC}, 0, b_2^{\beta_1})$ \\
$\beta_2$ &
  \Eq$(i_{\func(0)}, 0, b_1^{\beta_2})$ &
  \Eq$(i_{\funcC}, 1, b_2^{\beta_2})$ \\
$\beta_3$ &
  \Eq$(i_{\func(1)}, 1, b_1^{\beta_3})$ &
  \Eq$(i_{\funcC}, 0, b_2^{\beta_3})$ \\
$\beta_4$ &
  \Eq$(i_{\func(1)}, 1, b_1^{\beta_4})$ &
  \Eq$(i_{\funcC}, 1, b_2^{\beta_4})$ \\
\end{tabular}
}
\end{center}

Z tabulky je patrné, že pro literál $\lit_1$, pro ohodnocení $\beta_2$
můžeme použít stejnou proměnnou a podmínku jako pro ohodnocení $\beta_1$.
Podobně pro ohodnocení $\beta_3$ můžeme použít stejnou
proměnnou a podmínku jako pro ohodnocení $\beta_4$.
Pro literál $\lit_2$ můžeme rovněž ušetřit dvě booleovské
proměnné a dvě podmínky.

\subsubsection{Obecné klauzule}

\def\Horner{\textproc{Horner}}

Pro každý funkční symbol $\func$ arity
$\langle \sort_1 \comdots \sort_n, \sort \rangle$ přidáme
pole $I_\func$ celočíselných proměnných. Délka
pole bude $\prod_{j=1}^n n_{\sort_j}$. Pole bude obsahovat
proměnné pro buňky funkce $\func^\interp$.
Proměnná pro buňku $\func(v_1 \comdots v_n) \in \cells_\func$
bude v poli $I_\func$ pod indexem
\Horner$(v_1 \comdots v_n, n_{\sort_1} \comdots n_{\sort_n})$.
Funkce \Horner{} je definována následovně:
\medskip
\begin{algorithmic}
\Function{Horner}{$v_1 \comdots v_n, n_{\sort_1} \comdots n_{\sort_n}$}
  \State $j \gets 0$
  \For{$k = 1 \comdots n$}
    \State $j \gets j \cdot n_{\sort_k} + v_k$
  \EndFor
  \State \Return $j$
\EndFunction
\end{algorithmic}
\medskip

\noindent
Pro každý predikátový symbol $\pred$ arity
$\langle \sort_1 \comdots \sort_n \rangle$
přidáme pole $B_\pred$ booleovských proměnných.
Délka pole bude $\prod_{j=1}^n n_{\sort_j}$. Pole bude obsahovat
booleovské proměnné pro buňky $\pred^\interp$.
Proměnná pro buňku $\pred(v_1 \comdots v_n) \in \cells_\pred$
bude v poli $B_\pred$ pod indexem
\Horner$(v_1 \comdots v_n, n_{\sort_1} \comdots n_{\sort_n})$.

Díky právě přidaným polím můžeme zakódovat neplochou klauzuli
$\func(\funcC) \approx \funcC$. Zavedeme pomocnou
proměnnou $i$ a přidáme podmínku \Element$(I_\func, \funcC, i)$,
která zajistí, že proměnná $i$ obsahuje hodnotu funkce $\func^\interp$
v bodě $\funcC^\interp$. Jelikož se tato hodnota musí rovnat hodnotě
$\funcC^\interp$, přidáme stejně jako v případě plochých klauzulí
pomocnou booleovskou proměnnou $b$ a podmínku \Eq$(i, i_\funcC, b)$.
Nakonec přidáme podmínku \Clause$([b], [])$.

Důvodem, proč pro zakódování klauzule potřebujeme pole $I_\func$,
je, že nevíme, jakou buňku $\func(\funcC)$ přesně označuje
(v případě plochých klauzulí jsme to věděli), a tudíž
do podmínky \Eq{} nemůžeme dát proměnnou přímo pro tuto buňku.
Víme pouze to, že proměnná pro buňku $\func(\funcC)$
je v poli $I_\func$ a má index $\funcC^\interp$ -- k získání
její hodnoty použijeme podmínku \Element.

\def\Index{\textproc{Index}}
\def\TermValue{\textproc{Term-Value}}
\def\VarForAtom{\textproc{Var-For-Atom}}

V obecném případě budeme potřebovat zjistit, jakou buňku
označuje term $\func(\term_1 \comdots \term_n)$ resp.
atom $\pred(\term_1 \comdots \term_n)$ při ohodnocení
proměnných $\beta$. Před\-po\-klá\-dejme,
že máme k dispozici funkci \Index{} takovou,
že volání \Index$(\beta, \term_1 \comdots \term_n)$
vrátí index proměnné pro buňku $\func(\term_1 \comdots \term_n)$ resp.
$\pred(\term_1 \comdots \term_n)$ v poli $I_\func$ resp. $B_\pred$
při ohodnocení $\beta$. Index je buď celočíselná proměnná, nebo
konstanta. S pomocí funkce \Index{}
vytvoříme funkce:
\begin{itemize}
\item \TermValue$(\beta, \term)$.
  Vrací celočíselnou proměnnou nebo konstantu
  s hodnotou termu $\term$ při ohodnocení proměnných $\beta$.
\item \VarForAtom$(\beta, \atom)$.
  Vrací booleovskou proměnnou, jenž obsahuje
  hodnotu atomu $\atom$ při ohodnocení proměnných $\beta$.
\end{itemize}

Je-li $\term$ term tvaru $\var$,
funkce \TermValue$(\beta, \term)$ vrátí konstantu
$\beta(\var)$.
Je-li $\term$ term tvaru $\func(\term_1 \comdots \term_n)$,
funkce \TermValue$(\beta, \term)$
zavolá \Index$(\beta, \term_1 \comdots \term_n)$,
čímž dostane index do pole $I_\func$.
Pokud je index konstanta, vrátí funkce proměnnou
z pole $I_\func$ s daným indexem. V opačném případě je
index proměnná $i$. Pak funkce \TermValue{}
zavede novou pomocnou proměnnou $i'$, přidá
podmínku \Element$(I_\func, i, i')$ a vrátí proměnnou $i'$.

Je-li $\atom$ atom tvaru $\term \approx \termS$,
\VarForAtom$(\beta, \atom)$
zavolá \TermValue$(\beta, \term)$
a \TermValue$(\beta, \termS)$, čímž dostane
proměnné nebo konstanty $i$ a $i'$ s hodnotami $\term$ a $\termS$.
Dále \VarForAtom{} zavede novou pomocnou proměnnou $b$,
přidá podmínku \Eq$(i, i', b)$ a vrátí proměnnou $b$.
Je-li $\atom$ atom tvaru $\pred(\term_1 \comdots \term_n)$,
postupuje funkce \VarForAtom$(\beta, \atom)$
analogicky jako funkce \TermValue{} v případě,
že term není proměnná.

Funkce \Index$(\beta, \term_1 \comdots \term_n)$
pro každý term $t_j$, kde $j \in \{ 1 \comdots n \}$ zavolá funkci
\TermValue$(\beta, t_j)$, čímž dostane proměnné
nebo konstanty $i_1 \comdots i_n$.
Mají-li termy $\term_1 \comdots \term_n$ sorty $\sort_1 \comdots \sort_n$,
pak \Index{} musí vrátit hodnotu
\Horner$(i_1 \comdots i_n, n_{\sort_1} \comdots n_{\sort_n})$.
Pokud jsou všechny $i_1 \comdots i_n$ konstanty,
tak vrácená hodnota bude rovněž konstanta. Pokud je mezi
$i_1 \comdots i_n$ proměnná, pak funkce \Index{}
zavede pomocnou proměnnou $i$, pomocí podmínky \Linear{} zajistí,
že $i$ má hodnotu
\Horner$(i_1 \comdots i_n, n_{\sort_1} \comdots n_{\sort_n})$,
a vrátí proměnnou $i$.

Při konstrukci podmínek \Clause{} se literálům přiřadí
booleovské proměnné vrácené funkcí \VarForAtom.

Převod neploché klauzule do podmínek řešiče Gecode si ukážeme na
klauzuli $\func(\funcG(\var), \varY, \funcG(\var)) \approx \varY$,
kde proměnné $\var, \varY$ mají sortu s doménou velikosti 2 a
sorta výsledku $\funcG$ má doménu velikosti 3.

Začneme s podmínkami pro ohodnocení $\beta_1$. Abychom
přiřadili booleovskou proměnnou jedinému literálu
z klauzule, zavoláme funkci \VarForAtom$(\beta_1, \allowbreak
\func(\funcG(\var), \varY, \funcG(\var)) \approx \varY)$.
Tato funkce následně zavolá
\begin{align*}
&\TermValue(\beta_1, \func(\funcG(\var), \varY, \funcG(\var))) \quad\text{a} \\
&\TermValue(\beta_1, \varY).
\end{align*}
Druhé volání vrátí hodnotu proměnné $\varY$ při ohodnocení $\beta_1$,
což je konstanta 0. První volání napřed spočte hodnoty
termů $\var, \funcG(\var), \varY$. Hodnotou termů $\var$ a $\varY$
je konstanta 0. Hodnota termu $\funcG(\var)$ je nultá proměnná
z pole $I_\funcG$, tj. $i_{\funcG(0)}$.
Na základě těchto hodnot musí funkce \Index{} spočítat
index proměnné v poli $I_\func$, jenž obsahuje hodnotu termu
$\func(\funcG(\var), \varY, \funcG(\var))$. Víme,
že hodnota indexu je dána
\Horner$(i_{\funcG(0)}, 0, i_{\funcG(0)}, 3, 2, 3)$.
Zavedeme-li pro index novou pomocnou proměnnou $i_1$, pak musí platit
\[
  (i_{\funcG(0)} \cdot 2 + 0) \cdot 3 + i_{\funcG(0)} = i_1.
\]
Úpravou dostaneme rovnici
\[
  7 \cdot i_{\funcG(0)} + (-1) \cdot i_1 = 0,
\]
jenž zakódujeme pomocí podmínky \Linear$([i_{\funcG(0)}, i_1], [7, -1], 0)$.
Funkce \Index{} vrátí proměnnou $i_1$. Funkce \TermValue{}
zavede novou pomocnou proměnnou $i'_1$ a pomocí podmínky
\Element$(I_\func, i_1, i'_1)$ zajistí, že $i'_1$ bude obsahovat hodnotu termu
$\func(\funcG(\var), \varY, \funcG(\var))$. Funkce
\TermValue$(\beta_1, \func(\funcG(\var), \varY, \funcG(\var)))$
vrátí proměnou $i'_1$.

Nyní jsme zpět ve funkci \VarForAtom{} a známe hodnoty
termů na levé a pravé straně. Jelikož se tyto hodnoty
mají rovnat, přidá \VarForAtom{} novou booleovskou proměnnou $b_1$,
podmínku \Eq$(i'_1, 0, b_1)$ a vrátí $b_1$.
Nakonec je přidána podmínka \Clause$([b_1], [])$.
Stejně lze postupovat pro ohodnocení $\beta_2, \beta_3$ a $\beta_4$.

Pomocné proměnné zavedené pro podmínky \Linear{}, \Element{} a \Eq{}
lze používat opakovaně (pro podmínky \Eq{} viz konec sekce
Ploché klauzule).

\subsubsection{LNH}

Redukci symetrií LNH implementujeme prostřednictvím podmínek
\LowerEq{} a \Precede{}. Popíšeme LNH pro funkční symboly
se sortou výsledku $\sort$. Před\-po\-klá\-dáme,
že všechny hodnoty z $D_\sort$ jsou nepoužité.

Buňky funkčních symbolů se sortou výsledku $\sort$
uspořádáme do posloupnosti.
Buňky, jenž nemají žádný argument sorty $\sort$, budou
před buňkami, jenž mají alespoň jeden
argument sorty $\sort$. Buňky s argumenty sorty $\sort$
budou uspořádány tak, že je-li buňka $c$
před buňkou $c'$, tak maximální argument z $D_\sort$ buňky $c$
není větší než maximální argument z $D_\sort$ buňky $c'$.

Pro buňky funkčních symbolů $c_1 \comdots c_n$,
jenž nemají argument z $D_\sort$,
se přidá podmínka
\[
\begin{split}
  \Precede([&i_{c_1} \comdots i_{c_n}], \\
           [&0 \comdots \min \{ n, n_\sort \} - 1]).
\end{split}
\]
Pro buňky $c'_1 \comdots c'_{n'}$, kde maximální argument z $D_\sort$
je 0, se přidá podmínka
\[
\begin{split}
  \Precede([&i_{c_1} \comdots i_{c_n}, \\
            &i_{c'_1} \comdots i_{c'_{n'}}], \\
           [&1 \comdots \min \{ m + n + n', n_\sort \} - 1]),
\end{split}
\]
kde $m = 0$, pokud $n > 0$, jinak $m = 1$.
Pro buňky $c''_1 \comdots c''_{n''}$ s maximálním argumentem
1 se přidá podmínka
\[
\begin{split}
  \Precede([&i_{c_1} \comdots i_{c_n},\\
            &i_{c'_1} \comdots i_{c'_{n'}},\\
            &i_{c''_1} \comdots i_{c''_{n''}}], \\
           [&2 \comdots \min \{ m + n + n' + n'', n_\sort \} - 1]).
\end{split}
\]
Pro buňky, kde je maximální argument z $D_\sort$ vyšší,
se postupuje analogicky. Podmínka \LowerEq{} se používá
pro omezování velikosti domén.

}

\subsection{Redundantní podmínky pro Gecode}

Abychom posílili propagaci řešiče Gecode, přidáme redundantní
globální pod\-mín\-ky.
Mezi vstupními klauzulemi a klauzulemi z dokazovače E
budeme hledat axiomy grup, kvazigrup a involutorních funkcí.
Tabulky násobení v grupách a kvazigrupách jsou latinské čtverce,
pro každý řádek a sloupec takové tabulky přidáme jednu podmínku
\textproc{All-Different}, jenž vynutí, že daný řádek nebo
sloupec obsahuje každou hodnotu právě jednou.
Tabulky involutorních funkcí obsahují každou hodnotu právě jednou,
pro každou involutorní funkci přidáme jednu podmínku
\textproc{All-Different}.

\subsection{Hledání všech neizomorfních modelů}

V této sekci popíšeme, jak najít více modelů a jak odfiltrovat
izomorfní modely. Řešič Gecode prozkoumává ohodnocení
proměnných systematicky a snadno tak najde všechna ohodnocení splňující
zadané podmínky. U SAT řešičů je situace obvykle odlišná,
proto po nalezení každého modelu přidáme klauzuli,
která zakáže ohodnocení výrokových proměnných, jenž vedou na daný model.
Program \crossbow{} přidává klauzuli, jenž pro každou
funkci z modelu a každou její buňku $c$ s hodnotou $v$
obsahuje literál $\neg A_{c=v}$ a pro každou relaci z modelu a každou její
buňku s hodnotou $v$ obsahuje buňku $A_{c=1}$, pokud $v = 0$, nebo
$\neg A_{c=1}$, pokud $v = 1$.

Nyní dokážeme postupně generovat různé modely.
LNH v obecném případě však nezabrání situaci,
že vygenerovaný model je izomorfní jinému modelu, který byl
vygenerován dříve.
Abychom tyto izomorfní modely odstranili,
mohli bychom pro každý nalezený model
zkusit najít izomorfismus mezi právě nalezeným modelem
a dříve nalezenými modely.
Nevýhoda tohoto řešení spočívá v tom,
že s~rostoucím počtem nalezených modelů bude růst
i čas potřebný na nalezení izomorfismu.

Jiné řešení je použít zobrazení kanonických reprezentantů pro číselné
interpretace. Pomocí něj každý číselný model převedeme
na kanonický číselný model. Díky tomu, že kanonické číselné modely
jsou izomorfní právě tehdy, když se rovnají, můžeme velmi rychle
detekovat, zda je nově nalezený model izomorfní již
dříve nalezenému modelu.

K implementaci zobrazení kanonických reprezentantů
pro číselné interpretace využijeme implementaci
zobrazení kanonických reprezentantů pro orientované barevné grafy
(dále jen grafy). Číselnou interpretaci převedeme na graf,
ke grafu najdeme kanonický graf a na základě kanonického grafu
zkonstruujeme kanonickou číselnou interpretaci.
K nalezení kanonického grafu využívá program
\crossbow{} knihovnu bliss \cite{bliss}.

Konstrukce grafu z interpretace:
\begin{itemize}
\item Pro každou doménu $D_\sort$ vezmeme dosud nepoužitou barvu $i_{D_\sort}$
  a pro každý prvek domény $v \in D_\sort$ přidáme vrchol $u_v$ barvy
  $i_{D_\sort}$.
\item Pro každou relaci $\pred^\interp$ arity
  $\langle \sort_1 \comdots \sort_n \rangle$, kde $n > 0$
  vezmeme $n$ dosud nepoužitých barev, označme je
  $i_1 \comdots i_n$.
  Pro každou buňku $\pred(v_1 \comdots v_n)$ relace $\pred^\interp$
  s~hodnotou 1 přidáme nové vrcholy $u_1^\pred \comdots u_n^\pred$,
  kde vrchol $u_k^\pred$ má barvu $i_k$ pro $k \in \{ 1 \comdots n \}$.
  Nově přidané vrcholy propojíme hranami $(u_{k-1}^\pred, u_k^\pred)$ pro
  $k \in \{ 2 \comdots n \}$. Dále propojíme hranami
  vrcholy pro argumenty s jejich hodnotami, tj. pro
  každé $k \in \{ 1 \comdots n \}$ přidáme hranu
  $(u_k^\pred, u_{v_k})$, kde $u_{v_k}$ je vrchol pro hodnotu
  $v_k$ z domény $D_{\sort_k}$.
\item Pro každou funkci $\func^\interp$ arity
  $\langle \sort_1 \comdots \sort_n, \sort_0 \rangle$
  vezmeme $n + 1$ dosud ne\-pou\-ži\-tých barev, označme je
  $i_0 \comdots i_n$.
  Pro každou buňku $\func(v_1 \comdots v_n)$ funkce $\func^\interp$
  s hodnotou $v_0$ přidáme nové vrcholy $u_0^\func \comdots u_n^\func$,
  kde vrchol $u_k^\func$ má barvu $i_k$ pro $k \in \{ 0 \comdots n \}$.
  Nově přidané vrcholy propojíme hranami $(u_{k-1}^\func, u_k^\func)$ pro
  $k \in \{ 1 \comdots n \}$. Dále propojíme hranami
  vrcholy pro argumenty a vrchol pro hodnotu buňky s jejich hodnotami, tj. pro
  každé $k \in \{ 0 \comdots n \}$ přidáme hranu
  $(u_k^\func, u_{v_k})$, kde $u_{v_k}$ je vrchol pro hodnotu
  $v_k$ z domény $D_{\sort_k}$.
\end{itemize}
Jsou-li dva modely izomorfní, pak\footnote{Nejedná se o ekvivalenci.
Může nastat situace, že grafy budou izomorfní i pro dva neizomorfní modely.}
grafy zkonstruované pro tyto modely výše popsaným způsobem budou
rovněž izomorfní.

% Chceme vlastně normalizovat pořadí prvků v každé doméně.
% Usporádání prvků v doméně provádíme na základě vztahu
% prvků k buňkám.
%
% Příkladem dvou neizomorfních modelů s izomorfními grafy
% jsou modely P=0 a P=1, kde P je nulární predikát.
% Grafy těchto modelů budou obsahovat pouze vrcholy pro prvky
% domén, žádné hrany.

Společně s kanonickým grafem bliss vydá i bijekci vrcholů,
jenž dokazuje, že grafy jsou skutečně izomorfní.
Restrikce této bijekce na vrcholy pro prvky domén
určuje bijekci na prvcích domén.
S pomocí této bijekce můžeme přejmenovat prvky
domén, čímž dostaneme kanonický model.

\subsection{SAT řešič s podporou dalších podmínek}

Další možností, jak snížit počet proměnných nebo počet klauzulí
nebo zlepšit propagaci, je rozšířit SAT řešič o další typy podmínek.
Do SAT řešiče MiniSat 2.2 \cite{minisat} jsme přidali speciální
typ výrokových klauzulí, které jsou splněny právě tehdy,
když obsahují právě jeden splněný literál.
Nový řešič se jmenuje Josat.
Díky speciálním klauzulím není třeba přidávat klauzule
z bodu 1b) sekce \ref{sec:mace-basic}.

Aktuální implementace nedovoluje uvnitř speciálních klauzulí užít negaci
a každá výroková proměnná se může vyskytovat nejvýše v jedné
speciální klauzuli.

\section{Výběr programovacího jazyka}

\section{Architektura programu}

\chapter{Experimenty}

V této kapitole srovnáme program \crossbow{} s jinými programy
pro hledání modelů. Srovnání budeme provádět na splnitelných problémech
s \texttt{cnf} formulemi z~kolekce TPTP 6.1.0 \cite{sutcliffe2009tptp}.

Programy budeme spouštět na počítači s procesorem
Intel Core i5-3570, ope\-račním systémem openSUSE 13.1 a
verzí Linuxu 4.1.2. Na vyřešení jednoho problému bude mít
každý program k dispozici 11 GiB RAM a 2 minuty procesorového
času\footnote{Procesorový čas na jednotlivých jádrech se sčítá. To
znamená, že program může vyčerpat časový limit například i za jednu minutu,
pokud bude po dobu jedné minuty používat dvě jádra procesoru.}.

Program \crossbow{} budeme srovnávat s následujícími programy:
\begin{itemize}
\item Mace4\footnote{Program Mace4 neumí hledat
  konečné modely s velikostí domény 1.} \cite{mccune03mace4},
\item Paradox \cite{paradox} (program byl upraven,
  aby ho bylo možné přeložit GHC 7.6.3),
\item iProver 1.0 \cite{iprover}.
\end{itemize}
Ke kompilaci programů použijeme kompilátory GCC 4.8.1,
OCaml 4.02.2 a GHC 7.6.3 s knihovnami z Haskell Platform 2013.2.

Pro překlad programu \crossbow{} použijeme následující programy
a knihovny (dostupné z repozitáře OPAM \cite{opam}):
\begin{itemize}
\item batteries 2.3.1,
\item cmdliner 0.9.7,
\item menhir 20141215,
\item ocamlfind 1.5.5,
\item oclock 0.4.0,
\item omake 0.9.8.6-0.rc1,
\item ounit 2.0.0,
\item pprint 20140424,
\item ppx\_tools 0.99.2,
\item sexplib 112.24.01,
\item sqlite 3.2.0.9,
\item tptp 0.3.1,
\item zarith 1.3.
\end{itemize}

Bohužel v našem srovnání nemáme hledače modelů
založené na metodách Model Evolution
a SGGS \cite{bonacina2015}.
Implementace E-Darwin 1.4 \cite{edarwin} a MELIA 0.1.3 \cite{melia}
metody Model Evolution se nám totiž nepodařilo zprovoznit
a o žádné implementaci SGGS nevíme.

\bigskip
Nyní budeme prezentovat výsledky našeho srovnání.
Začneme tabulkou, jenž obsahuje počty vyřešených problémů.
V prvním sloupci tabulky jsou názvy kategorií problémů
z kolekce TPTP. Kategorie, kde bylo ostře méně než 10
splnitelných problémů, byly sloučeny pod položku \uv{Ostatní}.
První řádek tabulky obsahuje názvy konfigurací\footnote{Soubor
\texttt{mereni/run\_all\_provers} na přiloženém DVD obsahuje
přesné parametry, s nimiž byly hledače modelů spouštěny.}
hledačů modelů. Pro cizí programy jsou názvy konfigurací:
\begin{itemize}
\item Mace. Program Mace4.
\item Paradox. Program Paradox.
\item iProver. Program iProver, hledání modelů.
\item iProver/Fin. Program iProver, hledání konečných modelů.
\end{itemize}
Ostatní názvy konfigurací značí program \crossbow{}.
Použitý řešič se pozná z prefixu názvu konfigurace (CMSat značí
CryptoMiniSat). Pokud název konfigurace končí řetězcem \uv{+E},
znamená to, že byl po dobu 5 sekund spuštěn dokazovač E za účelem
generování redundantních klauzulí. V opačném případě nebyl dokazovač E
spuštěn vůbec.

\newcommand\summary[1]{\noindent
\begin{minipage}{1.0\textwidth}
\begin{center}
\include{#1}
\end{center}
\end{minipage}}

\newpage

Tabulka s počty vyřešených problémů:

\summary{sum_counts}

Jak je vidět, nejlépe si vedl program \crossbow{} s řešičem MiniSat
a se zapnutým generováním redundantních klauzulí.
Všimněme si, že program \crossbow{} vý\-raz\-něji zaostává
pouze na problémech z kategorií NLP a SYN.
Důvodem je velikost explicitně reprezentovaných modelů. Modely z těchto
kategorií mají při vypsání programem \crossbow{} do formátu TPTP
velikost i několik gigabajtů. Například Josat našel
v kategorii NLP více modelů než iProver, jenže
modely problémů 75, 82-93 nestihl celé vypsat.
Tyto problémy také ukazují výhody řešiče Josat --
řešiči MiniSat došla paměť, zatímco řešič Josat dokázal modely najít.
Domníváme se, že výhody řešiče Josat by se projevily ještě více,
kdyby modely problémů měly větší domény.

Zklamáním je naopak řešič Gecode. Je pomalejší a spotřebovává více
paměti než SAT řešiče. Důvodem jsou pravděpodobně použité podmínky
--  SAT řešiče pracují s klauzulemi efektivněji než Gecode.

\newpage

Nyní se podíváme na graf ukazující vývoj počtu vyřešených problémů
v čase:

\summary{sum_plot}

Z grafu je patrné, že iProver po jedné minutě zkouší jinou metodu.
Od osmdesáté sekundy se počet vyřešených problémů mění pouze velmi
mírně -- nezdá se, že by zvýšení časového limitu dramaticky změnilo
výsledky.

\newpage

Gecode podporuje jak ploché, tak i neploché klauzule.
Následující tabulka ukazuje, že kvůli kategorii LAT\footnote{Důvodem
jsou klauzule, jejichž zploštění způsobí výrazný nárůst počtu proměnných.}
je lepší nezplošťovat:

\summary{sum_countsF}

Gecode+E označuje konfiguraci bez zplošťování,
Gecode-F+E označuje konfiguraci se zplošťováním.

\newpage

Některé modifikace lze v programu \crossbow{} deaktivovat.
Díky tomu můžeme zkoumat, jaký přínos tyto modifikace mají.
Zde jsou výsledky pro definice termů:

\summary{sum_countsNDef}

Řetězec NDef v názvu konfigurace značí, že
definice termů nebyly použity. Pro řešič MiniSat
je největší přínos v kategorii LAT, důvodem
jsou opět klauzule, jejichž zploštění způsobí
výrazný nárůst počtu proměnných.
Díky opakovanému používání pomocných proměnných zavedených
pro podmínky \Linear{}, \Element{}, \Eq{}
a faktu, že Gecode+E nezplošťuje, nemá řešič Gecode
s kategorií LAT obtíže, i když jsou definice termů vypnuté.

\newpage

Ukazuje se, že detekce axiomů komutativity,
grup, kvazigrup a involutorních funkcí nemá prakticky
žádný přínos:

\summary{sum_countsNDet}

Řetězec NDet v názvu konfigurace značí, že
detekce axiomů nebyla použita.

\newpage

Odzplošťování nemá na problémy z kolekce TPTP vůbec žádný efekt:

\summary{sum_countsNU}

Řetězec NU v názvu konfigurace značí, že
odzplošťování nebylo použito.

\newpage

Na rozdíl od předešlých dvou modifikací je přínos
rozdělování klauzulí značný:

\summary{sum_countsNS}

Řetězec NS v názvu konfigurace značí, že
rozdělování klauzulí nebylo použito.
Jelikož řešiči Gecode dáváme neploché klauzule,
není přínos rozdělování klauzulí tak výrazný.

% Zplošťování přidává nové proměnné -- bez zplošťování
% se rozdělování klauzulí nemůže tolik projevit.

\newpage

Nakonec se podíváme, jaký přínos má inference sort:

\summary{sum_countsNSI}

Řetězec NSI v názvu konfigurace značí, že
inference sort nebyla použita.
Inference sort má výrazný přínos
pouze v kategorii NLP pro řešič Gecode.
Bohužel neznáme přesné důvody\footnote{Jelikož
problémy z kategorie NLP obsahují značné množství
funkčních symbolů, je možným důvodem
velikost podmínek přidaných LNH pro Gecode.
S větším množstvím sort (u některých problémů z kategorie NLP
lze inferovat stovky sort)
klesá velikost těchto podmínek.},
proč tomu tak je.

% Alternativně: Může být přínos způsoben blokováním LNH?
% Tj. LNH se u některých sort vůbec neprovádí
% kvůli předpokladu, že všechny hodnoty z $D_\sort$ jsou nepoužité.

\chapter{Závěr}

V této práci jsme vytvořili hledač konečných modelů \crossbow{},
který dokáže hledat jeden konečný model nebo všechny navzájem
neizomorfní konečné modely dané velikosti.
Implementovali jsme metodu MACE, některé její známé modifikace
a také několik nových modifikací, jenž, pokud je nám
známo, dosud nikdo nezkoušel.
Domníváme se tedy, že zadání diplomové práce bylo splněno.

Ve srovnání s ostatními programy pro hledání modelů si program \crossbow{}
nevede špatně. Experimenty z minulé kapitoly však naznačují,
že program implementující metodu z dokazovače iProver společně
s metodou MACE by si mohl vést ještě lépe\footnote{K volbě
vhodné metody pro daný problém lze použít strojové učení.}.

Při popisu metod pro hledání modelů jsme kromě metody MACE
věnovali značnou pozornost i metodě SEM.
Její implementace, program Mace4, si však v našem experimentálním
srovnání příliš dobře nevedla. Podle našeho názoru
je hlavním důvodem absence učení klauzulí.
Námětem na další práci je tedy implementace metody SEM
s lepší propagací pro SEM, s učením klauzulí
a případně i se dvěma sledovanými literály.
Takto implementovanou metodu SEM můžeme chápat jako krok
mezi DPLL s učením klauzulí a metodami Model Evolution a SGGS.

% Takto implementovaná metoda SEM bude mít silnější, leč
% pomalejší, propagaci, než je jednotková propagace v SAT řešičích.

% Idea, jak rozšířit učení klauzulí na SEM, je poměrně triviální
% (zajímavější je to s efektivní implementací).
%
% Při konfliktu máme klauzuli/podmínku, kde jsou všechny literály false.
% Vezmeme plochou klauzuli (normálně by zploštění přidalo
% proměnné, jenže my známe hodnoty všech buněk, tudíž můžeme
% za tyto proměnné dosadit hodnoty tak, že dostaneme plochou klauzuli,
% jenž má všechny literály false, a rovněž neobsahuje proměnné).
%
% Nyní z této ploché klauzule budeme odrezolvovávat buňky
% z aktuální rozhodovací úrovně -- v pořadí podle stopy (trailu) --
% dokud tam nezbyde jedna buňka z aktuální RÚ.
% Předchůdce buňky je klauzule, která vynutila při propagaci její ohodnocení.
% Například při negativní propagaci z P(0, f(0), f(0)) | C1 a
% ~P(0, 1, f(0)) | C2, kde C1 a C2 obsahují literály jenž jsou false,
% je předchůdce ohodnocení f(0) != 1 klauzule
% f(0) != 1 | C1 | C2. Podobně demodulace se chová jako paramodulace.
% Například demodulace pomocí f(0) = 1 | C1, kde C1 obsahuje
% literály, jenž jsou false, musí do demodulované klauzule přidat
% i literály C1 (jedná se tedy o paramodulaci, která je řízena
% aktuálním částečným ohodnocením -- provádíme ji pouze, když jsou
% všechny literály kromě jednoho false). Kdybychom
% ignorovali C1 nebo C2, tak by naučené klauzule byly
% platné pouze pro aktuální částečné ohodnocení (nebo jeho rozšíření).
%
% S takto získanou klauzulí pak můžeme provést backjump standardním způsobem.
%
% Poznamenejme, že u SEMu jsou klauzule, jenž zaručují, že buňka
% má alespoň jednu hodnotu a nemá dvě nebo více hodnot, implicitní.
% Při rezoluci si je tedy musíme zkonstruovat.
%
% Zplošťovat není třeba celou konfliktní klauzuli, stačí jen
% buňky, co chceme odrezolvovat. Alternativně bychom místo rezoluce
% mohli zkusit nějakou jinou metodu, pak by v některých případech
% zplošťování nemuselo být třeba vůbec.

Při vývoji programu \crossbow{} jsme nepočítali s tím,
že některé modely budou mít po uložení do formátu TPTP
velikost několik gigabajtů. To se projevilo zejména v kategorii
NLP, kde program \crossbow{} některé modely našel,
ale nestihl je v daném časovém limitu celé vypsat.
Tuto nepříjemnost můžeme napravit tak, že do
souborů s modely přestaneme ukládat hodnoty buněk,
jenž lze odvodit propagací.

% Pokud to nepomůže, můžeme kód pro vypisování
% modelů, jenž se snaží formule hezky naformátovat,
% nahradit kódem, který žádné formátování nedělá.

Při návrhu kódování klauzulí do podmínek řešiče Gecode
jsme se snažili omezit množství a velikost
podmínek. I přesto řešiči Gecode velmi často docházela paměť.
Bylo by tedy vhodné přijít s kódováním,
jenž je paměťově šetrnější.
Kromě problémů s nedostatkem paměti byl
řešič Gecode obvykle pomalejší než SAT řešiče.
K vyšší rychlosti by možná pomohla
v řešiči Gecode vestavěná redukce symetrií LDSB \cite{ldsb}.
Bohužel se nám nepodařilo vymyslet způsob,
jak ji použít, aniž by došlo k nárůstu
množství proměnných a podmínek.

V našich experimentech měly hledače modelů pouze 2 minuty
na vyřešení problému. Je otázkou, jak by se změnily
výsledky, kdybychom časový limit zvedli na několik hodin?

% Hlavní otázkou je, jestli nenastává problém horizontu.

% Další vylepšení lze čerpat ze SAT řešičů, SMT řešičů
% a řešičů omezujících podmínek.
% Příkladem je deaktivování naučených kauzulí nebo SEM s teoriemi
% (například lineární aritmetikou).


\renewcommand\bibname{Seznam použité literatury}

\printbibliography[heading=bibintoc]

\end{document}
