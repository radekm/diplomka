\chapter{Implementace}

Cílem této kapitoly je popsat náš hledač konečných modelů, který
se jmenuje \crossbow. \crossbow{} implementuje metodu MACE
a některé její modifikace.
Kromě modifikací obsažených v minulé kapitole jsou implementovány
i úplně nové, dosud nevyzkoušené, modifikace.
Tato kapitola začíná popisem nových modifikací metody MACE.
Poté následuje stručný popis samotné implementace.

Hlavním důvodem, proč jsme se rozhodli založit program
\crossbow{} právě na metodě MACE,
byly výsledky metody MACE při řešení problémů --
například program Paradox \cite{paradox} vyhrál
6 ročníků soutěže CASC \cite{sutcliffe2006casc}. Další příjemnou vlastností
metody MACE je, že funguje s existujícími SAT řešiči --
není třeba vytvářet specializovaný řešič jako u metody SEM.

\section{Další modifikace metody MACE}

\subsection{Zplošťování}

\subsection{Odzplošťování}

\subsection{Redundantní klauzule}

\subsection{Komutativní funkce}

Momentálně se detekují pouze komutativní funkce, symetrické relace nikoliv.

\subsection{Převod pro Gecode}

\subsection{Redundantní podmínky pro Gecode}

\subsection{Hledání všech neizomorfních modelů}

\subsection{SAT řešič s podporou dalších podmínek}

Josat, SAT řešič s podporou funkcí.

\section{Výběr programovacího jazyka}

\section{Architektura programu}
