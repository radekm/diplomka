\chapter{Implementace}

Cílem této kapitoly je popsat náš hledač konečných modelů, který
se jmenuje \crossbow. \crossbow{} implementuje metodu MACE
a některé její modifikace.
Kromě modifikací obsažených v minulé kapitole jsou implementovány
i úplně nové, dosud nevyzkoušené, modifikace.
Tato kapitola začíná popisem nových modifikací metody MACE.
Poté následuje stručný popis samotné implementace.

Hlavním důvodem, proč jsme se rozhodli založit program
\crossbow{} právě na metodě MACE,
byly výsledky metody MACE při řešení problémů --
například program Paradox \cite{paradox} vyhrál
6 ročníků soutěže CASC \cite{sutcliffe2006casc}. Další příjemnou vlastností
metody MACE je, že funguje s existujícími SAT řešiči --
není třeba vytvářet specializovaný řešič jako u metody SEM.

\section{Další modifikace metody MACE}

\subsection{Zplošťování}

Jako první popíšeme algoritmus zplošťování používaný
programem \crossbow. Pravidla pro zplošťování klauzulí
uvedená v předchozí kapitole lze použít mnoha různými způsoby.
Například klauzuli
\begin{equation} \label{eq:flatten-input}
  \func(\var) \approx \funcC \vee \funcG(\var) \approx \funcC
\end{equation}
můžeme zploštit na
\begin{equation} \label{eq:flatten-bad-output}
  \func(\var) \not\approx \varY \vee \funcG(\var) \not\approx \varZ \vee
    \varY \approx \funcC \vee \varZ \approx \funcC
\end{equation}
nebo na
\begin{equation} \label{eq:flatten-good-output}
  \funcC \not\approx \varY \vee
    \func(\var) \approx \varY \vee \funcG(\var) \approx \varY.
\end{equation}
V prvním případě byly přidány dvě nové proměnné, ve druhém případě
pouze jedna nová proměnná. Náš algoritmus se pochopitelně snaží
přidat co nejméně proměnných.

Vstupem algoritmu je klauzule, jenž se má zploštit.
Výstupem algoritmu je buď plochá klauzule nebo rozhodnutí,
že se má klauzule odstranit, neboť je vždy splněná.

Algoritmus se skládá ze sedmi kroků, které se, pokud
není řečeno jinak, vykonávají postupně.
Každý krok se skládá z předpokladu a akce.
Akce se provádí pouze v případě, že je splněn předpoklad.
Pokud není splněn předpoklad, pokračuje se dalším krokem.
Některé kroky využívají tzv. nové proměnné, což
jsou proměnné, které se v klauzuli nevyskytovaly
předtím, než se daný krok začal vykonávat.

{
\def\assumpt{\textbf{Předpoklad:}}
\def\action{\textbf{Akce:}}
\def\goto#1{Dále se pokračuje krokem #1).}
\begin{itemize}
\item[1)]
\assumpt{} Klauzule obsahuje literál $\term \approx \term$ nebo
literály $\lit$ a $\neg \lit$ nebo literály $t \approx s$
a $s \not\approx t$.

\action{} Klauzule se odstraní, zplošťování končí.

\item[2)]
\assumpt{} Klauzule obsahuje literál $\lit$ tvaru $\term \not\approx \term$.

\action{} Literál $\lit$ se odstraní z klauzule.
\goto{2}

\item[3)]
\assumpt{} Klauzule obsahuje literál
$\lit$ tvaru $\var \not\approx \varY$.

\action{} Z klauzule se odstraní literál $\lit$ a všechny
výskyty $\var$ se nahradí $\varY$.
\goto{1}

% Musí se pokračovat krokem 1):
% Máme-li x != y | f(y) = z | f(x) != z, tak zrušením rovnosti proměnných
% dostaneme $f(y) = z | f(y) != z$.

\item[4)]
\assumpt{} Klauzule obsahuje literál $\lit$ tvaru
$\term \not\approx \var$ (resp. $\var \not\approx \term$) a
$\term$ se vyskytuje mimo $\lit$.

\action{} Všechny výskyty $\term$ mimo výskytu v $\lit$ se nahradí $\var$.
\goto{1}

% $\term$ má v $\lit$ právě jeden výskyt
% (díky 3) totiž víme, že $\term$ není proměnná).

\item[5)]
\assumpt{} Klauzule obsahuje atom
$\pred(\ldots, \term, \ldots)$ nebo
$\func(\ldots, \term, \ldots) \approx \termS$
(resp. $\termS \approx \func(\ldots, \term, \ldots)$),
kde $\term$ není proměnná.

\action{} Do klauzule se přidá literál $\term \not\approx \var$, kde
$\var$ je nová proměnná.
\goto{4}

\item[6)]
\assumpt{} Klauzule obsahuje literál $\lit$ tvaru
$\term \not\approx \termS$, kde $\term$ a $\termS$ nejsou
proměnné.

\action{} Z klauzule se odstraní literál $\lit$ a přidají
se tam literály $\term \not\approx \var$ a $\var \not\approx \termS$,
kde $\var$ je nová proměnná.
\goto{4}

\item[7)]
Nechť
$\term_1 \approx \term_2 \comdots \term_{2n-1} \approx \term_{2n}$
jsou všechny literály z klauzule, jenž mají tvar $\term \approx \termS$,
kde $\term$ ani $\termS$ nejsou proměnné.
(Tj. $\term_i$ pro každé $i \in \{ 1 \comdots 2n \}$ není proměnná.)

\assumpt{} $n \ge 1$.

\action{} Vytvoří se neorientovaný graf s vrcholy
$\term_1 \comdots \term_{2n}$. Mezi vrcholy $\term_i$ a $\term_j$ vede hrana
právě tehdy, když literál $\term_i \approx \term_j$ je v klauzuli.
Najde se minimální vrcholové pokrytí $\term_{i_1} \comdots \term_{i_k}$.
Do klauzule se přidá literál $\term_{i_j} \not\approx \var_j$ pro
každé $j \in \{ 1 \comdots k \}$, kde $\var_1 \comdots \var_k$ jsou
navzájem různé nové proměnné.
\goto{4}
\end{itemize}
}

Například u klauzule (\ref{eq:flatten-input}) se v pravidle 7)
vytvoří graf s vrcholy $\func(\var), \funcC, \funcG(\var)$
a hranami $\{ \func(\var), \funcC \}$ a $\{ \funcG(\var), \funcC \}$.
Minimální vrcholové pokrytí tohoto grafu je vrchol $\funcC$,
proto se do klauzule přidá literál $\funcC \not\approx y$.
Dále se pokračuje krokem 4), který nahradí výskyty $\funcC$ mimo literál
$\funcC \not\approx y$, čímž dostaneme klauzuli
(\ref{eq:flatten-good-output}).

Akci v pravidle 6) bychom mohli změnit tak, že by se literál $\lit$
neodstraňoval a pouze by se přidal jeden z literálů
$\term \not\approx \var$, nebo $\var \not\approx \termS$.
Pravidlo 4) by pak s~pomocí přidaného literálu změnilo literál $\lit$
na nepřidaný literál.

Všimněme si asymetrie mezi rovností a nerovností neproměnných.
Pravidlo 6) přidáním jedné nové proměnné získá
dva literály $\term \not\approx \var$ a $\var \not\approx \termS$,
které může použít pravidlo 4) k nahrazení termů $\term$ a $\termS$
proměnnou $\var$.

Na druhé straně pravidlo 7) použité
na rovnost neproměnných $\term \approx \termS$ přidáním jedné
nové proměnné získá pouze jeden z literálů $\term \not\approx \var$,
nebo $\termS \not\approx \var$, které může použít pravidlo 4).
Abychom získali i druhý literál, museli bychom přidat další novou proměnnou.
Na rozdíl od pravidla 6) tedy pravidlo 7) musí pečlivě zvažovat,
zda chce literál pro levou stranu rovnosti, nebo pro pravou stranu
rovnosti, nebo literály pro obě strany rovnosti.

Například pro zploštění klauzule
\[
\funcC \approx \funcC' \vee
  \funcC \approx \funcC_1 \vee \funcC \approx \funcC_2 \vee
  \funcC' \approx \funcC'_1 \vee \funcC' \approx \funcC'_2,
\]
kde $\funcC, \funcC_1, \funcC_2, \funcC', \funcC'_1, \funcC'_2$ jsou
funkční symboly bez argumentů, přidá pravidlo 7)
literály $\funcC \not\approx \var_1$ a $\funcC' \not\approx \var_2$,
jenž obsahují dvě nové proměnné $x_1$ a $x_2$.
Kdyby však první literál klauzule byl nerovnost, tj.
$\funcC \not\approx \funcC'$, tak by pravidlu 6)
stačila pouze jedna nová proměnná.

\subsection{Odzplošťování}

Jelikož preferujeme klauzule s malým
počtem proměnných, snažili jsme se náš zplošťovací algoritmus
navrhnout tak, aby zaváděl co nejmenší množství nových proměnných.
Ovšem i tak existují klauzule, které jde zploštit lépe.
Příkladem takové klauzule je (\ref{eq:flatten-bad-output}).
Tato klauzule již je plochá a náš algoritmus ji nijak nezmění.
Víme však, že existuje ekvivalentní plochá klauzule,
jenž obsahuje méně proměnných, tou klauzulí je
(\ref{eq:flatten-good-output}).

Abychom při vstupu (\ref{eq:flatten-bad-output}) dostali
výstup (\ref{eq:flatten-good-output}),
provedeme odzplošťování před zplošťováním.
Odzplošťování je založeno na stejných pravidlech jako zplošťování
(viz sekce \ref{sec:mace-basic}).
Rozdíl je v tom, že odzplošťování používá pravidla naopak:
\begin{itemize}
\item Je-li $\lit$ literál z $\clause$ tvaru $\term \not\approx \var$
  (resp. $\var \not\approx \term$) takový, že $\var$ není obsažena
  v~$\term$ ani v ostatních literálech $\clause$ mimo $\lit$,
  pak můžeme literál $\lit$ odstranit\footnote{Ve skutečnosti
  se nejedná o inverzní pravidlo k prvnímu pravidlu zplošťování.
  Jednak proto, že při zplošťování se přidávaly literály
  tvaru $\term \not\approx \var$, nyní se však odebírají i literály tvaru
  $\var \not\approx \term$. Dále proto, že při zplošťování
  se přidával literál $\term \not\approx \var$ pouze v případě,
  kdy se term $\term$ vyskytoval v klauzuli $\clause$.
  Nyní se literál $\lit$ odstraňuje, i když se term $\term$ v ostatních
  literálech mimo $\lit$ nevyskytuje.}
  z~klauzule $\clause$.
\item Je-li $\lit$ literál z $\clause$ tvaru $\term \not\approx \var$
  (resp. $\var \not\approx \term$),
  pak můžeme $\clause$ upravit tak, že jeden výskyt $\var$
  mimo $\lit$ nahradíme $\term$.
\end{itemize}

Zkombinováním obou opačných pravidel dostaneme pravidlo odzplošťování:
\begin{itemize}
\item Je-li $\lit$ literál z $\clause$ tvaru $\term \not\approx \var$
  (resp. $\var \not\approx \term$) takový, že $\var$ není obsažena
  v~$\term$, pak můžeme všechny výskyty $\var$ mimo výskytu
  v $\lit$ nahradit $\term$
  a literál $\lit$ odstranit.
\end{itemize}
Odzplošťování klauzule $\clause$ se provádí tak,
že se na ni aplikuje uvedené pravidlo, dokud se klauzule mění.

Například z klauzule (\ref{eq:flatten-bad-output})
vytvoří odzplošťování klauzuli (\ref{eq:flatten-input}),
kterou pak zplošťování převede na klauzuli (\ref{eq:flatten-good-output}).

Poznamenejme, že náš algoritmus zplošťování neobsahuje následující pravidlo
z programu Paradox, jenž jsme uvedli v sekci \ref{sec:flatten-extended}:
\begin{itemize}
\item Klauzuli $\clause = \lit_1 \vee \cdots \vee \lit_n$, kde literál
  $\lit_i$ je tvaru $\var \approx \varY$ a literál $L_j$ je tvaru
  $\term \not\approx \varY$, můžeme nahradit klauzulí
  $\clause' = \term \approx \var \vee
  \bigvee_{k \in \{ 1 \comdots n \} \setminus \{ i, j \}} \lit_k$, pokud $\clause'$
  neobsahuje $\varY$.
\end{itemize}
Místo něj lze totiž použít odzplošťování (term $\term$ neobsahuje
proměnnou $\varY$) ná\-sle\-do\-va\-né zplošťováním.

\subsection{Redundantní klauzule}

\subsection{Komutativní funkce}

Momentálně se detekují pouze komutativní funkce, symetrické relace nikoliv.

\subsection{Převod pro Gecode}

\subsection{Redundantní podmínky pro Gecode}

\subsection{Hledání všech neizomorfních modelů}

\subsection{SAT řešič s podporou dalších podmínek}

Josat, SAT řešič s podporou funkcí.

\section{Výběr programovacího jazyka}

\section{Architektura programu}
