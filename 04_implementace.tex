\chapter{Implementace}

Cílem této kapitoly je popsat náš hledač konečných modelů, který
se jmenuje \crossbow. \crossbow{} implementuje metodu MACE
a některé její modifikace.
Kromě modifikací obsažených v minulé kapitole jsou implementovány
i úplně nové, dosud nevyzkoušené, modifikace.
Tato kapitola začíná popisem nových modifikací metody MACE.
Poté následuje stručný popis samotné implementace.

Hlavním důvodem, proč jsme se rozhodli založit program
\crossbow{} právě na metodě MACE,
byly výsledky metody MACE při řešení problémů --
například program Paradox \cite{paradox} vyhrál
6 ročníků soutěže CASC \cite{sutcliffe2006casc}. Další příjemnou vlastností
metody MACE je, že funguje s existujícími SAT řešiči --
není třeba vytvářet specializovaný řešič jako u metody SEM.

\section{Další modifikace metody MACE}

\subsection{Zplošťování}

Jako první popíšeme algoritmus zplošťování používaný
programem \crossbow. Pravidla pro zplošťování klauzulí
uvedená v předchozí kapitole lze použít mnoha různými způsoby.
Například klauzuli
\begin{equation} \label{eq:flatten-input}
  \func(\var) \approx \funcC \vee \funcG(\var) \approx \funcC
\end{equation}
můžeme zploštit na
\begin{equation} \label{eq:flatten-bad-output}
  \func(\var) \not\approx \varY \vee \funcG(\var) \not\approx \varZ \vee
    \varY \approx \funcC \vee \varZ \approx \funcC
\end{equation}
nebo na
\begin{equation} \label{eq:flatten-good-output}
  \funcC \not\approx \varY \vee
    \func(\var) \approx \varY \vee \funcG(\var) \approx \varY.
\end{equation}
V prvním případě byly přidány dvě nové proměnné, ve druhém případě
pouze jedna nová proměnná. Náš algoritmus se pochopitelně snaží
přidat co nejméně proměnných.

Vstupem algoritmu je klauzule, jenž se má zploštit.
Výstupem algoritmu je buď plochá klauzule nebo rozhodnutí,
že se má klauzule odstranit, neboť je vždy splněná.

Algoritmus se skládá ze sedmi kroků, které se, pokud
není řečeno jinak, vykonávají postupně.
Každý krok se skládá z předpokladu a akce.
Akce se provádí pouze v případě, že je splněn předpoklad.
Pokud není splněn předpoklad, pokračuje se dalším krokem.
Některé kroky využívají tzv. nové proměnné, což
jsou proměnné, které se v klauzuli nevyskytovaly
předtím, než se daný krok začal vykonávat.

{
\def\assumpt{\textbf{Předpoklad:}}
\def\action{\textbf{Akce:}}
\def\goto#1{Dále se pokračuje krokem #1).}
\begin{itemize}
\item[1)]
\assumpt{} Klauzule obsahuje literál $\term \approx \term$ nebo
literály $\lit$ a $\neg \lit$ nebo literály $t \approx s$
a $s \not\approx t$.

\action{} Klauzule se odstraní, zplošťování končí.

\item[2)]
\assumpt{} Klauzule obsahuje literál $\lit$ tvaru $\term \not\approx \term$.

\action{} Literál $\lit$ se odstraní z klauzule.
\goto{2}

\item[3)]
\assumpt{} Klauzule obsahuje literál
$\lit$ tvaru $\var \not\approx \varY$.

\action{} Z klauzule se odstraní literál $\lit$ a všechny
výskyty $\var$ se nahradí $\varY$.
\goto{1}

% Musí se pokračovat krokem 1):
% Máme-li x != y | f(y) = z | f(x) != z, tak zrušením rovnosti proměnných
% dostaneme $f(y) = z | f(y) != z$.

\item[4)]
\assumpt{} Klauzule obsahuje literál $\lit$ tvaru
$\term \not\approx \var$ (resp. $\var \not\approx \term$) a
$\term$ se vyskytuje mimo $\lit$.

\action{} Všechny výskyty $\term$ mimo výskytu v $\lit$ se nahradí $\var$.
\goto{1}

% $\term$ má v $\lit$ právě jeden výskyt
% (díky 3) totiž víme, že $\term$ není proměnná).

\item[5)]
\assumpt{} Klauzule obsahuje atom
$\pred(\ldots, \term, \ldots)$ nebo
$\func(\ldots, \term, \ldots) \approx \termS$
(resp. $\termS \approx \func(\ldots, \term, \ldots)$),
kde $\term$ není proměnná.

\action{} Do klauzule se přidá literál $\term \not\approx \var$, kde
$\var$ je nová proměnná.
\goto{4}

\item[6)]
\assumpt{} Klauzule obsahuje literál $\lit$ tvaru
$\term \not\approx \termS$, kde $\term$ a $\termS$ nejsou
proměnné.

\action{} Z klauzule se odstraní literál $\lit$ a přidají
se tam literály $\term \not\approx \var$ a $\var \not\approx \termS$,
kde $\var$ je nová proměnná.
\goto{4}

\item[7)]
Nechť
$\term_1 \approx \term_2 \comdots \term_{2n-1} \approx \term_{2n}$
jsou všechny literály z klauzule, jenž mají tvar $\term \approx \termS$,
kde $\term$ ani $\termS$ nejsou proměnné.
(Tj. $\term_i$ pro každé $i \in \{ 1 \comdots 2n \}$ není proměnná.)

\assumpt{} $n \ge 1$.

\action{} Vytvoří se neorientovaný graf s vrcholy
$\term_1 \comdots \term_{2n}$. Mezi vrcholy $\term_i$ a $\term_j$ vede hrana
právě tehdy, když literál $\term_i \approx \term_j$ je v klauzuli.
Najde se minimální vrcholové pokrytí $\term_{i_1} \comdots \term_{i_k}$.
Do klauzule se přidá literál $\term_{i_j} \not\approx \var_j$ pro
každé $j \in \{ 1 \comdots k \}$, kde $\var_1 \comdots \var_k$ jsou
navzájem různé nové proměnné.
\goto{4}
\end{itemize}
}

Například u klauzule (\ref{eq:flatten-input}) se v pravidle 7)
vytvoří graf s vrcholy $\func(\var), \funcC, \funcG(\var)$
a hranami $\{ \func(\var), \funcC \}$ a $\{ \funcG(\var), \funcC \}$.
Minimální vrcholové pokrytí tohoto grafu je vrchol $\funcC$,
proto se do klauzule přidá literál $\funcC \not\approx y$.
Dále se pokračuje krokem 4), který nahradí výskyty $\funcC$ mimo literál
$\funcC \not\approx y$, čímž dostaneme klauzuli
(\ref{eq:flatten-good-output}).

Akci v pravidle 6) bychom mohli změnit tak, že by se literál $\lit$
neodstraňoval a pouze by se přidal jeden z literálů
$\term \not\approx \var$, nebo $\var \not\approx \termS$.
Pravidlo 4) by pak s~pomocí přidaného literálu změnilo literál $\lit$
na nepřidaný literál.

Všimněme si asymetrie mezi rovností a nerovností neproměnných.
Pravidlo 6) přidáním jedné nové proměnné získá
dva literály $\term \not\approx \var$ a $\var \not\approx \termS$,
které může použít pravidlo 4) k nahrazení termů $\term$ a $\termS$
proměnnou $\var$.

Na druhé straně pravidlo 7) použité
na rovnost neproměnných $\term \approx \termS$ přidáním jedné
nové proměnné získá pouze jeden z literálů $\term \not\approx \var$,
nebo $\termS \not\approx \var$, které může použít pravidlo 4).
Abychom získali i druhý literál, museli bychom přidat další novou proměnnou.
Na rozdíl od pravidla 6) tedy pravidlo 7) musí pečlivě zvažovat,
zda chce literál pro levou stranu rovnosti, nebo pro pravou stranu
rovnosti, nebo literály pro obě strany rovnosti.

Například pro zploštění klauzule
\[
\funcC \approx \funcC' \vee
  \funcC \approx \funcC_1 \vee \funcC \approx \funcC_2 \vee
  \funcC' \approx \funcC'_1 \vee \funcC' \approx \funcC'_2,
\]
kde $\funcC, \funcC_1, \funcC_2, \funcC', \funcC'_1, \funcC'_2$ jsou
funkční symboly bez argumentů, přidá pravidlo 7)
literály $\funcC \not\approx \var_1$ a $\funcC' \not\approx \var_2$,
jenž obsahují dvě nové proměnné $x_1$ a $x_2$.
Kdyby však první literál klauzule byl nerovnost, tj.
$\funcC \not\approx \funcC'$, tak by pravidlu 6)
stačila pouze jedna nová proměnná.

\subsection{Odzplošťování}

Jelikož preferujeme klauzule s malým
počtem proměnných, snažili jsme se náš zplošťovací algoritmus
navrhnout tak, aby zaváděl co nejmenší množství nových proměnných.
Ovšem i tak existují klauzule, které jde zploštit lépe.
Příkladem takové klauzule je (\ref{eq:flatten-bad-output}).
Tato klauzule již je plochá a náš algoritmus ji nijak nezmění.
Víme však, že existuje ekvivalentní plochá klauzule,
jenž obsahuje méně proměnných, tou klauzulí je
(\ref{eq:flatten-good-output}).

Abychom při vstupu (\ref{eq:flatten-bad-output}) dostali
výstup (\ref{eq:flatten-good-output}),
provedeme odzplošťování před zplošťováním.
Odzplošťování je založeno na stejných pravidlech jako zplošťování
(viz sekce \ref{sec:mace-basic}).
Rozdíl je v tom, že odzplošťování používá pravidla naopak:
\begin{itemize}
\item Je-li $\lit$ literál z $\clause$ tvaru $\term \not\approx \var$
  (resp. $\var \not\approx \term$) takový, že $\var$ není obsažena
  v~$\term$ ani v ostatních literálech $\clause$ mimo $\lit$,
  pak můžeme literál $\lit$ odstranit\footnote{Ve skutečnosti
  se nejedná o inverzní pravidlo k prvnímu pravidlu zplošťování.
  Jednak proto, že při zplošťování se přidávaly literály
  tvaru $\term \not\approx \var$, nyní se však odebírají i literály tvaru
  $\var \not\approx \term$. Dále proto, že při zplošťování
  se přidával literál $\term \not\approx \var$ pouze v případě,
  kdy se term $\term$ vyskytoval v klauzuli $\clause$.
  Nyní se literál $\lit$ odstraňuje, i když se term $\term$ v ostatních
  literálech mimo $\lit$ nevyskytuje.}
  z~klauzule $\clause$.
\item Je-li $\lit$ literál z $\clause$ tvaru $\term \not\approx \var$
  (resp. $\var \not\approx \term$),
  pak můžeme $\clause$ upravit tak, že jeden výskyt $\var$
  mimo $\lit$ nahradíme $\term$.
\end{itemize}

Zkombinováním obou opačných pravidel dostaneme pravidlo odzplošťování:
\begin{itemize}
\item Je-li $\lit$ literál z $\clause$ tvaru $\term \not\approx \var$
  (resp. $\var \not\approx \term$) takový, že $\var$ není obsažena
  v~$\term$, pak můžeme všechny výskyty $\var$ mimo výskytu
  v $\lit$ nahradit $\term$
  a literál $\lit$ odstranit.
\end{itemize}
Odzplošťování klauzule $\clause$ se provádí tak,
že se na ni aplikuje uvedené pravidlo, dokud se klauzule mění.

Například z klauzule (\ref{eq:flatten-bad-output})
vytvoří odzplošťování klauzuli (\ref{eq:flatten-input}),
kterou pak zplošťování převede na klauzuli (\ref{eq:flatten-good-output}).

Poznamenejme, že náš algoritmus zplošťování neobsahuje následující pravidlo
z programu Paradox, jenž jsme uvedli v sekci \ref{sec:flatten-extended}:
\begin{itemize}
\item Klauzuli $\clause = \lit_1 \vee \cdots \vee \lit_n$, kde literál
  $\lit_i$ je tvaru $\var \approx \varY$ a literál $L_j$ je tvaru
  $\term \not\approx \varY$, můžeme nahradit klauzulí
  $\clause' = \term \approx \var \vee
  \bigvee_{k \in \{ 1 \comdots n \} \setminus \{ i, j \}} \lit_k$, pokud $\clause'$
  neobsahuje $\varY$.
\end{itemize}
Místo něj lze totiž použít odzplošťování (term $\term$ neobsahuje
proměnnou $\varY$) ná\-sle\-do\-va\-né zplošťováním.

\subsection{Redundantní klauzule}

Abychom posílili propagaci SAT řešiče, přidáme
ke vstupní množině klauzulí $\clauses$ další klauzule,
jenž plynou z $\clauses$.

Program \crossbow{} tyto klauzule získá pomocí dokazovače
E \cite{eprover18}.
Z klauzulí odvozených dokazovačem E vybere malé klauzule
a přidá je do $\clauses$.

\subsection{Komutativní funkce}

Pokud víme, že funkce $\func^\interp$ je komutativní,
můžeme pro buňky $\func(v_1, v_2)$ a $\func(v_2, v_1)$
používat stejné výrokové proměnné.
To znamená, že při vytváření instance SATu (viz \ref{sec:mace-basic})
budeme v kroku 1) přidávat výrokové proměnné a klauzule pouze
pro buňky $\func(v_1, v_2)$, kde $v_1 \le v_2$.
Každou výrokovou proměnnou $A_{\func(v_1, v_2) = v}$, kde $v_1 > v_2$,
nahradíme proměnnou $A_{\func(v_2, v_1) = v}$.

Program \crossbow{} mezi vstupními klauzulemi a mezi všemi
klauzulemi z~dokazovače E hledá klauzuli ekvivalentní
s klauzulí $\func(\var, \varY) \approx \func(\varY, \var)$.
Pokud takovou klauzuli najde, označí funkci $\func^\interp$
jako komutativní a při vytváření instance SATu se použije
optimalizace uvedená v předchozím odstavci.

Podobně bychom mohli postupovat i u symetrických relací,
to však program \crossbow{} nedělá.

\subsection{Převod pro Gecode}

{
\def\Eq{\textproc{Eq}}
\def\LowerEq{\textproc{Lower-Eq}}
\def\Clause{\textproc{Clause}}
\def\Element{\textproc{Element}}
\def\Linear{\textproc{Linear}}
\def\Precede{\textproc{Precede}}

Chceme-li se vyhnout zplošťování klauzulí a s ním spojenému nárůstu počtu
proměnných, můžeme klauzule kódovat do bohatšího jazyka, než je SAT.
Program \crossbow{} umí klauzule kódovat do jazyka
omezujících podmínek řešiče Gecode \cite{gecode}.
Při kódování se používají booleovské proměnné, celočíselné proměnné,
pole booleovských i celočíselných proměnných, celočíselné konstanty
a jejich pole. Pole jsou indexovány od 0.
Booleovské proměnné značíme písmenem
$b$, celočíselné proměnné nebo konstanty značíme písmenem $i$,
pole booleovských proměnných značíme písmenem $B$
a pole celočíselných proměnných nebo konstant značíme písmenem $I$.
Z~podmínek se používají:
\begin{itemize}
\item \Eq$(i_1, i_2, b)$. Podmínka je splněna právě tehdy,
  když $i_1 = i_2 \wedge b = 1$ nebo $i_1 \neq i_2 \wedge b = 0$.
\item \LowerEq$(i_1, i_2)$. Podmínka je splněna právě tehdy,
  když $i_1 \le i_2$.
\item \Clause$(B_P, B_N)$. Podmínka je splněna právě tehdy,
  když je nějaká booleovská proměnná z $B_P$ ohodnocena 1
  nebo je nějaká booleovská proměnná z~$B_N$ ohodnocena 0.
\item \Element$(H, i, h)$, kde $H$ resp. $h$ je buď pole booleovských
  proměnných resp. booleovská proměnná, nebo pole celočíselných proměnných
  resp. celo\-číselná proměnná. $i$ je vždy celočíselná proměnná.
  Podmínka je splněna právě tehdy,
  když hodnota proměnné v poli $H$ s indexem $i$ je stejná jako
  hodnota proměnné $h$.
\item \Linear$(I_c, I_x, i)$, kde $I_c$ a $I_x$ jsou pole stejné délky,
  $I_c$ je pole konstant, $I_x$ je pole proměnných a $i$ je konstanta.
  Podmínka je splněna právě tehdy, když
  \[
    \sum^{|I_c|-1}_{k=0} I_c(k) \cdot I_x(k) = i.
  \]
\item \Precede$(I_x, I_c)$, kde $I_x$ je pole proměnných a
  $I_c$ je pole konstant. Podmínka je splněna právě tehdy, když
  pro každé $j \in \{ 0 \comdots |I_x| - 1 \}$ a pro každé
  $k \in \{ 1 \comdots |I_c| - 1 \}$ platí
  \[
    I_x(j) = I_c(k) \implies
      \exists j' \in \{ 0 \comdots j - 1 \}: I_x(j') = I_c(k - 1).
  \]
  Jinak řečeno, kdykoliv má proměnná z $I_x$ hodnotu $i$
  z $I_c(1 \comdots \star)$, tak v $I_x$ existuje proměnná
  s nižším indexem a s hodnotou, jenž je v poli $I_c$ těsně před $i$.
\end{itemize}

Začneme tím, že ukážeme, jak do uvedených podmínek zakódovat
ploché klauzule. Následně výklad rozšíříme i o neploché klauzule
a o LNH.

\subsubsection{Ploché klauzule}

Pro zakódování plochých klauzulí stačí podmínky \Eq{} a \Clause{}.
Napřed pro každou funkci $\func^\interp$ a každou její buňku
$c \in \cells_\func$ přidáme celočíselnou proměnnou $i_c$
s doménou $D_\sort$, kde $\sort$ je sorta výsledku $\func$.
Dále pro každou relaci $\pred^\interp$ a každou její buňku
$c \in \cells_\pred$ přidáme booleovskou proměnnou $b_c$.
Při přidávání proměnných pro buňky uplatňuje program \crossbow{}
trik s komutativními funkcemi.

Nyní přidáme podmínky pro klauzule. Mějme klauzuli
$\clause \in \clauses$ a ohodnocení proměnných $\beta$.
Pokud $\clause$ obsahuje literál $\lit$ s atomem $\var \approx \varY$
takový, že $\lit$ je splněný při ohodnocení $\beta$, tak
pro klauzuli $\clause$ a ohodnocení $\beta$ žádné podmínky nepřidáváme.
Dále předpokládejme, že klauzule žádný takový literál neobsahuje.
Označme $\lit_1 \comdots \lit_j$ všechny literály z klauzule,
jenž neobsahují atom tvaru $\var \approx \varY$.
Každému literálu z $\lit_k$ z $\lit_1 \comdots \lit_j$
přiřadíme booleovskou proměnnou $b_k$:
\begin{itemize}
\item Obsahuje-li $\lit_k$ atom tvaru
  $\func(\var_1 \comdots \var_n) \approx \var$ (resp.
  $\var \approx \func(\var_1 \comdots \var_n)$),
  tak $b_k$ bude nová booleovská proměnná
  a přidáme podmínku
  \[
    \Eq(i_{\func(\beta(\var_1) \comdots \beta(\var_n))}, \beta(\var), b_k).
  \]
\item Obsahuje-li $\lit_k$ atom tvaru
  $\pred(\var_1 \comdots \var_n)$, tak $b_k$ bude existující proměnná
  \[
    b_{\pred(\beta(\var_1) \comdots \beta(\var_n))}
  \]
  a žádnou podmínku nebudeme přidávat.
\end{itemize}
Následně přidáme podmínku \Clause$(B_P, B_N)$, kde pole
proměnných $B_P$ obsahuje právě ty proměnné
$b_k$, jenž jsou přiřazeny literálům bez negace.
Naopak pole proměnných $B_N$ obsahuje právě ty proměnné $b_k$,
jenž jsou přiřazeny literálům s~negací.

Abychom dostali podmínky pro všechny klauzule,
použijeme uvedený postup na každou klauzuli $\clause \in \clauses$
se všemi ohodnoceními $\beta$, jenž se liší na proměnných
v klauzuli $\clause$.

Ukažme si, jak získat podmínky pro klauzuli
$\pred(\var) \vee \func(\var) \not\approx \varY \vee \var \approx \varY$
s~jedinou doménou velikosti 2. Uvedený postup musíme
aplikovat na 4 ohodnocení $\beta_1, \beta_2, \beta_3$ a $\beta_4$,
která splňují:
\begin{align*}
\beta_1(\var) &= 0, & \beta_1(\varY) &= 0, \\
\beta_2(\var) &= 0, & \beta_2(\varY) &= 1, \\
\beta_3(\var) &= 1, & \beta_3(\varY) &= 0, \\
\beta_4(\var) &= 1, & \beta_4(\varY) &= 1.
\end{align*}

Jelikož je literál $\var \approx \varY$ splněný při ohodnoceních
$\beta_1$ a $\beta_4$, nebudeme pro tato ohodnocení žádné podmínky přidávat.
Pro obě zbylá ohodnocení označíme literály, jenž neobsahují
atom rovnosti:
\begin{align*}
\lit_1 &= \pred(\var), \\
\lit_2 &= \func(\var) \not\approx \varY.
\end{align*}
Pro ohodnocení $\beta_2$ přiřadíme literálu $\lit_1$
existující booleovskou proměnnou $b_{\pred(0)}$, literálu $\lit_2$
přiřadíme novou booleovskou proměnnou $b_2^{\beta_2}$
a přidáme podmínku \Eq$(i_{\func(0)}, 1, b_2^{\beta_2})$.
Nakonec pro ohodnocení $\beta_2$ přidáme podmínku
\[
  \Clause([b_{\pred(0)}], [b_2^{\beta_2}]).
\]
Pro ohodnocení $\beta_3$ přiřadíme literálu $\lit_1$
existující booleovskou proměnnou $b_{\pred(1)}$, literálu $\lit_2$
přiřadíme novou booleovskou proměnnou $b_2^{\beta_3}$
a přidáme podmínku \Eq$(i_{\func(1)}, 0, b_2^{\beta_3})$.
Nakonec pro ohodnocení $\beta_3$ přidáme podmínku
\[
  \Clause([b_{\pred(1)}], [b_2^{\beta_3}]).
\]
Pro klauzuli
$\pred(\var) \vee \func(\var) \not\approx \varY \vee \var \approx \varY$
jsme dohromady přidali 4 podmínky.

Pro literály s atomem tvaru
$\func(\var_1 \comdots \var_n) \approx \var$ (resp.
$\var \approx \func(\var_1 \comdots \var_n)$), není třeba
v některých situacích přidávat novou booleovskou proměnnou
a podmínku. Uvažme například klauzuli
$\func(\var) \not\approx \var \vee \varY \approx \funcC$
s jedinou doménou velikosti 2.

Označme literály klauzule
\begin{align*}
\lit_1 &= \func(\var) \not\approx \var, \\
\lit_2 &= \varY \approx \funcC.
\end{align*}
Následující tabulka ukazuje proměnné a podmínky,
jenž je třeba přidat pro oba literály pro ohodnocení
$\beta_1, \beta_2, \beta_3$ a $\beta_4$:

\begin{center}
{
\renewcommand{\arraystretch}{1.2}
\begin{tabular}{c|c|c}
& $\lit_1$ & $\lit_2$ \\
\hline
$\beta_1$ &
  \Eq$(i_{\func(0)}, 0, b_1^{\beta_1})$ &
  \Eq$(i_{\funcC}, 0, b_2^{\beta_1})$ \\
$\beta_2$ &
  \Eq$(i_{\func(0)}, 0, b_1^{\beta_2})$ &
  \Eq$(i_{\funcC}, 1, b_2^{\beta_2})$ \\
$\beta_3$ &
  \Eq$(i_{\func(1)}, 1, b_1^{\beta_3})$ &
  \Eq$(i_{\funcC}, 0, b_2^{\beta_3})$ \\
$\beta_4$ &
  \Eq$(i_{\func(1)}, 1, b_1^{\beta_4})$ &
  \Eq$(i_{\funcC}, 1, b_2^{\beta_4})$ \\
\end{tabular}
}
\end{center}

Z tabulky je patrné, že pro literál $\lit_1$, pro ohodnocení $\beta_2$
můžeme použít stejnou proměnnou a podmínku jako pro ohodnocení $\beta_1$.
Podobně pro ohodnocení $\beta_3$ můžeme použít stejnou
proměnnou a podmínku jako pro ohodnocení $\beta_4$.
Pro literál $\lit_2$ můžeme rovněž ušetřit dvě booleovské
proměnné a dvě podmínky.

\subsubsection{Obecné klauzule}

\def\Horner{\textproc{Horner}}

Pro každý funkční symbol $\func$ arity
$\langle \sort_1 \comdots \sort_n, \sort \rangle$ přidáme
pole $I_\func$ celočíselných proměnných. Délka
pole bude $\prod_{j=1}^n n_{\sort_j}$. Pole bude obsahovat
proměnné pro buňky funkce $\func^\interp$.
Proměnná pro buňku $\func(v_1 \comdots v_n) \in \cells_\func$
bude v poli $I_\func$ pod indexem
\Horner$(v_1 \comdots v_n, n_{\sort_1} \comdots n_{\sort_n})$.
Funkce \Horner{} je definována následovně:
\medskip
\begin{algorithmic}
\Function{Horner}{$v_1 \comdots v_n, n_{\sort_1} \comdots n_{\sort_n}$}
  \State $j \gets 0$
  \For{$k = 1 \comdots n$}
    \State $j \gets j \cdot n_{\sort_k} + v_k$
  \EndFor
  \State \Return $j$
\EndFunction
\end{algorithmic}
\medskip

\noindent
Pro každý predikátový symbol $\pred$ arity
$\langle \sort_1 \comdots \sort_n \rangle$
přidáme pole $B_\pred$ booleovských proměnných.
Délka pole bude $\prod_{j=1}^n n_{\sort_j}$. Pole bude obsahovat
booleovské proměnné pro buňky $\pred^\interp$.
Proměnná pro buňku $\pred(v_1 \comdots v_n) \in \cells_\pred$
bude v poli $B_\pred$ pod indexem
\Horner$(v_1 \comdots v_n, n_{\sort_1} \comdots n_{\sort_n})$.

Díky právě přidaným polím můžeme zakódovat neplochou klauzuli
$\func(\funcC) \approx \funcC$. Zavedeme pomocnou
proměnnou $i$ a přidáme podmínku \Element$(I_\func, \funcC, i)$,
která zajistí, že proměnná $i$ obsahuje hodnotu funkce $\func^\interp$
v bodě $\funcC^\interp$. Jelikož se tato hodnota musí rovnat hodnotě
$\funcC^\interp$, přidáme stejně jako v případě plochých klauzulí
pomocnou booleovskou proměnnou $b$ a podmínku \Eq$(i, i_\funcC, b)$.
Nakonec přidáme podmínku \Clause$([b], [])$.

Důvodem, proč pro zakódování klauzule potřebujeme pole $I_\func$,
je, že nevíme, jakou buňku $\func(\funcC)$ přesně označuje
(v případě plochých klauzulí jsme to věděli), a tudíž
do podmínky \Eq{} nemůžeme dát proměnnou přímo pro tuto buňku.
Víme pouze to, že proměnná pro buňku $\func(\funcC)$
je v poli $I_\func$ a má index $\funcC^\interp$ -- k získání
její hodnoty použijeme podmínku \Element.

\def\Index{\textproc{Index}}
\def\TermValue{\textproc{Term-Value}}
\def\VarForAtom{\textproc{Var-For-Atom}}

V obecném případě budeme potřebovat zjistit, jakou buňku
označuje term $\func(\term_1 \comdots \term_n)$ resp.
atom $\pred(\term_1 \comdots \term_n)$ při ohodnocení
proměnných $\beta$. Před\-po\-klá\-dejme,
že máme k dispozici funkci \Index{} takovou,
že volání \Index$(\beta, \term_1 \comdots \term_n)$
vrátí index proměnné pro buňku $\func(\term_1 \comdots \term_n)$ resp.
$\pred(\term_1 \comdots \term_n)$ v poli $I_\func$ resp. $B_\pred$
při ohodnocení $\beta$. Index je buď celočíselná proměnná, nebo
konstanta. S pomocí funkce \Index{}
vytvoříme funkce:
\begin{itemize}
\item \TermValue$(\beta, \term)$.
  Vrací celočíselnou proměnnou nebo konstantu
  s hodnotou termu $\term$ při ohodnocení proměnných $\beta$.
\item \VarForAtom$(\beta, \atom)$.
  Vrací booleovskou proměnnou, jenž obsahuje
  hodnotu atomu $\atom$ při ohodnocení proměnných $\beta$.
\end{itemize}

Je-li $\term$ term tvaru $\var$,
funkce \TermValue$(\beta, \term)$ vrátí konstantu
$\beta(\var)$.
Je-li $\term$ term tvaru $\func(\term_1 \comdots \term_n)$,
funkce \TermValue$(\beta, \term)$
zavolá \Index$(\beta, \term_1 \comdots \term_n)$,
čímž dostane index do pole $I_\func$.
Pokud je index konstanta, vrátí funkce proměnnou
z pole $I_\func$ s daným indexem. V opačném případě je
index proměnná $i$. Pak funkce \TermValue{}
zavede novou pomocnou proměnnou $i'$, přidá
podmínku \Element$(I_\func, i, i')$ a vrátí proměnnou $i'$.

Je-li $\atom$ atom tvaru $\term \approx \termS$,
\VarForAtom$(\beta, \atom)$
zavolá \TermValue$(\beta, \term)$
a \TermValue$(\beta, \termS)$, čímž dostane
proměnné nebo konstanty $i$ a $i'$ s hodnotami $\term$ a $\termS$.
Dále \VarForAtom{} zavede novou pomocnou proměnnou $b$,
přidá podmínku \Eq$(i, i', b)$ a vrátí proměnnou $b$.
Je-li $\atom$ atom tvaru $\pred(\term_1 \comdots \term_n)$,
postupuje funkce \VarForAtom$(\beta, \atom)$
analogicky jako funkce \TermValue{} v případě,
že term není proměnná.

Funkce \Index$(\beta, \term_1 \comdots \term_n)$
pro každý term $t_j$, kde $j \in \{ 1 \comdots n \}$ zavolá funkci
\TermValue$(\beta, t_j)$, čímž dostane proměnné
nebo konstanty $i_1 \comdots i_n$.
Mají-li termy $\term_1 \comdots \term_n$ sorty $\sort_1 \comdots \sort_n$,
pak \Index{} musí vrátit hodnotu
\Horner$(i_1 \comdots i_n, n_{\sort_1} \comdots n_{\sort_n})$.
Pokud jsou všechny $i_1 \comdots i_n$ konstanty,
tak vrácená hodnota bude rovněž konstanta. Pokud je mezi
$i_1 \comdots i_n$ proměnná, pak funkce \Index{}
zavede pomocnou proměnnou $i$, pomocí podmínky \Linear{} zajistí,
že $i$ má hodnotu
\Horner$(i_1 \comdots i_n, n_{\sort_1} \comdots n_{\sort_n})$,
a vrátí proměnnou $i$.

Při konstrukci podmínek \Clause{} se literálům přiřadí
booleovské proměnné vrácené funkcí \VarForAtom.

Převod neploché klauzule do podmínek řešiče Gecode si ukážeme na
klauzuli $\func(\funcG(\var), \varY, \funcG(\var)) \approx \varY$,
kde proměnné $\var, \varY$ mají sortu s doménou velikosti 2 a
sorta výsledku $\funcG$ má doménu velikosti 3.

Začneme s podmínkami pro ohodnocení $\beta_1$. Abychom
přiřadili booleovskou proměnnou jedinému literálu
z klauzule, zavoláme funkci \VarForAtom$(\beta_1, \allowbreak
\func(\funcG(\var), \varY, \funcG(\var)) \approx \varY)$.
Tato funkce následně zavolá
\begin{align*}
&\TermValue(\beta_1, \func(\funcG(\var), \varY, \funcG(\var))) \quad\text{a} \\
&\TermValue(\beta_1, \varY).
\end{align*}
Druhé volání vrátí hodnotu proměnné $\varY$ při ohodnocení $\beta_1$,
což je konstanta 0. První volání napřed spočte hodnoty
termů $\var, \funcG(\var), \varY$. Hodnotou termů $\var$ a $\varY$
je konstanta 0. Hodnota termu $\funcG(\var)$ je nultá proměnná
z pole $I_\funcG$, tj. $i_{\funcG(0)}$.
Na základě těchto hodnot musí funkce \Index{} spočítat
index proměnné v poli $I_\func$, jenž obsahuje hodnotu termu
$\func(\funcG(\var), \varY, \funcG(\var))$. Víme,
že hodnota indexu je dána
\Horner$(i_{\funcG(0)}, 0, i_{\funcG(0)}, 3, 2, 3)$.
Zavedeme-li pro index novou pomocnou proměnnou $i_1$, pak musí platit
\[
  (i_{\funcG(0)} \cdot 2 + 0) \cdot 3 + i_{\funcG(0)} = i_1.
\]
Úpravou dostaneme rovnici
\[
  7 \cdot i_{\funcG(0)} + (-1) \cdot i_1 = 0,
\]
jenž zakódujeme pomocí podmínky \Linear$([i_{\funcG(0)}, i_1], [7, -1], 0)$.
Funkce \Index{} vrátí proměnnou $i_1$. Funkce \TermValue{}
zavede novou pomocnou proměnnou $i'_1$ a pomocí podmínky
\Element$(I_\func, i_1, i'_1)$ zajistí, že $i'_1$ bude obsahovat hodnotu termu
$\func(\funcG(\var), \varY, \funcG(\var))$. Funkce
\TermValue$(\beta_1, \func(\funcG(\var), \varY, \funcG(\var)))$
vrátí proměnou $i'_1$.

Nyní jsme zpět ve funkci \VarForAtom{} a známe hodnoty
termů na levé a pravé straně. Jelikož se tyto hodnoty
mají rovnat, přidá \VarForAtom{} novou booleovskou proměnnou $b_1$,
podmínku \Eq$(i'_1, 0, b_1)$ a vrátí $b_1$.
Nakonec je přidána podmínka \Clause$([b_1], [])$.
Stejně lze postupovat pro ohodnocení $\beta_2, \beta_3$ a $\beta_4$.

Pomocné proměnné zavedené pro podmínky \Linear{}, \Element{} a \Eq{}
lze používat opakovaně (pro podmínky \Eq{} viz konec sekce
Ploché klauzule).

\subsubsection{LNH}

Redukci symetrií LNH implementujeme prostřednictvím podmínek
\LowerEq{} a \Precede{}. Popíšeme LNH pro funkční symboly
se sortou výsledku $\sort$. Před\-po\-klá\-dáme,
že všechny hodnoty z $D_\sort$ jsou nepoužité.

Buňky funkčních symbolů se sortou výsledku $\sort$
uspořádáme do posloupnosti.
Buňky, jenž nemají žádný argument sorty $\sort$, budou
před buňkami, jenž mají alespoň jeden
argument sorty $\sort$. Buňky s argumenty sorty $\sort$
budou uspořádány tak, že je-li buňka $c$
před buňkou $c'$, tak maximální argument z $D_\sort$ buňky $c$
není větší než maximální argument z $D_\sort$ buňky $c'$.

Pro buňky funkčních symbolů $c_1 \comdots c_n$,
jenž nemají argument z $D_\sort$,
se přidá podmínka
\[
\begin{split}
  \Precede([&i_{c_1} \comdots i_{c_n}], \\
           [&0 \comdots \min \{ n, n_\sort \} - 1]).
\end{split}
\]
Pro buňky $c'_1 \comdots c'_{n'}$, kde maximální argument z $D_\sort$
je 0, se přidá podmínka
\[
\begin{split}
  \Precede([&i_{c_1} \comdots i_{c_n}, \\
            &i_{c'_1} \comdots i_{c'_{n'}}], \\
           [&1 \comdots \min \{ m + n + n', n_\sort \} - 1]),
\end{split}
\]
kde $m = 0$, pokud $n > 0$, jinak $m = 1$.
Pro buňky $c''_1 \comdots c''_{n''}$ s maximálním argumentem
1 se přidá podmínka
\[
\begin{split}
  \Precede([&i_{c_1} \comdots i_{c_n},\\
            &i_{c'_1} \comdots i_{c'_{n'}},\\
            &i_{c''_1} \comdots i_{c''_{n''}}], \\
           [&2 \comdots \min \{ m + n + n' + n'', n_\sort \} - 1]).
\end{split}
\]
Pro buňky, kde je maximální argument z $D_\sort$ vyšší,
se postupuje analogicky. Podmínka \LowerEq{} se používá
pro omezování velikosti domén.

}

\subsection{Redundantní podmínky pro Gecode}

Abychom posílili propagaci řešiče Gecode, přidáme redundantní
globální pod\-mín\-ky.
Mezi vstupními klauzulemi a klauzulemi z dokazovače E
budeme hledat axiomy grup, kvazigrup a involutorních funkcí.
Tabulky násobení v grupách a kvazigrupách jsou latinské čtverce,
pro každý řádek a sloupec takové tabulky přidáme jednu podmínku
\textproc{All-Different}, jenž vynutí, že daný řádek nebo
sloupec obsahuje každou hodnotu právě jednou.
Tabulky involutorních funkcí obsahují každou hodnotu právě jednou,
pro každou involutorní funkci přidáme jednu podmínku
\textproc{All-Different}.

\subsection{Hledání všech neizomorfních modelů}

V této sekci popíšeme, jak najít více modelů a jak odfiltrovat
izomorfní modely. Řešič Gecode prozkoumává ohodnocení
proměnných systematicky a snadno tak najde všechna ohodnocení splňující
zadané podmínky. U SAT řešičů je situace obvykle odlišná,
proto po nalezení každého modelu přidáme klauzuli,
která zakáže ohodnocení výrokových proměnných, jenž vedou na daný model.
Program \crossbow{} přidává klauzuli, jenž pro každou
funkci z modelu a každou její buňku $c$ s hodnotou $v$
obsahuje literál $\neg A_{c=v}$ a pro každou relaci z modelu a každou její
buňku s hodnotou $v$ obsahuje buňku $A_{c=1}$, pokud $v = 0$, nebo
$\neg A_{c=1}$, pokud $v = 1$.

Nyní dokážeme postupně generovat různé modely.
LNH v obecném případě však nezabrání situaci,
že vygenerovaný model je izomorfní jinému modelu, který byl
vygenerován dříve.
Abychom tyto izomorfní modely odstranili,
mohli bychom pro každý nalezený model
zkusit najít izomorfismus mezi právě nalezeným modelem
a dříve nalezenými modely.
Nevýhoda tohoto řešení spočívá v tom,
že s~rostoucím počtem nalezených modelů bude růst
i čas potřebný na nalezení izomorfismu.

Jiné řešení je použít zobrazení kanonických reprezentantů pro číselné
interpretace. Pomocí něj každý číselný model převedeme
na kanonický číselný model. Díky tomu, že kanonické číselné modely
jsou izomorfní právě tehdy, když se rovnají, můžeme velmi rychle
detekovat, zda je nově nalezený model izomorfní již
dříve nalezenému modelu.

K implementaci zobrazení kanonických reprezentantů
pro číselné interpretace využijeme implementaci
zobrazení kanonických reprezentantů pro orientované barevné grafy
(dále jen grafy). Číselnou interpretaci převedeme na graf,
ke grafu najdeme kanonický graf a na základě kanonického grafu
zkonstruujeme kanonickou číselnou interpretaci.
K nalezení kanonického grafu využívá program
\crossbow{} knihovnu bliss \cite{bliss}.

Konstrukce grafu z interpretace:
\begin{itemize}
\item Pro každou doménu $D_\sort$ vezmeme dosud nepoužitou barvu $i_{D_\sort}$
  a pro každý prvek domény $v \in D_\sort$ přidáme vrchol $u_v$ barvy
  $i_{D_\sort}$.
\item Pro každou relaci $\pred^\interp$ arity
  $\langle \sort_1 \comdots \sort_n \rangle$, kde $n > 0$
  vezmeme $n$ dosud nepoužitých barev, označme je
  $i_1 \comdots i_n$.
  Pro každou buňku $\pred(v_1 \comdots v_n)$ relace $\pred^\interp$
  s~hodnotou 1 přidáme nové vrcholy $u_1^\pred \comdots u_n^\pred$,
  kde vrchol $u_k^\pred$ má barvu $i_k$ pro $k \in \{ 1 \comdots n \}$.
  Nově přidané vrcholy propojíme hranami $(u_{k-1}^\pred, u_k^\pred)$ pro
  $k \in \{ 2 \comdots n \}$. Dále propojíme hranami
  vrcholy pro argumenty s jejich hodnotami, tj. pro
  každé $k \in \{ 1 \comdots n \}$ přidáme hranu
  $(u_k^\pred, u_{v_k})$, kde $u_{v_k}$ je vrchol pro hodnotu
  $v_k$ z domény $D_{\sort_k}$.
\item Pro každou funkci $\func^\interp$ arity
  $\langle \sort_1 \comdots \sort_n, \sort_0 \rangle$
  vezmeme $n + 1$ dosud ne\-pou\-ži\-tých barev, označme je
  $i_0 \comdots i_n$.
  Pro každou buňku $\func(v_1 \comdots v_n)$ funkce $\func^\interp$
  s hodnotou $v_0$ přidáme nové vrcholy $u_0^\func \comdots u_n^\func$,
  kde vrchol $u_k^\func$ má barvu $i_k$ pro $k \in \{ 0 \comdots n \}$.
  Nově přidané vrcholy propojíme hranami $(u_{k-1}^\func, u_k^\func)$ pro
  $k \in \{ 1 \comdots n \}$. Dále propojíme hranami
  vrcholy pro argumenty a vrchol pro hodnotu buňky s jejich hodnotami, tj. pro
  každé $k \in \{ 0 \comdots n \}$ přidáme hranu
  $(u_k^\func, u_{v_k})$, kde $u_{v_k}$ je vrchol pro hodnotu
  $v_k$ z domény $D_{\sort_k}$.
\end{itemize}
Jsou-li dva modely izomorfní, pak\footnote{Nejedná se o ekvivalenci.
Může nastat situace, že grafy budou izomorfní i pro dva neizomorfní modely.}
grafy zkonstruované pro tyto modely výše popsaným způsobem budou
rovněž izomorfní.

% Chceme vlastně normalizovat pořadí prvků v každé doméně.
% Usporádání prvků v doméně provádíme na základě vztahu
% prvků k buňkám.
%
% Příkladem dvou neizomorfních modelů s izomorfními grafy
% jsou modely P=0 a P=1, kde P je nulární predikát.
% Grafy těchto modelů budou obsahovat pouze vrcholy pro prvky
% domén, žádné hrany.

Společně s kanonickým grafem bliss vydá i bijekci vrcholů,
jenž dokazuje, že grafy jsou skutečně izomorfní.
Restrikce této bijekce na vrcholy pro prvky domén
určuje bijekci na prvcích domén.
S pomocí této bijekce můžeme přejmenovat prvky
domén, čímž dostaneme kanonický model.

\subsection{SAT řešič s podporou dalších podmínek}

Další možností, jak snížit počet proměnných nebo počet klauzulí
nebo zlepšit propagaci, je rozšířit SAT řešič o další typy podmínek.
Do SAT řešiče MiniSat 2.2 \cite{minisat} jsme přidali speciální
typ výrokových klauzulí, které jsou splněny právě tehdy,
když obsahují právě jeden splněný literál.
Nový řešič se jmenuje Josat.
Díky speciálním klauzulím není třeba přidávat klauzule
z bodu 1b) sekce \ref{sec:mace-basic}.

Aktuální implementace nedovoluje uvnitř speciálních klauzulí užít negaci
a každá výroková proměnná se může vyskytovat nejvýše v jedné
speciální klauzuli.

\section{Výběr programovacího jazyka}

\section{Architektura programu}
