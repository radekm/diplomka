\chapter{Závěr}

V této práci jsme vytvořili hledač konečných modelů \crossbow{},
který dokáže hledat jeden konečný model nebo všechny navzájem
neizomorfní konečné modely dané velikosti.
Implementovali jsme metodu MACE, některé její známé modifikace
a také několik nových modifikací, jenž, pokud je nám
známo, dosud nikdo nezkoušel.
Domníváme se tedy, že zadání diplomové práce bylo splněno.

Ve srovnání s ostatními programy pro hledání modelů si program \crossbow{}
nevede špatně. Experimenty z minulé kapitoly však naznačují,
že program implementující metodu z dokazovače iProver společně
s metodou MACE by si mohl vést ještě lépe\footnote{K volbě
vhodné metody pro daný problém lze použít strojové učení.}.

Řada modifikací metody MACE neměla v našem experimentálním
srovnání prakticky žádný efekt. To se týká nejen nově představených
modifikací jako odzplošťování a přidávání redundantních klauzulí,
ale i dříve známých modifikací jako definic termů nebo inference sort
(inference sort výrazně pomohla pouze řešiči Gecode v kategorii NLP).
Naopak velký přínos měla již dříve známá modifikace rozdělování klauzulí.

I přesto, že jsme se při návrhu kódování klauzulí
do podmínek řešiče Gecode snažili omezit množství a velikost podmínek,
spotřebovával řešič Gecode více paměti než SAT řešiče.
Bylo by tedy vhodné přijít s kódováním,
jenž je paměťově šetrnější.
Kromě problémů se spotřebou paměti byl
řešič Gecode obvykle i pomalejší než SAT řešiče.
K vyšší rychlosti by možná pomohla
v řešiči Gecode vestavěná redukce symetrií LDSB \cite{ldsb}.
Bohužel se nám nepodařilo vymyslet způsob,
jak ji použít, aniž by došlo k nárůstu
množství proměnných a podmínek.

Při vývoji programu \crossbow{} jsme nepočítali s tím,
že některé modely budou mít po uložení do formátu TPTP
velikost několik gigabajtů. To se projevilo zejména v kategorii
NLP, kde program \crossbow{} některé modely našel,
ale nestihl je v daném časovém limitu celé vypsat.
Tuto nepříjemnost můžeme napravit tak, že do
souborů s modely přestaneme ukládat hodnoty buněk,
jenž lze odvodit propagací.

% Pokud to nepomůže, můžeme kód pro vypisování
% modelů, jenž se snaží formule hezky naformátovat,
% nahradit kódem, který žádné formátování nedělá.

Kromě metody MACE jsme
věnovali značnou pozornost i metodě SEM.
Její implementace, program Mace4, si však v našem experimentálním
srovnání příliš dobře nevedla. Podle našeho názoru
je hlavním důvodem absence učení klauzulí.
Námětem na další práci je tedy implementace metody SEM
s lepší propagací pro SEM, s učením klauzulí
a případně i se dvěma sledovanými literály.
Takto implementovanou metodu SEM můžeme chápat jako krok
mezi DPLL s učením klauzulí a metodami Model Evolution a SGGS.

% Takto implementovaná metoda SEM bude mít silnější, leč
% pomalejší, propagaci, než je jednotková propagace v SAT řešičích.

% Idea, jak rozšířit učení klauzulí na SEM, je poměrně triviální
% (zajímavější je to s efektivní implementací).
%
% Při konfliktu máme klauzuli/podmínku, kde jsou všechny literály false.
% Vezmeme plochou klauzuli (normálně by zploštění přidalo
% proměnné, jenže my známe hodnoty všech buněk, tudíž můžeme
% za tyto proměnné dosadit hodnoty tak, že dostaneme plochou klauzuli,
% jenž má všechny literály false, a rovněž neobsahuje proměnné).
%
% Nyní z této ploché klauzule budeme odrezolvovávat buňky
% z aktuální rozhodovací úrovně -- v pořadí podle stopy (trailu) --
% dokud tam nezbyde jedna buňka z aktuální RÚ.
% Předchůdce buňky je klauzule, která vynutila při propagaci její ohodnocení.
% Například při negativní propagaci z P(0, f(0), f(0)) | C1 a
% ~P(0, 1, f(0)) | C2, kde C1 a C2 obsahují literály jenž jsou false,
% je předchůdce ohodnocení f(0) != 1 klauzule
% f(0) != 1 | C1 | C2. Podobně demodulace se chová jako paramodulace.
% Například demodulace pomocí f(0) = 1 | C1, kde C1 obsahuje
% literály, jenž jsou false, musí do demodulované klauzule přidat
% i literály C1 (jedná se tedy o paramodulaci, která je řízena
% aktuálním částečným ohodnocením -- provádíme ji pouze, když jsou
% všechny literály kromě jednoho false). Kdybychom
% ignorovali C1 nebo C2, tak by naučené klauzule byly
% platné pouze pro aktuální částečné ohodnocení (nebo jeho rozšíření).
%
% S takto získanou klauzulí pak můžeme provést backjump standardním způsobem.
%
% Poznamenejme, že u SEMu jsou klauzule, jenž zaručují, že buňka
% má alespoň jednu hodnotu a nemá dvě nebo více hodnot, implicitní.
% Při rezoluci si je tedy musíme zkonstruovat.
%
% Zplošťovat není třeba celou konfliktní klauzuli, stačí jen
% buňky, co chceme odrezolvovat. Alternativně bychom místo rezoluce
% mohli zkusit nějakou jinou metodu, pak by v některých případech
% zplošťování nemuselo být třeba vůbec.



% V našich experimentech měly hledače modelů pouze 2 minuty
% na vyřešení problému. Je otázkou, jak by se změnily
% výsledky, kdybychom časový limit zvedli na několik hodin?
%
% Hlavní otázkou je, jestli nenastává problém horizontu.

% Další vylepšení lze čerpat ze SAT řešičů, SMT řešičů
% a řešičů omezujících podmínek.
% Příkladem je deaktivování naučených kauzulí nebo SEM s teoriemi
% (například lineární aritmetikou).
